%\documentclass[handout]{beamer} 
\documentclass[t,12pt,numbers,fleqn]{beamer}
%\documentclass[ignorenonframetext]{beamer}

\newif\ifquestions
%\questionstrue
\questionsfalse

\usepackage{pgfpages} 
\usepackage{hyperref}
\hypersetup{colorlinks=true,
    linkcolor=blue,
    citecolor=blue,
    filecolor=blue,
    urlcolor=blue,
    unicode=false}
\urlstyle{same}

\usepackage{booktabs}
\usepackage{multirow}

\bibliographystyle{plain}

%\usetheme{Iimenau}

\useoutertheme{split} %so the footline can be seen, without needing pgfpages

%\pgfpagesuselayout{resize to}[letterpaper,border shrink=5mm,landscape]  %if this is uncommented, the hyperref links do not work

\mode<presentation>{}

\input{../def-beamer}

\newcommand{\topic}{17 Math Review Plus MIS Example}

%Title page information for 1D04 lectures slides

% Define year specific parameters - used in title page and footer

\newcommand{\season}{Fall} %use to switch between Winter and Fall
\newcommand{\instructor}{Dr.~Spencer Smith} %use to switch instructor
\newcommand{\instructSmall}{Dr.~Smith}
\newcommand{\yr}{2019}
\newcommand{\courseCode}{CAS 741, CES 741}
\newcommand{\courseTitle}{Development of Scientific Computing Software}

%\setbeamerfont{structure}{series=\bfseries}
%\usefonttheme[stillsansseriftext,stillsansserifmath]{serif}
\setbeamertemplate{navigation symbols}{} 
\setbeamertemplate{itemize item}[ball]

\title{
  {\normalsize \bf 
    \borange{\courseCode~(\courseTitle)\\ \season~\yr}}\\[2ex]
  {\Large \bf \topic}}

\author[Smith]{\instructor}

\institute{
  Faculty of Engineering,
  McMaster University}

\date{
\today
%January 2011\\
\bc
  \includegraphics[scale = 0.2, keepaspectratio]
  {../mcmaster-logo-full-color.jpg}
\ec
}

\renewcommand{\borange}[1] %orange is too hard to read
{
   \bred{#1}
}

\begin{document}

\input{../footline}

%%%%%%%%%%%%%%%%%%%%%%%%%%%%%%%%%%%%%%%%%%%%%%%%%%%%%%

\begin{frame}
\frametitle{Math Review}

\bi
\item Administrative details
\item Questions?
\item SRS feedback
\item Math review introduction
\item Review of sets, relations and functions
\item Review of logic
\item Review of types, sets, sequence and tuples
\item Multiple assignment statement
\item Conditional rules
\item Finite State Machines
\item Example MIS: 2D Data
\ei
\end{frame}

%%%%%%%%%%%%%%%%%%%%%%%%%%%%%%%%%%%%%%%%%%%%%%%%%%%%%%

\begin{frame}
\frametitle{Administrative Details}

\bi
\item SRS grades on Avenue
\item If follow feedback, final doc grades will be higher
\item GitHub issues for colleagues
\bi
\item Assigned 1 colleague (see \texttt{Repos.xlsx} in repo)
\item Provide at least 3 issues on their MG
\item Grading as before
\item Due by Thursday, Nov 8, 11:59 pm
\ei
\item MG marking scheme in Avenue
\item MIS marking scheme in Avenue
\item Updated MIS template in CAS 741 repo (soon)
\ei

\end{frame}

%%%%%%%%%%%%%%%%%%%%%%%%%%%%%%%%%%%%%%%%%%%%%%%%%%%%%%

\begin{frame}
\frametitle{Administrative Details: Deadlines}
~\newline
\begin{tabular}{l l l}
\textbf{MIS Present} & Week 10 & Week of Nov 12\\
\textbf{MIS} & Week 11 & Nov 19\\
Unit VnV or Impl.\ Present & Week 12 & Week of Nov 26\\
Unit VnV Plan & Week 13 & Dec 3\\
Final Doc & Week 14 & Dec 10\\
\end {tabular}

\end{frame}

%%%%%%%%%%%%%%%%%%%%%%%%%%%%%%%%%%%%%%%%%%%%%%%%%%%%%%

\begin{frame}
\frametitle{Administrative Details: Presentation Schedule}

\bi
\item MIS Present
\bi
\item \textbf{Wednesday: Malavika, Robert}
\item \textbf{Friday: Hanane,  Jennifer}
\ei
\item Unit VnV Plan or Impl.\ Present
\bi
\item Wednesday: Brooks, Vajiheh
\item Friday: Olu, Karol
\ei
\ei

\end{frame}

%%%%%%%%%%%%%%%%%%%%%%%%%%%%%%%%%%%%%%%%%%%%%%%%%%%%%%

\begin{frame}
\frametitle{Questions?}
\begin{itemize}
\item Questions about Module Guides?
\item Questions about MIS?
\end{itemize}
\end{frame}

%%%%%%%%%%%%%%%%%%%%%%%%%%%%%%%%%%%%%%%%%%%%%%%%%%%%%%

\begin{frame}
  
\frametitle{SRS Feedback: Relationships Between Parts}

\begin{figure}[H]
  \includegraphics[scale=0.59]{../Figures/RelationsBetweenTM_GD_IM_DD_A.pdf}
\end{figure}

\end{frame}

%%%%%%%%%%%%%%%%%%%%%%%%%%%%%%%%%%%%%%%%%%%%%%%%%%%%%%

\begin{frame}

\frametitle{SRS Feedback}

\begin{itemize}
\item \texttt{*.DS\_Store} should be in \texttt{.gitignore}
\item \LaTeX{} and formatting rules
\begin{itemize}
\item Variables are italic, everything else not, includes subscripts (link to
  document)
\bi
\item \href{https://physics.nist.gov/cuu/pdf/typefaces.pdf}{Conventions}
\item Watch out for implied multiplication
\ei
\item Use BibTeX
\item Use cross-referencing
\end{itemize}
\item Grammar and writing rules
\begin{itemize}
\item Acronyms expanded on first usage (not just in table of acronyms)
\item ``In order to'' should be ``to''
\end{itemize}
\end{itemize}
\end{frame}

%%%%%%%%%%%%%%%%%%%%%%%%%%%%%%%%%%%%%%%%%%%%%%%%%%%%%%

\begin{frame}

\frametitle{SRS Feedback on Applying Template}

\begin{itemize}
\item Difference between physical and software constraints
\item Properties of a correct solution means \emph{additional} properties, not
  a restating of the requirements (may be ``not applicable'' for your problem).
  If you have a table of output constraints, then these are properties of a
  correct solution.
\item Assumptions have to be invoked somewhere
\item ``Referenced by'' implies that there is an explicit reference
\item Think of traceability matrix, list of assumption invokations and list of
  reference by fields as automatically generatable
\item If you say the format of the output (plot, table etc), then your
  requirement could be more abstract
\item For families the notion of binding time should be introduced
\item Think of families as a library, not as a single program
\end{itemize}
\end{frame}

%%%%%%%%%%%%%%%%%%%%%%%%%%%%%%%%%%%%%%%%%%%%%%%%%%%%%%

\begin{frame}
\frametitle{Mathematical Review: Introduction}
\begin{itemize}
\item The material in these slides should hopefully be review, or
  reasonably easy to pick up
\item Shows the simple mathematics that can be used to build your MIS
\item Shows a syntax that you can use
\item The presentation follows \cite{HoffmanAndStrooper1995} (Chapter
  3) and \cite{GriesAndSchneider1993}
\end{itemize}
\end{frame}

%%%%%%%%%%%%%%%%%%%%%%%%%%%%%%%%%%%%%%%%%%%%%%%%%%%%%%

\begin{frame}
\frametitle{Sets, Relations and Functions}
\begin{itemize}
\item A set is an unordered collection of elements
\item A binary relation is a set of ordered pairs
\item A function is a relation in which each element in the domain
  appears exactly once as the first component in the ordered pair
\end{itemize}
\end{frame}

%%%%%%%%%%%%%%%%%%%%%%%%%%%%%%%%%%%%%%%%%%%%%%%%%%%%%%

\begin{frame}
\frametitle{Sets}
\begin{itemize}
\item An element either belongs to a set or it does not
\item $x \in S$ versus $x \notin S$
\item Defining a set
\begin{itemize}
\item Enumerate $\{ x_1, x_2, x_3, ..., x_n \}$
\item Logical condition (rule) $\{x | p(x) \}$
\item Notation from \cite{GriesAndSchneider1993} $\{x: X | R : E \}$
\item An integer range $[2 .. 4] = \{2, 3, 4\}$, $[7 .. 4] = \{\}$
\end{itemize}
\item Examples
\begin{itemize}
\item $S = \{ 1, 7, 6 \}$
\item $S = \{ x |$ $x$ is an integer between $1$ and $4$ inclusive
  $\}$
\item $S = \{ x: \mathbb{N} |$ $0 \leq x \leq 4$ : $x$ $\}$
%\item Clarify that there is no notion of ordering in sets
\end{itemize}
\item Does $\{ 1, 7, 6 \} = \{ 7, 1, 6 \}$?
\end{itemize}
\end{frame}

%%%%%%%%%%%%%%%%%%%%%%%%%%%%%%%%%%%%%%%%%%%%%%%%%%%%%%

\begin{frame}
\frametitle{Relations}
\begin{itemize}
\item Let $<x, y>$ denote an ordered pair
\begin{itemize}
\item $dom(R) = \{x | <x, y> \in R\}$
\item $ran(R) = \{y | <x, y> \in R\}$
\end{itemize}
\item Defining a relation
\begin{itemize}
\item Enumerate $\{ <0, 1>, <0, 2>, <2, 3> \}$
\item Rule $\{ <x, y>|$ $x$ and $y$ are integers and $x < y \}$
\item $\{ x, y: \mathbb{N} |$ $x < y$ : $<x, y> \}$
\end{itemize}
\end{itemize}
\end{frame}

%%%%%%%%%%%%%%%%%%%%%%%%%%%%%%%%%%%%%%%%%%%%%%%%%%%%%%

\begin{frame}
\frametitle{Functions}
\begin{itemize}
\item Let $<x, y>$ denote an ordered pair
\item Each element of the domain is associated with a unique element
  of the range
\item Defining a function
\begin{itemize}
\item Enumerate $\{ <0, 1>, <1, 2>, <2, 3> \}$
\item Rule $\{ <x, y> |$ $x$ and $y$ are integers and $y = x^2 \}$
\end{itemize}
\item Notation
\begin{itemize}
\item $f(a) = b$ means $<a, b> \in f$
\item $f(x) = x^2$
\item $f: T_1 \rightarrow T_2$
\item $\{ <<x_1, x_2>, y> |$ $x_1$, $x_2$ are integers and $y = x_1 + x_2 \}$
\end{itemize}
\item Is $\{ <0, 1>, <0, 2>, <2, 3> \}$ a function?
\item Is $\{ <x, y> |$ $x$ and $y$ are integers and $y^2 = x\}$?
\end{itemize}
\end{frame}

%%%%%%%%%%%%%%%%%%%%%%%%%%%%%%%%%%%%%%%%%%%%%%%%%%%%%%

\begin{frame}
\frametitle{Logic}
\begin{itemize}
\item A logical expression is a statement whose truth values can be
  determined ($6 < 7$?)
\item Truth values are either \emph{true} or \emph{false}
\item Complex expressions are formed from simpler ones using logical
  connectives ($\neg$, $\wedge$, $\vee$, $\rightarrow$,
  $\leftrightarrow$)
\item Truth tables
\item Evaluation
\begin{itemize}
\item Decreasing order of precedence: $\neg$, $\wedge$, $\vee$,
$\rightarrow$, $\leftrightarrow$
\item Evaluate from left to right
\item Use rules of boolean algebra
\end{itemize}
\end{itemize}
\end{frame}

%%%%%%%%%%%%%%%%%%%%%%%%%%%%%%%%%%%%%%%%%%%%%%%%%%%%%%

\begin{frame}
\frametitle{Quantifiers}
\begin{itemize}
\item Variables are often used inside logical expressions
\item Variables have types
\item A type is a set of values from which the variable can take its value
\item Often quantify a logical expression over a given variable
\begin{itemize}
\item Universal quantification
\item Existential quantification
\end{itemize}
\end{itemize}
\end{frame}

%%%%%%%%%%%%%%%%%%%%%%%%%%%%%%%%%%%%%%%%%%%%%%%%%%%%%%

\begin{frame}
\frametitle{Quantifiers Continued}

\begin{itemize}
\item Prefer \cite[p.\ 143]{GriesAndSchneider1993} notation for quantification
  (and set building)
\item $(*x: X | R : P)$ means application of the operator $*$ to the values $P$ for
all $x$ of type $X$ for which range $R$ is true.  In the above equations, the
$*$ operators include $\forall$, $\exists$ and $+$ are used
\item Example on next slide for rank function specification
\end{itemize}

\end{frame}

%%%%%%%%%%%%%%%%%%%%%%%%%%%%%%%%%%%%%%%%%%%%%%%%%%%%%%

\begin{frame}
%\frametitle{Quantifiers Example for Rank Function}

\noindent $\mbox{rank}(a, A): \mathbb{R} \times \mathbb{R}^n \rightarrow \mathbb{R}$\newline
$\mbox{rank}(a, A) \equiv \mbox{avg}(\mbox{indexSet}(a, \mbox{sort}(A)))$\newline

\noindent $\mbox{indexSet}(a, B): \mathbb{R} \times \mathbb{R}^n \rightarrow \mbox{ set of }
\mathbb{N}$\newline
$\mbox{indexSet}(a, B) \equiv \{j: \mathbb{N} | j \in [1..|B|]
\wedge B_j = a : j \}$\newline

\noindent $\mbox{sort}(A): \mathbb{R}^n \rightarrow \mathbb{R}^n$\newline
$\mbox{sort}(A) \equiv B: \mathbb{R}^n, \mbox{ such that }$\newline
$\forall (a: \mathbb{R} | a \in A : \exists(b: \mathbb{R} | b \in B: b = a)
\wedge \mbox{count}(a, A) = \mbox{count}(b, B)) \wedge \forall (i: \mathbb{N} | i \in [1..|A|-1] : B_i \leq B_{i+1})$\newline

\noindent $\mbox{count}(a, A): \mathbb{R} \times \mathbb{R}^n \rightarrow \mathbb{N}$\newline
$\mbox{count}(a, A) \equiv + (x: \mathbb{R} | x \in A \wedge x = a : 1)$\newline

\noindent $\mbox{avg}(C): \mbox{ set of } \mathbb{N} \rightarrow \mathbb{R}$\newline
$\mbox{avg}(C) \equiv + (x: \mathbb{N} | x \in C : x) / |C|$\newline

\end{frame}

%%%%%%%%%%%%%%%%%%%%%%%%%%%%%%%%%%%%%%%%%%%%%%%%%%%%%%

\begin{frame}
\frametitle{Quantifiers Continued}
\begin{itemize}
\item Bound variables appear in the scope of the quantifier
\item Free variables are not bound to any quantifier
\item Free variables in an expression often  mean that we cannot determine the truth value of the expression
\end{itemize}
\end{frame}

%%%%%%%%%%%%%%%%%%%%%%%%%%%%%%%%%%%%%%%%%%%%%%%%%%%%%%

\begin{frame}
\frametitle{Types, Sets, Sequence and Tuples}
\begin{itemize}
\item A type is a set of values, so any precisely defined set is a type
\item Primitive types are often integer, boolean, character, string
  and real
\item Types can include functions ($T_1 \rightarrow T_2$)
\item User-defined types
\begin{itemize}
\item The set of values has to be given
\item Often use type constructors
\end{itemize}
\item Useful type constructors
\begin{itemize}
\item Set
\item Sequence
\item Tuple
\end{itemize}
\end{itemize}
\end{frame}

%%%%%%%%%%%%%%%%%%%%%%%%%%%%%%%%%%%%%%%%%%%%%%%%%%%%%%

\begin{frame}
\frametitle{Types}
\begin{itemize}
\item Specify the type of a variable
\begin{itemize}
\item $x_1, x_2, ..., x_n : T$
\item $x: integer$
\item $a, b, c: string$
\end{itemize}
\item Type definition
\begin{itemize}
\item $T = d$
\item $float = real$
\item $colour = \{red, white, blue \}$
\item $testtype = \{uniaxial, biaxial, shear \}$
\item $x: testtype$
\item $motionT = \{ forward, backward, stop \}$
\end{itemize}
\end{itemize}
\end{frame}

%%%%%%%%%%%%%%%%%%%%%%%%%%%%%%%%%%%%%%%%%%%%%%%%%%%%%%

\begin{frame}
\frametitle{Primitive Types}
\begin{itemize}
\item Integer
\begin{itemize}
\item $\{ ... -2, -1, 0, 1, 2, ... \}$
\item $+, -, \times, /$
\item $=, \neq$
\item $<, \leq, \geq, >$
\end{itemize}
\item Real
\begin{itemize}
\item $\{ all~real~numbers \}$
\item $+, -, \times, /, \sin (), \cos (), \exp ()$ etc.
\item $=, \neq$
\item $<, \leq, \geq, >$
\end{itemize}
\end{itemize}
\end{frame}

%%%%%%%%%%%%%%%%%%%%%%%%%%%%%%%%%%%%%%%%%%%%%%%%%%%%%%

\begin{frame}
\frametitle{Primitive Types Continued}
\begin{itemize}
\item Boolean type
\begin{itemize}
\item $\{ true, false \}$
\item $\neg$, $\wedge$, $\vee$, $\rightarrow$, $\leftrightarrow$
\end{itemize}
\item Char type
\begin{itemize}
\item Set of ASCII characters
\item Character values appear in quotes $'a'$, $'b'$, $'c'$, etc.
\item $=, \neq$
\end{itemize}
\end{itemize}
\end{frame}

%%%%%%%%%%%%%%%%%%%%%%%%%%%%%%%%%%%%%%%%%%%%%%%%%%%%%%

\begin{frame}
\frametitle{Primitive Types Continued}
\begin{itemize}
\item String type
\begin{itemize}
\item All finite sequences of characters
\item String constants are in double quotes $''abc''$
\item $s[i..j]$ is the substring of $s$ from position $i$ to position $j$
\item $s_1 || s_2$ concatenates strings $s_1$ and $s_2$
\item $=, \neq$ for is equal and not equal
\item $\in, \notin$ for is member and not a member
\item $s[i]$ is the $i$th character of $s$
\item $|s|$ is the length of $s$
\item Positions in strings are zero relative
\end{itemize}
\end{itemize}
\end{frame}

%%%%%%%%%%%%%%%%%%%%%%%%%%%%%%%%%%%%%%%%%%%%%%%%%%%%%%

\begin{frame}
\frametitle{Sets}
\begin{itemize}
\item A set is an unordered collection of elements of the same type
\item Declare a set of type $T$ as $set~of~T$
\item Example
\begin{itemize}
\item $T = set~of~ \{red, green, blue \}$ defines type $T$ as the power set of $\{red, green, blue \}$
\item $x: set~of~integer$
\end{itemize}
\item What are some possible values for $x: set~of~integer$?
\end{itemize}
\end{frame}

%%%%%%%%%%%%%%%%%%%%%%%%%%%%%%%%%%%%%%%%%%%%%%%%%%%%%%

\begin{frame}
\frametitle{Operations on Sets}
\begin{itemize}
\item $\cup$ union
\item $\cap$ intersection
\item $-$ difference
\item $\times$ Cartesian product
\item $=, \neq$ equal, not equal
\item $\in, \notin$ member, non-member
\item $|s|$ size of set $s$
\end{itemize}
\end{frame}

%%%%%%%%%%%%%%%%%%%%%%%%%%%%%%%%%%%%%%%%%%%%%%%%%%%%%%

\begin{frame}
\frametitle{Sequences}
\begin{itemize}
\item A sequence is an ordered collection of elements of the same type
\begin{itemize}
\item Elements can occur more than once
\item Sometimes referred to as a list
\item Similar to an array
\end{itemize}
\item Declare a sequence of type $T$ by $sequence~of~T$
\item $< x_0, x_1, ..., x_n >$ for $n \geq 0$ for a sequence with elements $x_0, x_1, ..., x_n$
\item $< >$ is the empty sequence
\item Position in a sequence is zero relative
\end{itemize}
\end{frame}

%%%%%%%%%%%%%%%%%%%%%%%%%%%%%%%%%%%%%%%%%%%%%%%%%%%%%%

\begin{frame}
\frametitle{Sequences Continued}
\begin{itemize}
\item Examples
\begin{itemize}
\item $T = sequence~of~\{ red, green, blue \}$ defines the type $T$ as the set of all sequences of elements from
$\{red, green, blue \}$
\item $x: sequence~of~integer$
\end{itemize}
\item Fixed-length sequence of type $T$ with length $l$
\begin{itemize}
\item $sequence~[l]~of~T$
\item $l$ is a positive integer
\item $sequence~[l_1, l_2, ..., l_n ]~of~T$ is a shorthand for $sequence~[l_1]~of~sequence~[l_2]~of ... sequence~
[l_n]~ of~T$
\end{itemize}
\end{itemize}
\end{frame}

%%%%%%%%%%%%%%%%%%%%%%%%%%%%%%%%%%%%%%%%%%%%%%%%%%%%%%

\begin{frame}
\frametitle{Operations on Sequences}
\begin{itemize}
\item $s[i..j]$ is the subsequence of $s$ from position $i$ to position $j$
\item $s[i..j]$ is undefined if $i \notin [0..|s|-1] \vee j \notin [0 .. |s|-1]$
\item $s_1 || s_2$ concatenates sequences $s_1$ and $s_2$
\item $=, \neq$ for is equal and not equal
\item $\in, \notin$ for is member and not a member
\item $s[i]$ is the $i$th element of $s$
\item $s[i]$ is undefined if $i \notin [0..|s|-1]$
\item $|s|$ is the length of $s$
\item A string is a sequence of characters
\end{itemize}
\end{frame}

%%%%%%%%%%%%%%%%%%%%%%%%%%%%%%%%%%%%%%%%%%%%%%%%%%%%%%

\begin{frame}
\frametitle{Tuples}
\begin{itemize}
\item A tuple is a collection of elements of possibly different types
\item Each tuple has one or more fields
\item Each field has a unique identifier called the field name
\item Similar to a record or a structure
\item To declare a tuple use
\begin{itemize}
\item $tuple~of~(f_1: T_1, f_2: T_2, ..., f_n: T_n )$ with $n \geq 1$
\item $f_i$ is the name of the $i$th field
\item $T_i$ is the type of the $i$th field
\item $tuple~of~(f_1, f_2, ..., f_n: T )$ if all fields are of the same type
\end{itemize}
\end{itemize}
\end{frame}

%%%%%%%%%%%%%%%%%%%%%%%%%%%%%%%%%%%%%%%%%%%%%%%%%%%%%%

\begin{frame}
\frametitle{Example Tuples}
\begin{itemize}
\item Examples
\begin{itemize}
\item $pair = tuple~of~(id: integer, val: string)$
\item $experimentT = tuple~of~(b_{cond}: bcondT, control: controlT)$
\end{itemize}
\item Define the value of a tuple by using an expression of the form
\begin{itemize}
\item $<x_1, x_2, ..., x_n >$
\item $<4, ''cat''>$ is a value of type pair
\end{itemize}
\end{itemize}
\end{frame}

%%%%%%%%%%%%%%%%%%%%%%%%%%%%%%%%%%%%%%%%%%%%%%%%%%%%%%

\begin{frame}
\frametitle{Operations on Tuples}
\begin{itemize}
\item $=, \neq$ equal, not equal
\item $t.f$ is the value of field $f$ of tuple $t$
\end{itemize}
\end{frame}

%%%%%%%%%%%%%%%%%%%%%%%%%%%%%%%%%%%%%%%%%%%%%%%%%%%%%%

\begin{frame}
\frametitle{Using Type Constructors}
\begin{itemize}
\item $bcondT = \{ uniaxial, biaxial, multiaxial, shear \}$
\item $controlT = \{ load\_controlled, displacement\_controlled \}$
\item $experimentT = tuple~of~(b_{cond}: bcondT, control: controlT)$
\item $experiment: experimentT$
\item $directionT = \{clockwise, counterclockwise \}$
\item $powerT = [MIN\_POWER ... MAX\_POWER]$
\item $motorT = tuple~of~ (powerOn: Boolean,~direction: directionT,~powerLevel: powerT)$
\end{itemize}
\end{frame}

%%%%%%%%%%%%%%%%%%%%%%%%%%%%%%%%%%%%%%%%%%%%%%%%%%%%%%

\begin{frame}
\frametitle{Multiple Assignment Statement}
\begin{itemize}
\item $v_1, v_2, ..., v_n := e_1, e_2, ...,, e_n$ with $n \geq 1$
\item The $v_i$s are distinct variables and each $e_i$ is an expression of the same type as $v_i$
\item Compute the values of all the expression $e_i$ and then assign these values simultaneously
\item Example
\begin{itemize}
\item $x, y := 0, 10$
\item $x, y := 10, x$
\item $x, y := y, x$
\end{itemize}
\item Convenient for defining the meaning of pieces of code
\item Use as a function on the state space of a program
\end{itemize}
\end{frame}

%%%%%%%%%%%%%%%%%%%%%%%%%%%%%%%%%%%%%%%%%%%%%%%%%%%%%%

\begin{frame}
\frametitle{Conditional Rules}
\begin{itemize}
\item $(c_1 \Rightarrow r_1 | ... | c_n \Rightarrow r_n )$, where $n \geq 1$
\item $c_i$s are the logical expressions
\item $r_i$s are the rules
\item $c_i \Rightarrow r_i$ is the $i$th component of the rule
\item The first $c_i$ that evaluates to true applies rule $r_i$
\item If no condition is true then the conditional rule is undefined
\end{itemize}
\end{frame}

%%%%%%%%%%%%%%%%%%%%%%%%%%%%%%%%%%%%%%%%%%%%%%%%%%%%%%

\begin{frame}
\frametitle{Uses of Conditional Rules}
\begin{itemize}
\item To define the value of a function
\item $min(x,y) = (x \leq y \Rightarrow x | x > y \Rightarrow y)$
\item To define the meaning of a program
\begin{itemize}
\item If $(x < y)$ then $z := x$ else $z := y$
\item $(x < y \Rightarrow z := x | x \geq y \Rightarrow z:= y)$
\item $(x < y \Rightarrow x, y := x, y | x \geq y \Rightarrow x, y := y, x)$
\end{itemize}
\item Conditional rules can be expressed in tables
\end{itemize}
\end{frame}

%%%%%%%%%%%%%%%%%%%%%%%%%%%%%%%%%%%%%%%%%%%%%%%%%%%%%%

\begin{frame}
\frametitle{Finite State Machines}
\begin{itemize}
\item A FSM is a tuple $(S, s_0, I, O_E, O_O, T, E, C)$ where
\item $S$ is a finite set of states
\item $s_0$ is the initial state in $S$ $(s_0 \in S)$
\item $I$ is a finite set of inputs
\item $T: S \times I \rightarrow S$ is the transition function
\item $O_E$ is a finite set of event outputs
\item $E: S\times I \rightarrow O_E$ is the event output
\item $O_C$ is a finite set of condition outputs
\item $C: S \rightarrow O_C$ is the condition output
\end{itemize}
\end{frame}

%%%%%%%%%%%%%%%%%%%%%%%%%%%%%%%%%%%%%%%%%%%%%%%%%%%%%%

\begin{frame}
\frametitle{Example 2D Data}
\begin{itemize}
\item
  \href{https://gitlab.cas.mcmaster.ca/smiths/se2aa4_cs2me3/blob/master/Assignments/A2/A2.pdf}
  {Problem Description}
\item \href{https://gitlab.cas.mcmaster.ca/smiths/se2aa4_cs2me3/tree/master/Assignments/A2/A2Soln/src}{Source Code}
\end{itemize}
\end{frame}

%%%%%%%%%%%%%%%%%%%%%%%%%%%%%%%%%%%%%%%%%%%%%%%%%%%%%%

\begin{frame}[allowframebreaks]
\frametitle{References}

\bibliography{../../ReferenceMaterial/References}

\end{frame}

%%%%%%%%%%%%%%%%%%%%%%%%%%%%%%%%%%%%%%%%%%%%%%%%%%%%%%

\end{document}