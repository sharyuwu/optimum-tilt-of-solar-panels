%\documentclass[handout]{beamer} 
\documentclass[t,12pt,numbers,fleqn]{beamer}
%\documentclass[ignorenonframetext]{beamer}

\newif\ifquestions
%\questionstrue
\questionsfalse

\usepackage{pgfpages} 
\usepackage{hyperref}
\hypersetup{colorlinks=true,
    linkcolor=blue,
    citecolor=blue,
    filecolor=blue,
    urlcolor=blue,
    unicode=false}
\urlstyle{same}

\usepackage{booktabs}
\usepackage{hhline}
\usepackage{multirow}
\usepackage{multicol}
\usepackage{array}
\usepackage{listings}
\usepackage{bm}
\usepackage{colortbl}
\usepackage{bnf}

\usepackage{tikz}
\usetikzlibrary{positioning}

\newcommand{\colA}{2.1cm}
\newcommand{\colB}{6.9cm}
\newcommand{\colC}{1.1cm} %do not need this column if binding time is not listed
\newcommand{\colAwidth}{0.15\textwidth}
\newcommand{\colBwidth}{0.7\textwidth}

\newcounter{temp}
\setcounter{temp}{0}

\bibliographystyle{plain}

%\usetheme{Iimenau}

\useoutertheme{split} %so the footline can be seen, without needing pgfpages

%\pgfpagesuselayout{resize to}[letterpaper,border shrink=5mm,landscape]  %if this is uncommented, the hyperref links do not work

\mode<presentation>{}

\input{../def-beamer}

\newcommand{\topic}{05 Program Families}

%Title page information for 1D04 lectures slides

% Define year specific parameters - used in title page and footer

\newcommand{\season}{Fall} %use to switch between Winter and Fall
\newcommand{\instructor}{Dr.~Spencer Smith} %use to switch instructor
\newcommand{\instructSmall}{Dr.~Smith}
\newcommand{\yr}{2019}
\newcommand{\courseCode}{CAS 741, CES 741}
\newcommand{\courseTitle}{Development of Scientific Computing Software}

%\setbeamerfont{structure}{series=\bfseries}
%\usefonttheme[stillsansseriftext,stillsansserifmath]{serif}
\setbeamertemplate{navigation symbols}{} 
\setbeamertemplate{itemize item}[ball]

\title{
  {\normalsize \bf 
    \borange{\courseCode~(\courseTitle)\\ \season~\yr}}\\[2ex]
  {\Large \bf \topic}}

\author[Smith]{\instructor}

\institute{
  Faculty of Engineering,
  McMaster University}

\date{
\today
%January 2011\\
\bc
  \includegraphics[scale = 0.2, keepaspectratio]
  {../mcmaster-logo-full-color.jpg}
\ec
}

\renewcommand{\borange}[1] %orange is too hard to read
{
   \bred{#1}
}

\begin{document}

\input{../footline}

%%%%%%%%%%%%%%%%%%%%%%%%%%%%%%%%%%%%%%%%%%%%%%%%%%%%%%

\begin{frame}
\frametitle{Program Families}

\bi
\item Administrative details
\item Questions?
\item Finish up on SRS
\item Specification Qualities
\item Motivation
\item Proposed Family Methods
\item Family of Mesh Generators
\item Family of Linear Solvers
\item Family of Material Behaviour Models
\ei
\end{frame}

%%%%%%%%%%%%%%%%%%%%%%%%%%%%%%%%%%%%%%%%%%%%%%%%%%%%%%

\begin{frame}
\frametitle{Administrative Details}

\bi
%\item Problem statement should be clear on input and output
\item Presentations
\bi
\item VGA by default, ask if need adapter
\item Can use my laptop, but track pad is difficult to use
\ei
% \item Do NOT reproduce all of the cas 741 repo in your repo, just the blank
%   project template (moved to the top level)
%\item Use the same names as the original
%\item Delete example text from templates
\item \href{https://gitlab.cas.mcmaster.ca/smiths/cas741/blob/master/Repos.xlsx}{Repos.xlsx}
\item Domain experts - volunteers?
\item 80 columns in tex files
%\item Spell check
%\item Replace ``in order to'' by ``to''
%\item Use a \texttt{.gitignore} file
%\item Include the commit hash that closes the issue
%\item Close issues assigned to you
\item CA template now updated
\ei

\end{frame}

%%%%%%%%%%%%%%%%%%%%%%%%%%%%%%%%%%%%%%%%%%%%%%%%%%%%%%

\begin{frame}
\frametitle{Administrative Details: Report Deadlines}
~\newline
\begin{tabular}{l l l}
\textbf{SRS} & Week 06 & Oct 7\\
System VnV Plan & Week 08 & Oct 28\\
MG + MIS & Week 10 & Nov 25\\
Final Documentation & Week 14 & Dec 9\\
\end {tabular}

\bi
\item The written deliverables will be graded based on the repo contents as of
11:59 pm of the due date
\item If you need an extension, please ask
\item Two days after each major deliverable, your GitHub issues will be due
\item Domain expert code due 1 week after MIS deadline
\ei

\end{frame}

%%%%%%%%%%%%%%%%%%%%%%%%%%%%%%%%%%%%%%%%%%%%%%%%%%

\begin{frame}
\frametitle{Administrative Details: Presentations}

~\newline
\begin{tabular}{l l l}
\textbf{SRS Present} & Week 05 & Week of Sept 30\\
Syst.\ VnV Present & Week 07 & Week of Oct 21\\
MG + MIS Syntax Present & Week 9 & Week of Nov 4\\
MIS Semantics Present & Week 11 & Week of Nov 18\\
Unit VnV or Impl.\ Present & Week 12/13 & Week of Nov 28\\
\end {tabular}

\bi
\item Informal presentations with the goal of improving everyone's written
  deliverables
\item Domain experts and secondary reviewers (and others) will ask questions
\ei

\end{frame}

%%%%%%%%%%%%%%%%%%%%%%%%%%%%%%%%%%%%%%%%%%%%%%%%%%

\begin{frame}
\frametitle{Administrative Details: Presentation Schedule}

\bi
\item SRS (or CA) Present
\bi
\item \textbf{Monday: Deema, Sharon, Bo}
\item \textbf{Thursday: Sasha, Colin, Zhi}
\ei
\item Syst V\&V Plan Present
\bi
\item Monday: Deema, Peter
\item Thursday: Sharon, Ao
\ei
\item MG + MIS Syntax Present
\bi
\item Monday:  Deema, Bo
\item Thursday: Colin, Sasha
\ei
\item MIS Syntax + Semantics Present
\bi
\item Monday: Zhi, Peter
\item Thursday:  Sharon, Ao
\ei
\item Unit VnV Plan or Impl.\ Present
\bi
\item Monday: Bo, Sasha, Colin
\item Thursday: Zhi, Peter, Ao
\ei

\ei

\end{frame}

%%%%%%%%%%%%%%%%%%%%%%%%%%%%%%%%%%%%%%%%%%%%%%%%%%%%%%

\begin{frame}
\frametitle{Questions?}
\begin{itemize}
\item Questions about SRS?
\item Any questions on the 
  \href{https://gitlab.cas.mcmaster.ca/smiths/cas741/blob/master/BlankProjectTemplate/docs/SRS/SRS-Checklist.pdf}
  {SRS Checklist?}
\item Is $a = \frac{dv}{dt}$ a TM or a DD?
\end{itemize}
\end{frame}

%%%%%%%%%%%%%%%%%%%%%%%%%%%%%%%%%%%%%%%%%%%%%%%%%%%%%%

\begin{frame}
\frametitle{Kreyman and Parnas Five Variable Model}
\begin{itemize}
\item See \cite{KreymanAndParnas2002}
\item An alternative approach
\item Unfortunately the numerical algorithm is not hidden in the requirements specification
\item The analogy with real-time systems leads to some confusion
\end{itemize}
\end{frame}

%%%%%%%%%%%%%%%%%%%%%%%%%%%%%%%%%%%%%%%%%%%%%%%%%%%%%%

\begin{frame}
\frametitle{Examples}
\begin{itemize}
\item \href{https://github.com/smiths/swhs} {Solar Water Heating System}
\item
  \href{https://github.com/smiths/caseStudies/tree/master/CaseStudies/glass/docs/SRS}
  {GlassBR}
\end{itemize}
\end{frame}

%%%%%%%%%%%%%%%%%%%%%%%%%%%%%%%%%%%%%%%%%%%%%%%%%%%%%%

\begin{frame}
\frametitle{Specification Qualities}

\begin{itemize}

\item \structure{What are the important qualities for a specification?  What
    makes a specification a good specification?}

\end{itemize}

\end{frame}

%%%%%%%%%%%%%%%%%%%%%%%%%%%%%%%%%%%%%%%%%%%%%%%%%%%%%%

\begin{frame}
\frametitle{Specification Qualities}

\begin{itemize}
\item The qualities we previously discussed (usability, maintainability,
  reusability, verifiability etc.)
\item Clear, unambiguous,  understandable
\item Consistent
\item Complete
\begin{itemize}
\item Internal completeness
\item External completeness
\end{itemize}
\item Incremental
\item Validatable
\item Abstract
\item Traceable
\end{itemize}

Summarized in \cite[p.\ 406]{SmithAndKoothoor2016}

\end{frame}

%%%%%%%%%%%%%%%%%%%%%%%%%%%%%%%%%%%%%%%%%%%%%%%%%%%%%%

\begin{frame}
\frametitle{Clear, Unambiguous, Understandable}

\begin{itemize}

\item Specification fragment for a word-processor
\begin{itemize}
\item \structure{Selecting is the process of designating 
areas of the document that you want to 
work on. Most editing and formatting 
actions require two steps: first you 
select what you want to work on, 
such as text or graphics; then you 
initiate the appropriate action.}
\end{itemize}
\item What are the potential problems with this specification?
\begin{itemize}
\item<2-> {\alert{Can an area be scattered?}}
\item<2->{\alert{Can both text and graphics be selected?}}
\end{itemize}
\end{itemize}

\end{frame}

%%%%%%%%%%%%%%%%%%%%%%%%%%%%%%%%%%%%%%%%%%%%%%%%%%%%%%

\begin{frame}
\frametitle{Clear, Unambiguous, Understandable}

\begin{itemize}

\item Specification fragment from a real safety-critical system
\begin{itemize}
\item \structure{The message must be triplicated. The three
copies must be forwarded through three 
different physical channels. The receiver 
accepts the message on the basis of a 
two-out-of-three voting policy.}
\end{itemize}
\item What is a potential problems with this specification?
\begin{itemize}
\item<2-> {\alert{Can a message be accepted as soon as we receive 2 out of 3
      identical copies, or do we need to wait for receipt of the 3rd}}
\end{itemize}
\end{itemize}

\end{frame}

%%%%%%%%%%%%%%%%%%%%%%%%%%%%%%%%%%%%%%%%%%%%%%%%%%%%%%

\begin{frame}
\frametitle{Unambiguous, Validatable}

\begin{itemize}

\item Specification fragment for an end-user program
\begin{itemize}
\item \structure{The program shall be user friendly.}
\end{itemize}
\item What is a potential problems with this specification?
\begin{itemize}
\item<2-> {\alert{What does it mean to be user friendly?}}
\item<2-> {\alert{Who is a typical user?}}
\item<2-> {\alert{How would you measure success or failure in meeting this requirement?}}
\end{itemize}

\end{itemize}

\end{frame}

%%%%%%%%%%%%%%%%%%%%%%%%%%%%%%%%%%%%%%%%%%%%%%%%%%%%%%

\begin{frame}
\frametitle{Unambiguous, Validatable}

\begin{itemize}

\item Specification fragment for a linear solver
\begin{itemize}
\item \structure{Given $A$ and $b$, solve the linear system $A x = b$ for $x$,
    such that the error in any entry of $x$ is less than 5 \%.}
\end{itemize}
\item What is a potential problems with this specification?
\begin{itemize}
\item<2-> {\alert{Is $A$ constrained to be square?}}
\item<2-> {\alert{Can $A$ be singular?}}
\item<2-> {\alert{Even if the problem is made completely unambiguous, the requirement cannot be validated.}}
\end{itemize}

\end{itemize}

\end{frame}

%%%%%%%%%%%%%%%%%%%%%%%%%%%%%%%%%%%%%%%%%%%%%%%%%%%%%%

\begin{frame}
\frametitle{Consistent}

\begin{itemize}

\item Specification fragment for a word-processor
\begin{itemize}
\item \structure{The whole text should be kept in lines 
of equal length. The length is specified 
by the user. Unless the user gives an 
explicit hyphenation command, 
a carriage return should occur only 
at the end of a word.}
\end{itemize}
\item What is a potential problems with this specification?
\begin{itemize}
\item<2-> {\alert{What if the length of a word exceeds the length of the line?}}
\end{itemize}

\end{itemize}

\end{frame}

%%%%%%%%%%%%%%%%%%%%%%%%%%%%%%%%%%%%%%%%%%%%%%%%%%%%%%

\begin{frame}
\frametitle{Same Symbol/Term Different Meaning}

\begin{itemize}

\item \structure{Can you think of some symbols/terms that have different
    meanings depending on the context?}

\end{itemize}

\end{frame}

%%%%%%%%%%%%%%%%%%%%%%%%%%%%%%%%%%%%%%%%%%%%%%%%%%%%%%

\begin{frame}
\frametitle{Consistent}

\begin{itemize}

\item Language and terminology must be consistent within the specification
\item Potential problem with homonyms, for instance consider the symbol $\sigma$
\begin{itemize}
\item Represents standard deviation
\item Represents stress
\item Represents the Stefan-Boltzmann constant (for radiative heat transfer)
\end{itemize}
\item Changing the symbol may be necessary for consistency, but it could
  adversely effect understandability
\item Potential problem with synonyms
\begin{itemize}
\item Externally funded graduate students, versus eligible graduate students,
  versus non-VISA students
%ask who would think about graduate school?
\item Material behaviour model versus constitutive equation
\end{itemize}
\end{itemize}

\end{frame}

%%%%%%%%%%%%%%%%%%%%%%%%%%%%%%%%%%%%%%%%%%%%%%%%%%%%%%

\begin{frame}
\frametitle{Complete}

\begin{itemize}

\item Internal completeness
\begin{itemize}
\item The specification must define any new concept or terminology that it uses
\begin{itemize}
\item A glossary is helpful for this purpose
\end{itemize}
\end{itemize}
\item External completeness
\begin{itemize}
\item The specification must document all the needed requirements
\begin{itemize}
\item Difficulty: when should one stop?
\end{itemize}
\end{itemize}

\end{itemize}

\end{frame}

%%%%%%%%%%%%%%%%%%%%%%%%%%%%%%%%%%%%%%%%%%%%%%%%%%%%%%

\begin{frame}
\frametitle{Incremental}

\begin{itemize}

\item Referring to the specification process
\begin{itemize}
\item Start from a sketchy document and progressively add details
\item A document template can help with this
\end{itemize}
\item Referring to the specification document
\begin{itemize}
\item Document is structured and can be understood in increments
\item Again a document template can help with this
\end{itemize}

\end{itemize}

\end{frame}

%%%%%%%%%%%%%%%%%%%%%%%%%%%%%%%%%%%%%%%%%%%%%%%%%%%%%%

\begin{frame}
\frametitle{Traceable}

\begin{itemize}

\item Explicit links
\bi
\item Within document
\item Between documents
\ei
\item Use labels, cross-references, traceability matricies
\item Common sense suggests traceability improves maintainability
\item Shows consequence of change
\item Minimizes cost of recertification
\item Additional advantages
\bi
\item Program comprehension
\item Impact analysis
\item Reuse
\ei
\item \structure{Why is traceability important?} %design for change
\end{itemize}

\end{frame}

%%%%%%%%%%%%%%%%%%%%%%%%%%%%%%%%%%%%%%%%%%%%%%%%%%%%%%

\begin{frame}
\frametitle{Accuracy Versus Precision}

\begin{center}
 \includegraphics[width=1.0\textwidth]{../Figures/AccuracyPrecision_FromUniversityOfHawaiiAtManoa.png}
\end{center}

\structure{What is the distinction between accuracy and precision?}

\end{frame}

%%%%%%%%%%%%%%%%%%%%%%%%%%%%%%%%%%%%%%%%%%%%%%%%%%%%%%

%\hoffset=-.8in
\begin{frame}[plain, fragile]

\frametitle{Program Family Examples}

% if time add another program family example

\begin{tikzpicture}[remember picture,overlay]
\node [xshift=0cm,yshift=-3.cm] at (current page.center)
{
\includegraphics[width=1.2\textwidth]{../Figures/apple-mac-products-450x128.jpg}
};
\end{tikzpicture}

\begin{tikzpicture}[remember picture,overlay]
\node [xshift=0cm,yshift=1.cm] at (current page.center)
{
\includegraphics[width=0.8\textwidth]{../Figures/dodge-lineup.jpg}
};
\end{tikzpicture}

\end{frame}
\hoffset=0in

%%%%%%%%%%%%%%%%%%%%%%%%%%%%%%%%%%%%%%

\begin{frame}
\frametitle{Program Families}

\begin{itemize}

\item Can think of general purpose (or multi-purpose) SC software as a program
  family
\item Some examples of physical models are also appropriate for consideration as
  a family
\item A program family is a set of programs where it makes more sense to develop
  them together as opposed to separately
\item Analogous to families in other domains
\begin{itemize}
\item Automobiles
\item Computers
\item ...
\end{itemize}
\item Need to identify the commonalities
\item Need to identify the variabilities
\item Discussed in general in \cite{ClementsAndNorthrop2001,PohlEtAl2005}
\end{itemize}

\end{frame}

%%%%%%%%%%%%%%%%%%%%%%%%%%%%%%%%%%%%%%%%%%%%%%%%%%%%%%

\begin{frame}
\frametitle{Background}

\begin{itemize}

\item Program family idea since the 1970s (Dijkstra, Parnas, Weiss, Pohl, ...) - variabilities are often from a finite
set of simple options \cite{Parnas1976, Parnas1979, Dijkstra1972}
\item Families of algorithms and code generation in SC (Carette, ATLAS, Blitz++, ...) - not much emphasis on
requirements \cite{Carette2006, WhaleyEtAl2001, Veldhuizen1998, Blitz2010}
%\item Problem Solving Environments (PSEs)
\item Work on requirements for SC
\begin{itemize}
\item Template for a single physical model \cite{SmithEtAl2007, SmithAndLai2005}
\item Template for a family of multi-purpose tool \cite{Smith2006,
    SmithAndChen2004, SmithAndChen2004b}
\item Template for a family of physical models
  \cite{SmithMcCutchanAndCarette2017, SmithEtAl2008, McCutchan2007}
\end{itemize}
\end{itemize}

\end{frame}

 %%%%%%%%%%%%%%%%%%%%%%%%%%%%%%%%%%%%%%%%%%%%%%%%%%%%%%

\begin{frame}

\frametitle{Motivation}
%:Improve Quality of Product and Process
\begin{itemize}

\item Requirements documentation
\begin{itemize}
\item Allows judgement of quality %need to know what require, 
\item Improves communication
\begin{itemize}
\item Between domain experts% and domain experts %missing models etc.
\item Between domain experts and programmers %experts on num algos
\item Explicit assumptions
\item Range of applicability
%\item Tradeoffs between nonfunctional requirements %ends arguments about best, 
% need to know what is most imp, might not need same accuracy for every aspect of the problem
\end{itemize}
%\item Provides a foundation for incremental delivery
\end{itemize}

\item A family approach, potentially including a DSL to allow generation of specialized programs
\begin{itemize}
\item Improves efficiency of product and process
\item Facilitates reuse of requirements and design, which improves reliability
\item Improves usability and learnability%DSL
%\item Improves learnability for non-experts %DSL
\item Clarifies the state of the art
\end{itemize}

\end{itemize}

\end{frame}

%%%%%%%%%%%%%%%%%%%%%%%%%%%%%%%%%%%%%%%%%%%%%%%%%%%%%%

\begin{frame}

\frametitle{Advantages of Program Families to SC?}

\begin{itemize}
\item Usual benefits
\begin{itemize}
\item Reduced development time
\item Improved quality
\item Reduced maintenance effort
\item Increased ability to cope with complexity
\end{itemize}
\item Reusability
\begin{itemize}
\item Underused potential for reuse in SC
\item Reuse commonalities
\item Systematically handle variabilities
\end{itemize}
\item Usability
\begin{itemize}
\item Documentation often lacking in SC
\item Documentation part of program family methodology
\item Create family members that are only as general purpose as necessary
\end{itemize}
\item Improved performance
\end{itemize}

\end{frame}

%%%%%%%%%%%%%%%%%%%%%%%%%%%%%%%%%%%%%%%%%%%%%%%%%%%%%%

\begin{frame}

\frametitle{Is SC Suited to a Program Family Approach?}

Based on criteria from Weiss \cite{ArdisAndWeiss1997, Weiss1997, Weiss1998,
  CukaAndWeiss1997,WeissAndLai1999}
\begin{itemize}
\item The redevelopment hypothesis
\begin{itemize}
\item A significant portion of requirements, design and code should be common between family members
\item Common model of software development in SC is to rework an existing program
\item Progress is made by removing assumptions
\end{itemize}

\item The oracle hypothesis
\begin{itemize}
\item Likely changes should be predictable
\item Literature on SC, example systems, mathematics
\end{itemize}

\item The organizational hypothesis
\begin{itemize}
\item Design so that predicted changes can be made independently
\item Tight coupling between data structures and algorithms
\item Need a suitable abstraction
\end{itemize}

\end{itemize}

\end{frame}

%%%%%%%%%%%%%%%%%%%%%%%%%%%%%%%%%%%%%%%%%%%%%%%%%%%%%%

\begin{frame}

\frametitle{Challenges}

\begin{enumerate}
\setcounter{enumi}{\value{temp}}
\item Validatable
\begin{itemize}
\item Requirements can be complete, consistent, traceable and unambiguous, but still not validatable
\item Input and outputs are continuously valued variables
%\item Examples continuously valued variables: time, velocity, temperature, displacement, concentration, stress, etc.
%\item Infinite number of inputs and outputs
\item Correct solution is unknown a priori %difficult to have a test oracle
\item Given $dy/dt = f(t, y)$ and $y(t_0) = y_0$, find $y(t_n)$ 
%, where $y(t)$ is a function ($y:\mathbb{R} \rightarrow 
%\mathbb{R}$), $f(t,y)$ is a function ($f:\mathbb{R}\times \mathbb{R} \rightarrow \mathbb{R}$), $t$ is an independent
%variable (often time), $t_0$ is an initial value for $t$ and $t_n$ is the final value for $t$
%\item For arbitrary $f(t, y)$ the true solution is unknown, correct value is often not known a priori
%\item Same problem with $Ax = b$
%\item $\int e^{x^2} dx$
%\item Passing one test does not imply passing a nearby test
\end{itemize}
\item Abstract
\begin{itemize}
% \item Not difficult to be abstract
\item If too abstract, then difficult to meet NFRs for accuracy and speed
\item Assumptions can help restrict scope, but possibly as much work as solving the original problem
\begin{itemize}
\item $A x = b$
\item $x^T A x > 0, \forall x$
\end{itemize}
%say output should be solution, or say cannot compute, possibly with a reason
\item Algorithm selection should occur at the design stage

\end{itemize}

\setcounter{temp}{\value{enumi}}
\end{enumerate}

\end{frame}

%%%%%%%%%%%%%%%%%%%%%%%%%%%%%%%%%%%%%%%%%%%%%%%%%%%%%%

\begin{frame}

\frametitle{Challenges (Continued)}

\begin{enumerate}
\setcounter{enumi}{\value{temp}}
\item Nonfunctional requirements
\begin{itemize}
\item Proving accuracy requirements with a priori error analysis is a difficult mathematical exercise that generally
leads to weak error bounds
% mention Wilkinson
\item Context sensitive tradeoffs between NFRs can be difficult to specify
\item Absolute quantitative requirements are often unrealistic
% make the point about relative requirements elsewhere
\end{itemize}
\item Capture and Reuse Existing Knowledge
\begin{itemize}
\item Cannot ignore the enormous wealth of information that currently exists %not a green field development
\item A good design will often involve integrating existing software libraries
\item Reuse software and the requirements documentation
% scientific software is stable
\end{itemize}

\setcounter{temp}{\value{enumi}}
\end{enumerate}

\end{frame}

%%%%%%%%%%%%%%%%%%%%%%%%%%%%%%%%%%%%%%%%%%%%%%%%%%%%%%

\begin{frame}
\frametitle{Goal Statements for a Family of Linear Solvers?}

\structure{What would be a good goal statement for a library of linear solvers?}

\end{frame}

%%%%%%%%%%%%%%%%%%%%%%%%%%%%%%%%%%%%%%%%%%%%%%%%%%%%%%%%%%%%%%%%%

\begin{frame}
\frametitle{Goal Statements for a Family of Linear Solvers}

\begin{itemize}
	
\item[G1] Given a system of $n$ linear equations represented by matrix $A$ and
  column vector $b$, return $x$ such that $Ax = b$, if possible

\end{itemize}

\end{frame}

%%%%%%%%%%%%%%%%%%%%%%%%%%%%%%%%%%%%%%%%%%%%%%%%%%%%%%%%%%%%%%%%%

\begin{frame}
\frametitle{Theoretical Model for a Family of Linear Solvers?}

\begin{itemize}
	
\item \structure{Is the theoretical model a commonality or a variability?}
\item \structure{What is the theoretical model for a family of linear solvers?}
    
\end{itemize}

\end{frame}

%%%%%%%%%%%%%%%%%%%%%%%%%%%%%%%%%%%%%%%%%%%%%%%%%%%%%%%%%%%%%%%%%

\begin{frame}
\frametitle{Theoretical Model for a Family of Linear Solvers}

Given a square matrix $A$ and column vector $b$, the possible
solutions for $x$ are as follows:

\begin{enumerate}
\item A unique solution $x = A^{-1} b$, if $A$ is nonsingular
\item An infinite number of solutions if $A$ is singular and $b \in span(A)$
\item No solution if $A$ is singular and $b \notin span(A)$
\end{enumerate}

\cite{Anton1987}

\end{frame}

%%%%%%%%%%%%%%%%%%%%%%%%%%%%%%%%%%%%%%%%%%%%%%%%%%%%%%%%%%%%%%%%%

\begin{frame}
\frametitle{Instance Model for a Family of Linear Solvers?}

\begin{itemize}
	
\item \structure{Is there an instance model for a family of linear solvers?}
    
\end{itemize}

% No

\end{frame}

%%%%%%%%%%%%%%%%%%%%%%%%%%%%%%%%%%%%%%%%%%%%%%%%%%%%%%%%%%%%%%%%%

\begin{frame}
\frametitle{Symbols and Terminology for a Family of Linear Solvers?}

\begin{itemize}
	
\item \structure{What symbols and terminology will you need to define?}
    
\end{itemize}

\end{frame}

%%%%%%%%%%%%%%%%%%%%%%%%%%%%%%%%%%%%%%%%%%%%%%%%%%%%%%%%%%%%%%%%%

\begin{frame}
\frametitle{Sample Symbols and Terminology}

\begin{table}[h]
\begin{tabular}{ l p{7.5cm}}
$n: \mathbb{N}$ & number of linear equations/number of unknowns\\
$A: \mathbb{R}^{n \times n}$ & $n \times n$ real matrix\\
$x: \mathbb{R}^{n \times 1}$ & $n \times 1$ real column vector\\
$b: \mathbb{R}^{n \times 1}$ & $n \times 1$ real column vector\\
$I: \mathbb{R}^{n \times n}$ & an $n \times n$ matrix where all entries are $0$, except for the diagonal entries, which
are $1$\\
$|| v || $ & the norm (estimate of magnitude) of vector $v$\\
$A^{-1}: \mathbb{R}^{n \times n}$ & the inverse matrix, with the property that $A^{-1} A = I$\\
singular & matrix $A$ is singular if $A^{-1}$ does not exist\\
residual & $|| b - A x ||$\\
\end{tabular}
%\caption{Terminology for a Linear Solver}\label{Term_LinSolv}
\end{table}

\end{frame}

%%%%%%%%%%%%%%%%%%%%%%%%%%%%%%%%%%%%%%%%%%%%%%%%%%%%%%%%%%%%%%%%%

\begin{frame}
\frametitle{What Would be the Most General Binding Time?}

\begin{itemize}
	
\item \structure{What would be the most general binding time for the variabilities?}
    
\end{itemize}

% build time or run time
% scope time has already set some variabilities - a different sub-family could
% have different variabilities

\end{frame}

%%%%%%%%%%%%%%%%%%%%%%%%%%%%%%%%%%%%%%%%%%%%%%%%%%%%%%%%%%%%%%%%%

\begin{frame}
\frametitle{What Are Some Potential Input Variabilities?}

\begin{itemize}
	
\item \structure{What are some potential input variabilities?  What are the
    associated parameters of variation?}
    
\end{itemize}

\end{frame}

%%%%%%%%%%%%%%%%%%%%%%%%%%%%%%%%%%%%%%%%%%%%%%%%%%%%%%%%%%%%%%%%%

%\newcommand{\colA}{1.7cm}
%\newcommand{\colB}{5.6cm}
%\newcommand{\colC}{1.1cm} %do not need this column if binding time is not listed

%%%%%%%%%%%%%%%%%%%%%%%%%%%%%%%%%%%%%%%%%%%%%%%%%%%%%%%%%%%%%%%%%

\begin{frame}
%\frametitle{Input Variabilities}

\begin{table}
\begin{tabular}{| p{\colA} | p{\colB} | }
\hline
\textbf{Variability} & \textbf{Parameter of Variation} \\
\hline
Allowed structure of $A$ & Set of \{ full, sparse, banded, tridiagonal, block triangular,
block structured, diagonal, upper triangular, lower triangular, Hessenberg \} \\
\hline
Allowed definiteness for $A$ & Set of \{ not definite, positive definite, positive semi-definite,
negative definite, negative semi-definite \} \\
\hline
Allowed class of $A$ & Set of \{ diagonally dominant, Toeplitz, Vandermonde \} \\
\hline
Symmetric? & boolean \\
\hline
Values for $n$ & set of $\mathbb{N}$ \\
\hline
Entries in $A$ & set of $\mathbb{R}$ \\
\hline
Entries in $b$ & set of $\mathbb{R}$ \\
\hline
\end{tabular}
%cl\caption{Variabilities for Input Assumptions}\label{Var_Table_Input}
\end{table}

\end{frame}

%%%%%%%%%%%%%%%%%%%%%%%%%%%%%%%%%%%%%%%%%%%%%%%%%%%%%%%%%%%%%%%%%

\begin{frame}
%\frametitle{Input Variabilities}

\begin{table}
\begin{tabular}{| p{\colA} | p{\colB} | }
\hline
\textbf{Variability} & \textbf{Parameter of Variation} \\
\hline
Source\newline of input & Set of \{ from a file, through the user interface, passed in memory \} \\
\hline
Encoding of input & Set of \{binary, text \} \\
\hline
Format\newline of input $A$ & Set of \{arbitrary, by row, by column, by diagonal \} \\
\hline
Format\newline of input $b$ & Set of \{arbitrary, ordered \} \\
\hline
\end{tabular}
%cl\caption{Variabilities for Input Assumptions}\label{Var_Table_Input}
\end{table}

\end{frame}

%%%%%%%%%%%%%%%%%%%%%%%%%%%%%%%%%%%%%%%%%%%%%%%%%%%%%%%%%%%%%%%%%

\begin{frame}
\frametitle{What Are Some Potential Output Variabilities?}

\begin{itemize}
	
\item \structure{What are some potential output variabilities? What are the
    associated parameters of variation?}
    
\end{itemize}

\end{frame}

%%%%%%%%%%%%%%%%%%%%%%%%%%%%%%%%%%%%%%%%%%%%%%%%%%%%%%%%%%%%%%%%%

\begin{frame}
\frametitle{Output Variabilities}

\begin{table}
\begin{tabular}{| p{\colA} | p{\colB} | }
\hline
\textbf{Variability} & \textbf{Parameter of Variation} \\
\hline
Destination for output $x$ & Set of \{ to a file, to the screen, to memory \} \\
\hline
Encoding of output $x$ & Set of \{binary, text \} \\
\hline
Format of output $x$ & Set of \{arbitrary, ordered \} \\
\hline
Output residual & boolean (true if the program returns the residual) \\
\hline
Possible entries in $x$ & set of $\mathbb{R}$ \\
\hline
\end{tabular}
%\caption{Variabilities for Output}\label{Var_Table_Output}
\end{table}

\end{frame}

%%%%%%%%%%%%%%%%%%%%%%%%%%%%%%%%%%%%%%%%%%%%%%%%%%%%%%%%%%%%%%%%%

\begin{frame}
\frametitle{What Are Some Potential Calculation Variabilities?}

\begin{itemize}
	
\item \structure{What are some potential calculation variabilities? What are the
    associated parameters of variation?}
    
\end{itemize}

\end{frame}

%%%%%%%%%%%%%%%%%%%%%%%%%%%%%%%%%%%%%%%%%%%%%%%%%%%%%%%%%%%%%%%%%

\begin{frame}
\frametitle{Calculation Variabilities}

\begin{table}
\begin{tabular}{| p{\colA} | p{\colB} | }
\hline
\textbf{Variability} & \textbf{Parameter of Variation} \\
\hline
Check input? & boolean (false if the input is assumed to satisfy the input assumptions)\\ 
\hline
Exceptions generated? & boolean (false if the goal is non-stop arithmetic)\\ 
\hline
Norm used for residual & Set of \{1-norm, 2-norm, $\infty$-norm \} \\
\hline
\end{tabular}
%\caption{Variabilities for Calculation}\label{Var_Table_Calc}
\end{table}

\end{frame}

%%%%%%%%%%%%%%%%%%%%%%%%%%%%%%%%%%%%%%%%%%%%%%%%%%%%%%%%%%%%%%%%%

\begin{frame}[allowframebreaks]
\frametitle{References}

\bibliography{../../ReferenceMaterial/References}

\end{frame}

%%%%%%%%%%%%%%%%%%%%%%%%%%%%%%%%%%%%%%%%%%%%%%%%%%%%%%

\end{document}