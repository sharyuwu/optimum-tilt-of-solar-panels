\documentclass[12pt,fleqn]{article}

\usepackage{url}

\setlength {\topmargin} {-.15in}
\setlength {\textheight} {8.6in}

\newcommand{\be}{\begin{enumerate}}
\newcommand{\ee}{\end{enumerate}}
\newcommand{\bi}{\begin{itemize}}
\newcommand{\ei}{\end{itemize}}
\newcommand{\bc}{\begin{center}}
\newcommand{\ec}{\end{center}}
\newcommand{\bsp}{\begin{sloppypar}}
\newcommand{\esp}{\end{sloppypar}}
\renewcommand{\labelenumii}{\theenumii.}

\begin{document}

\bc

{\Large \textbf{CAS 741/CES 741}}\\[2mm]{\large \textbf{Dr.~Spencer Smith}}\\[2mm]
{\large \textbf{McMaster University}}\\[6mm]
{\LARGE \textbf{Git and GitLab/GitHub}}\\[4mm]
{\large Revised: September 7, 2017}

\ec

\medskip

\noindent

This tutorial will introduce GitLab, GitHub, Git and issue tracking.

\subsection*{Components of Lab}

\be
\item Introduction to GitLab/GitHub

%Assume students have no knowledge of version control (even though this is not
% the case for many of them).

%BEGIN INTRO SCRIPT
%

% What is git? **Explain**

%

% git workflows (the tutorial linked later explains this in great depth)
%		Centralized: Everyone clones and works on master
%		Feature-branch: Everyone clones and works on their own feature-specific branch.
%		Gitflow: Strict branches based around project release.
%			Contains master, develop, feature, release, and maintenance branches.
%			Might be a little overcomplicated for our projects.
%		Forking: Avoid using this workflow for this project.

% Basic breakdown of how a solo user would start to work with git
%		init repository (git init)
%		create files
%		stage changes to be committed (git status, git add)
%		commit staged changes (git commit -m ``message'')
%		push to remote if using one (git push)

% Group of users starting to use git (assume remote repo already created)
%		clone repo
%		make changes
%		stage changes
%		commit staged changes
%		push

% Regular work-habit strategies
%		begin session by pulling
%		commit after every separate issue is tackled (implementing a new module, fixing 
%			test cases, etc.) AND when you need to stop working, regardless of progress.
%			ALWAYS include a MEANINGFUL and DESCRIPTIVE commit message for the log.
% 		push every few hours OR after several commits, AND at the end of every session. 
%		if a push reveals conflicts, take appropriate action to merge/sort them out

%
%END INTRO SCRIPT


\item GitLab Exercises
\ee

\subsection*{Details}

\bi
\item Make sure that you have Git on your machine. 
If you do not have Git, download it from: \newline \url{https://git-scm.com/}

\item Try the interactive git tutorial available at:
\url{https://try.github.io}

\item Work through the Software Carpentry lab exercise on Git.  This is
  available at \url{http://swcarpentry.github.io/git-novice/}

\item Great resource for git help:
\newline\url{https://git-scm.com/book/en/v2/Git-Basics-Getting-a-Git-Repository}

\item Learn about branches and how/when to use them:
\newline\url{https://git-scm.com/book/en/v2/Git-Branching-Branches-in-a-Nutshell}

\item View some videos on Git:
%\bi
\url{https://git-scm.com/videos}
%\item \url{http://gitreal.codeschool.com/levels/1} removed because only level 1 is free
%\ei

\item Learn about a \verb|.gitignore| file:
\newline\url{https://help.github.com/articles/ignoring-files/}
	\newline Note: Generated files \textbf{should not} be placed into version control. 
	The only exception we recommend is for \LaTeX~generated pdf files. For
        convenience, they should often be under version control.

\item Read through the tutorial on git workflows: \newline
\url{https://www.atlassian.com/git/tutorials/comparing-workflows/}

\item Once you are comfortable with Git, set-up your project repo on GitHub.  We
  are using GitHub so that it is easy to share your project.  Creating a repo will involve
  the following steps 
\be
\item Create a new public repository on GitHub
\item Add the members (collaborators), including the instructor
\item You may later have to add some classmates
\item You may want to add your supervisor and/or some colleagues
\item Create the initial folder structure for your project, as given in the
  BlankProjectTemplate folder.
\item Create your .gitignore file and ignore .aux
files as an example. Commit this file to your repository.
\ee

\item Review the material on issue tracking also available in this folder, in
  the file instructions\_issue\_tracking.pdf.

\ei

\end{document}