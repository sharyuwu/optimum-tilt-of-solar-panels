\documentclass[12pt,fleqn]{article}
\usepackage{url}
\usepackage{scrextend}
\setlength {\topmargin} {-.15in}
\setlength {\textheight} {8.6in}
\newcommand{\be}{\begin{enumerate}}
\newcommand{\ee}{\end{enumerate}}
\newcommand{\bi}{\begin{itemize}}
\newcommand{\ei}{\end{itemize}}
\newcommand{\bc}{\begin{center}}
\newcommand{\ec}{\end{center}}
\newcommand{\bsp}{\begin{sloppypar}}
\newcommand{\esp}{\end{sloppypar}}
\renewcommand{\labelenumii}{\theenumii.}
\begin{document}
\bc
{\Large \textbf{SOFTWARE ENGINEERING 3XA3}}\\[2mm]
{\large \textbf{Software Engineering Practice \& Experience: Software Project
Management }}\\[6mm]
{\large \textbf{Dr.~Spencer Smith}}\\[2mm]
{\large \textbf{McMaster University, Fall 2016}}\\[6mm]
{\LARGE \textbf{Laboratory 12 Make}}\\[4mm]
{\large Revised: September 12, 2016}
\ec
\medskip
\noindent
This lab will introduce Make. Make is a program building tool that controls the
generation of executables of a program from the source files. When the number of
files increases in a project, Make saves lots of compilation time.
\subsection*{Components of Lab}

%Assume students have no knowledge of Make
%BEGIN INTRO SCRIPT
%
% What is Make? 
%
% Make is a tool that controls the generation of executables of a program from
% the source files.
% 
%Why Make?
%
% The following issues will arise as the number of files and modules increases
% in a program:
% - When we change a file it is hard to determine which files are dependent on
% the new change.
% - As a result, we might need to re-compile the whole program which is not
% efficient.
% When we use Make, it determines which files are changed and compiles them. It
% also figures out  their dependent files and compiles them as
% well. It automatically determines the proper order for updating files. As a
% result, there is no need to re-compile all of the program.
%
% Make also enables the end user to build and install packages without knowing
% the details.
%
% How does Make work?
%
% Make reads the file Makefile (default name) in the current directory to
% determine what to do.
% A Makefile typically starts with some variable definitions followed by a set
% of targets
% (for example classes in Java and .o & executable files in C/C++) and then a
% series of commands associated with the files which those commands need
% to be applied to. In other words, for each target, we need to write a list of the
% dependent files (aka sensitivity list) and then we write commands needed
% to create that target.
%
%END INTRO SCRIPT
\be
\item Introduction to Make
\item Make Exercises
\ee
\subsection*{Details}

\bi
\item Make sure you are using a Unix-based command-line interpreter.
\item Work through the Software Carpentry lab exercise on Make. This is
  available at \url{https://swcarpentry.github.io/make-novice/}
\item Read through the tutorial on Make: \newline
\url{https://www.youtube.com/watch?v=aEMr3VfIzb4}
\item Makefile Rules:
\\
\url{https://www.tutorialspoint.com/makefile/makefile_rules.htm}
\item Makefile dependencies:
\\
\url{https://www.tutorialspoint.com/makefile/makefile_dependencies.htm}
\item Makefile macros:
\\
\url{https://www.tutorialspoint.com/makefile/makefile_macros.htm}
\item Mainly Makefile contains a set of rules used to build an
application. These rules can be divided to three main parts: the target,
its prerequisite files, and the command(s) to perform:
\begin{addmargin*}[4em]{2em}
\textbf{\textit{target}}: \textit{prerequisites}
\\
\hspace*{13 mm} \textit{commands}
\end{addmargin*}
The target is the file that must be made. The prerequisite files are those
files that the target depends on them and
must be compiled before the target can be created. The commands are those shell
commands that
 needs to be executed to create the target from the prerequisites.
\bi
\item An example for Java makefile:
\\
\url{https://www.cs.swarthmore.edu/~newhall/unixhelp/javamakefiles.html}
\item An example for Python makefile:
\\
\url{http://www.wellho.net/resources/ex.php4?item=y212/makefile}
\ei
\item Once you are comfortable with Make, write a makefile for LaTeX so that it 
is possible to use Make to build your documents. 
\ei

\subsection*{For The TAs (Students can ignore this section)}
Before the conclusion of the lab, please make sure that you have tested each
student to verify that they understand the basics of make.  For students that
understand the basics, please give them a grade of 1 for the first Lab Exercise.


\end{document}
