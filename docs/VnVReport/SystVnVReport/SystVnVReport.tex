\documentclass[12pt, titlepage]{article}


\usepackage{amsmath, mathtools}
\usepackage{float}
\usepackage{booktabs}
\usepackage{tabularx}
\usepackage{hyperref}
\hypersetup{
    colorlinks,
    citecolor=black,
    filecolor=black,
    linkcolor=red,
    urlcolor=blue
}
%\usepackage[round]{natbib}
\usepackage[square,sort,comma,numbers]{natbib} 
\input{../../Comments}
%% Common Parts

\newcommand{\progname}{Sun Catcher} % PUT YOUR PROGRAM NAME HERE %Every program
                                % should have a name


\begin{document}

\title{Test Report: System Verification and Validation Report for Sun Catcher} 
\author{Sharon (Yu-Shiuan) Wu}
\date{\today}
	
\maketitle

\pagenumbering{roman}

\section{Revision History}

\begin{tabularx}{\textwidth}{p{3cm}p{2cm}X}
\toprule {\bf Date} & {\bf Version} & {\bf Notes}\\
\midrule
2019/12/21 & 1.0 & First Version\\
Date 2 & 1.1 & Notes\\
\bottomrule
\end{tabularx}

~\newpage

\section{Symbols, Abbreviations and Acronyms}

\renewcommand{\arraystretch}{1.2}
\begin{tabular}{l l} 
  \toprule		
  \textbf{symbol} & \textbf{description}\\
  \midrule 
  T & Test\\
  \bottomrule
\end{tabular}\\

\wss{symbols, abbreviations or acronyms -- you can reference the SRS tables if needed}

\newpage

\tableofcontents

\listoftables %if appropriate

\listoffigures %if appropriate

\newpage

\pagenumbering{arabic}

This document is the test report of the system testing for \progname

\section{Functional Requirements Evaluation}
\begin{enumerate}
\item{InputReading-id1\\}
This is the testing for ensuring the software has the ability to read the input value $P_{A_{h}}$ and $P_{A_{w}}$.
The the input can be find under the path ``../src/tiltAngPro/test/tests"\\
Input File Name : ``id1.inputReading"\\
Output File Nmae: ``id1.inputReading.golden"



\begin{table}[h!]
\centering
\noindent \begin{tabular}{l l l l} 
    \toprule		
    \textbf{id} & \textbf{Input} & \textbf{Output}\\ 
	\midrule
   id1.1 &  (1455, 665)  & (1455, 665)\\
   id1.2 &  (1455.54, 665.13) & (1455.54, 665.13)\\
    \bottomrule
  \end{tabular}
\caption{Actual Input and Expected Output}
\end{table}


Content:
\begin{center}
.inputReading\\
id1\_1\\
Input 1455.0 Absolute Erros = 0.0\\
Input 665.0 Absolute Erros = 0.0\\
id1\_2\\
Input 1455.54 Absolute Erros = 0.0\\
Input 665.13 Absolute Erros = 0.0\\
\end{center}



This result shows all the absolute error for the cases under InputReading-id1 is 0. Therefore the case success.

\item{InputReading-id2\\}

This is the testing for ensuring the software has the ability to read the input value $\Phi_P$.\\
The the input can be find under the path ``../src/tiltAngPro/test/tests"\\
Input File Name : ``id2.inputReading"\\
Output File Nmae: ``id2.inputReading.golden"

\begin{table}[h!]
\centering
\noindent \begin{tabular}{l l l l} 
    \toprule		
    \textbf{id} & \textbf{Input} & \textbf{Output}\\ 
	\midrule
   id2.1 &  90  & 90\\
   id2.2 & -90  & -90\\
   id2.3 & 3.2  & 3.2\\
   id2.4 & -3.2  & -3.2\\
   id2.5 & 0  & 0\\
    \bottomrule
  \end{tabular}
\caption{Actual Input and Expected Output}
\end{table}


\begin{center}

.inputReading\\
id2\_1\\
Input 90.0 Absolute Erros = 0.0\\
id2\_2\\
Input -90.0 Absolute Erros = 0.0\\
id2\_3\\
Input 3.2 Absolute Erros = 0.0\\
id2\_4\\
Input -3.2 Absolute Erros = 0.0\\
id2\_5\\
Input 0.0 Absolute Erros = 0.0\\
\end{center}

This result shows all the absolute error for the cases under InputReading-id2 is 0. Therefore the case success.

\item{InputReading-id3\\}

This is the testing for ensuring the software has the ability to read the input value  ($\mathit{year}_\text{Start}$, $\mathit{month}_\text{Start}$, $\mathit{day}_\text{Start}$)  ($\mathit{year}_\text{End}$, $\mathit{month}_\text{End}$, $\mathit{day}_\text{End}$).\\


The the input can be find under the path ``../src/tiltAngPro/test/tests"\\
Input File Name : ``id3.inputReading"\\
Output File Nmae: ``id3.inputReading.golden"

\begin{table}[h!]
\centering
\noindent \begin{tabular}{l l l l} 
    \toprule		
    \textbf{id} & \textbf{Input} & \textbf{Output}\\ 
	\midrule
   id3.1 & (2020, 02, 28) - (2021, 02, 28) & (2020, 02, 28) - (2021, 02, 28)\\
   id3.2 & (1996, 01, 03) - (2000, 01, 14)  & (1996, 01, 03) - (2000, 01, 14) \\
    \bottomrule
  \end{tabular}
\caption{Actual Input and Expected Output}
\end{table}

\begin{center}
.inputReading\\
id3\_1\\
Input 2020-02-28 Absolute Erros = 0\\
Input 2021-02-28 Absolute Erros = 0\\
id3\_2\\
Input 1996-01-03 Absolute Erros = 0\\
Input 2000-01-14 Absolute Erros = 0\\
\end{center}

This result shows all the absolute error for the cases under InputReading-id3 is 0. Therefore the case success.

\end{enumerate}
\section{Nonfunctional Requirements Evaluation}

\subsection{Usability}
		
\subsection{Performance}

\subsection{etc.}
	
\section{Comparison to Existing Implementation}	

This section will not be appropriate for every project.

\section{Unit Testing}

\section{Changes Due to Testing}

\section{Automated Testing}
		
\section{Trace to Requirements}
		
\section{Trace to Modules}		

\section{Code Coverage Metrics}

\bibliographystyle{plain}
\bibliography{../../../refs/References} 
%\bibliography{SRS}

\end{document}