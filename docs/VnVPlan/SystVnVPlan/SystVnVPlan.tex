\documentclass[12pt, titlepage]{article}

\usepackage{amsmath, mathtools}
\usepackage{booktabs}
\usepackage{tabularx} 
\usepackage{hyperref} 
\hypersetup{ colorlinks, citecolor=black, filecolor=black, linkcolor=red,
urlcolor=blue }

% For easy change of table widths
\newcommand{\colZwidth}{1.0\textwidth}
\newcommand{\colAwidth}{0.13\textwidth}
\newcommand{\colBwidth}{0.82\textwidth}
\newcommand{\colCwidth}{0.1\textwidth}
\newcommand{\colDwidth}{0.05\textwidth}
\newcommand{\colEwidth}{0.8\textwidth}
\newcommand{\colFwidth}{0.17\textwidth}
\newcommand{\colGwidth}{0.5\textwidth}
\newcommand{\colHwidth}{0.28\textwidth}

%\usepackage[round]{natbib}
\usepackage[square,sort,comma,numbers]{natbib} 
%% Comments

\usepackage{color}

\newif\ifcomments\commentstrue

\ifcomments
\newcommand{\authornote}[3]{\textcolor{#1}{[#3 ---#2]}}
\newcommand{\todo}[1]{\textcolor{red}{[TODO: #1]}}
\else
\newcommand{\authornote}[3]{}
\newcommand{\todo}[1]{}
\fi

\newcommand{\wss}[1]{\authornote{blue}{SS}{#1}} 
\newcommand{\plt}[1]{\authornote{magenta}{TPLT}{#1}} %For explanation of the template
\newcommand{\an}[1]{\authornote{cyan}{Author}{#1}}

%% Common Parts

\newcommand{\progname}{ProgName} % PUT YOUR PROGRAM NAME HERE %Every program
                                % should have a name

\begin{document}
\title{Project Title: System Verification and Validation Plan for Sun Catcher}

\author{Sharon(Yu-Shiuan) Wu} \date{\today} \maketitle
\pagenumbering{roman}

\section{Revision History}

\begin{tabularx}{\textwidth}{p{3cm}p{2cm}X}
\toprule {\bf Date} & {\bf Version} & {\bf Notes}\\ 
\midrule 
2019/11/04 1 & 1.0 & First version of System VnV\\ 
Date 2 & 1.1 & -\\ 
\bottomrule 
\end{tabularx} \newpage
\tableofcontents

\listoftables

\listoffigures

\newpage

\section{Symbols, Abbreviations and Acronyms}

\renewcommand{\arraystretch}{1.2}
\begin{tabular}{l l} \toprule \textbf{symbol} & \textbf{description}\\
 \midrule T & Test\\ 
SC & Sun Catcher\\ 
$\Phi_P$ & the latitude of the solar panel \\ 
$I_{S}$ &  the intensity of the sun measured by the satellites \\ 
$P_{A_{h}}$ & the height of the solar panel \\ 
$P_{A_{w}}$ & the width of the solar panel \\ 
$\mathit{year}_\text{Start}$ & the year of the calcuation's starting date\\ 
$\mathit{month}_\text{Start}$ & the month of the calcuation's starting date\\ 
$\mathit{day}_\text{Start}$ & the day of the calcuation's starting date\\ 
$\mathit{year}_\text{End}$ & the year of the calcuation's ending date\\ 
$\mathit{month}_\text{End}$ & the month of the calcuation's ending date\\ 
$\mathit{day}_\text{End}$ & the day of the calcuation's ending date\\ 
\bottomrule
\end{tabular}\\ 

\wss{symbols, abbreviations or acronyms -- you can simply
reference the SRS tables, if appropriate} 

\newpage

\pagenumbering{arabic}

This document provides an outline of the system verification and validation for
the \progname. The general introduction section provides readers a summary of
the functions in \progname ~and the related documents as the resources for
testing. The plan section provides readers the plan for verifying and
validating SC's software requirements specification(SRS) and introduces the
method, tools, and external data to implement the testing. The system test
description
provides the readers with the test cases related to functional and nonfunctional
requirements in \progname. The test builds for uncovering the errors and
boosting the confidence of the software while ensuring an acceptable
performance is provided. 

\wss{provide an introductory blurb and roadmap of the Verification and
Validation plan}


\section{General Information}
\subsection{Summary}

The subsection belows are the test cases of \progname. \progname ~is the
software that calculate the optimum tilt angle of the days duration that
decidedby users. Then it takes the calculated angle to estimate the optimal
solar energy output.

\wss{Say what software is being tested. Give its name and a brief overview of
its general functions.}

\subsection{Objectives}\label{Objectives}
The goal of the test is to build confidence in the software correctness and
strengthen the robustness of the software. The functional and nonfunctional
requirements of \progname~, and its related equations and constraints are 
found in the SRS \cite{YS2019}. The Verification and Validation Plan follows
by the requirements described in the SRS \cite{YS2019}.

The object of the Verification and Validation Plan is:
\noindent \begin{itemize}

\item[ ] To build the robustness of the system input.
\item[ ] To build the confidence of the correctness of the system output.

\noindent \end{itemize}

The functional requirements of  \progname~ are:
\noindent \begin{itemize}

\item[ ]R1: \progname~should able to read the user's input value from the
system's interface.

\item[ ]R2: \progname~should able to verify the input by the input's constrain.

\item[ ]R3: \progname~should able to output the result in
the corrected format.

\item[ ]R4: \progname~should able to determine the tilt angle

\item[ ]R5: \progname~should able to verify the output by the output's constrain. 

\item[ ]R6: \progname~should able to show the comparison
between the outputs corresponding to the different user's input in a graphic.

\item[ ]R7: \progname~should able to show the comparison
between the outputs corresponding to the different user's input in a table.

\item[ ]R8: The system output should remain consistent with
the corresponding of the input values.

\end{itemize}

The nonfunctional requirements  of  \progname~ are: 
\begin{itemize} 
\item[ ]NFR1. Correct: The output should give correct results.
\item[ ]NFR2. Verifiable: The code is tested with complete verification and
validationplan
\item[ ]NFR3. Understandable: The code is understandable for readers. 
\item[ ]NFR4. Reusable: The code is modularized
\item[ ]NFR5. Maintainable: The traceability between requirements, assumptions,
theoretical models, general definitions, data definitions, instance models,
likely changes, unlikely changes, and modules is completely recorded in
traceability matrices in the SRS and module guide
\item[ ]NFR6. Portable: The code is able to be run in different environments 
\end{itemize}

\wss{State what is intended to be accomplished. The objective will be around
the qualities that are most important for your project. You might have something
like: ``build confidence in the software correctness,'' ``demonstrate adequate
usability.'' etc. You won't list all of the qualities, just those that are most
important.}

\subsection{Relevant Documentation}\label{RD}

\begin{itemize} 
\item[ ]The SRS of the \progname can be found in [Wu~\cite{YS2019}] 
\item[ ]The external documents for verifying the equations used in \progname~
can be found in [Landau \cite{Charles2001} ] and
[MarkandVijaysinh\cite{JacobsonandJadhav} ]

 \end{itemize} 
\wss{Reference relevant documentation. This will definitely include your SRS}

 \section{Plan}	
\subsection{Verification and Validation Team} The test team includes the
following members:\\ 
Main reviewer: Sharon(Yu-Shiuan) Wu\\ 
Secondary reviewer: Deema Alomair, Bo Cao, Sasha Soraine, Zhi Zhang, 
and Doctor Smith\\
\wss{Probably just you. :-)}
\subsection{SRS Verification Plan}

\begin{itemize}

\item Get feedback from the reviewers: Sasha Soraine, Zhi Zhang, and Doctor
Smith, after SRS is completed and put to the GitHub.
\item Check the document by using SRS-Checklist and Writing-Checklist before
publishing to GitHub.

\end{itemize}

\wss{List any approaches you intend to use for SRS verification. This may just
be ad hoc feedback from reviewers, like your classmates, or you may have
something more rigorous/systematic in mind..}

\subsection{Design Verification Plan}
\begin{itemize}

\item The design should be verified by complete and success the test cases in
the system VnV plan under the section \ref{STD}.
\item The design should satisfy all the functional and nonfunctional
requirementthat stated in the SRS document \cite{YS2019}.

\end{itemize}

\wss{Plans for design verification}\label{Planfordv}

\subsection{Implementation Verification Plan} 
The following tools will be used to facilitate testing: 

\begin{itemize}
\item[ ]Rubber Duck Debugging: Performed by author, Sharon(Yu-Shiuan) Wu. The 
author should verbally explain the code line by line.
\item[ ]Haskell Program Coverage: Dynamic Testing Tool, a tool-kit to record 
and display the code coverage of a Haskell Program. It aims to reinforce the 
correctness of the software and to eliminate the infeasibility problems.(Gill
and Runciman\cite{GillandRunciman})
\item[ ]QuickCheck: Automatic testing tool for Haskell programs, a library for
random testing of program properties. It aims to boost the robustness of the 
software.(Claessen\cite{QuickCheck})
		
\end{itemize} 

\wss{You should at least point to the tests listed in this document and the
unittesting plan.} \wss{In this section you would also give any details of any
plansfor static verification of the implementation. Potential techniques
includecodewalkthroughs, code inspection, static analyzers, etc.}

\subsection{Software Validation Plan}

\progname should be valid by satisfied all the functional requirement in SRS 
plan.\\
Based on the physical concept of \progname, the author, Yu-Shiauan Wu, should
record the actual solar energy by using the output from \progname. Then verify
whether the calculated tilt angle can increase the energy gaining.

\wss{If there is any external data that can be used for validation, you should
point to it here. If there are no plans for validation, you should state that
here.}

\section{System Test Description}\label{STD}	
\subsection{Tests for Functional Requirements}

The subsection below is designed to cover the functional requirements of
\progname, which also describes in section \ref{Objectives}.\\
The test is divided into four subsections, which are input reading, input
bounds,
output calculation, and output verification. Input reading testing is designed
for
testing the ability to receive information from the software interface. Input
bounds
testing and output calculation testing are designed for testing the robustness
of the
software. Output verification testing is designed for the correctness of the 
implemented equation.

\wss{Subsets of the tests may be in related, so this section is divided into
different areas. If there are no identifiable subsets for the tests, this level
of document structure can be removed.} \wss{Include a blurb here to explain why
the subsections below cover the requirements. References to the SRS would be
good.}

\subsubsection{Input Reading}

This test covers the requirements, R1, in section \ref{Objectives}. Based on
the SRS document\cite{YS2019}, \progname has to identity users' inputs and
then assign the values to designated equations or modules.

\wss{It would be nice to have a blurb here to explain why the subsections below
cover the requirements. References to the SRS would be good. If a section
coverstests for input constraints, you should reference the data constraints
table in the SRS.}

\paragraph{Identity users' input}
\begin{enumerate}

\item{InputReading-id1\\}

Control: Manual. Input the input value from the keyboard.
					
Initial State: No input value
					
Input: Input the value of requirements of \progname. It required to input the
value of latitude, the area of the solar panel, the day started to estimate
angle and day when the estimation end.\\

 The given inputs are:\\
 ~$\Phi_P$: 43.250943 
~$P_{A_{h}}$:1455
~$P_{A_{w}}$:665
~$\mathit{year}_\text{Start}$:(2019) ~$\mathit{month}_\text{Start}$:(01) 
~$\mathit{day}_\text{Start}$:(01) ~$\mathit{year}_\text{End}$:(2019) 
~$\mathit{month}_\text{End}$:(12) 
~$\mathit{day}_\text{End}$:(31)\\

 Output: The expected result wil for the given inputs is:\\ 
~$\Phi_P$: 43.250943
~$P_{A_{h}}$:1455
~$P_{A_{w}}$:665
~$\mathit{year}_\text{Start}$:(2019) 
~$\mathit{month}_\text{Start}$:(01) 
~$\mathit{day}_\text{Start}$:(01) 
~$\mathit{year}_\text{End}$:(2019) 
~$\mathit{month}_\text{End}$:(12) 
~$\mathit{day}_\text{End}$:(31)\\

\wss{The expected result for the given inputs}

Test Case Derivation: The output is justified if the output value is equal to
the corresponding input value.
\wss{Justify the expected value given in the Output field} 

How test will be performed:
\begin{itemize}
\item Input the value from the keyboard following the instruction of the
software interface.
\item Verified the output showing on the screen by the test case derivation
instruction. 
\end{itemize}
\end{enumerate}


\subsubsection{Input Bounds}

This test covers the requirements, R1, in section \ref{Objectives}. Based on
the SRS document\cite{YS2019}, the input data constraints can be found in the
table, Specification Parameter Values, under the section, Input Data
Constraints.


\paragraph{The Robustness of Input Bounds} 
\begin{enumerate}
\item{InputBounds-id2\\} 

Control: Manual. 

Input the extreme ends of the value of latitude.\\
Initial State: No input value\\
Input: Based on SRS\cite{YS2019} of the \progname\cite{YS2019}, the boundaried
of latitude is $-90 \leq \Phi_P \leq 90$.\\
Therefore, the given inputs is:\\
input (1) $\Phi_P$: 90 
~input (2) $\Phi_P$: -90 
~input (3) $\Phi_P$: 91 
~input(4) $\Phi_P$: -91\\ 

Output: The expected result wil for the given inputs is:\\
output (1) $\Phi_P$: 90 
~output (2) $\Phi_P$: -90 
~output (3) $\Phi_P$: Latitude is not greater than 90 
~output (4) $\Phi_P$: Latitude is not less than -90\\

\wss{The expected result for the given inputs}

Test Case Derivation: The output is justified if the output value is equal to
the corresponding input value. When the input value is over the boundary 
of latitude, the system activates the error handler. \\ 
\wss{Justify the expected value given in the Output field} 

How test will be performed:

\begin{itemize} 

\item Input the value from the keyboard following the instruction of the
software interface.
\item Verified the output showing on the screen by the test case derivation
instruction.
\end{itemize}
\item{InputBounds-id3\\}
Control: Manual. 
 Input the extreme ends, valid date and invalid date of the value according to the Gregorian calendar. \\
Initial State: No any given value.\\

Input: Based on calendar, the given inputs is:\\ 
Input(1):
~$\mathit{year}_\text{Start}$:(0) ~$\mathit{month}_\text{Start}$:(0)
~$\mathit{day}_\text{Start}$:(0)\\  
Input(2):
~$\mathit{year}_\text{Start}$:(-1)
~$\mathit{month}_\text{Start}$:(-1) ~$\mathit{day}_\text{Start}$:(-1)\\

Input(3):
~$\mathit{year}_\text{Start}$:(2020)
~$\mathit{month}_\text{Start}$:(02) ~$\mathit{day}_\text{Start}$:(29)\\

Input(4):
~$\mathit{year}_\text{Start}$:(2020)
~$\mathit{month}_\text{Start}$:(02) ~$\mathit{day}_\text{Start}$:(28)\\


Output:The expected result wil for the given inputs is:\\
Output (1) 0,0,0 doesnot exist.\\ 
Output (2) -1,-1,-1 does not exist.\\
 Output(3) 2020.02.29 does not exist\\ 
 Output(4) 2020.02.28 exist\\ 

\wss{The expected result for the given inputs}

Test Case Derivation: The output is justified if the output value is equal to
the corresponding input value. When the input does not exist in the calender, the system activates the error handler. \\

 \wss{Justify the expected value given in the Output field} 

How test will be performed:
\begin{itemize} 
\item Input the value from the keyboard following the instruction of the
software interface. 
\item Verified the output showing on the screen by the test case derivation
 instruction. 
\end{itemize}

\item{InputBounds-id4\\}
Control: Manual. Input the test case where starting date is greater than ending date. Input
the test case where ending date is greater than starting date.\\
 
Initial State: No any given value.\\
Input: Based on the calendar, the given inputs is:\\ 
Input (1)
~$\mathit{year}_\text{Start}$:(2020)
~$\mathit{month}_\text{Start}$:(02) ~$\mathit{day}_\text{Start}$:(28)

~$\mathit{year}_\text{End}$:(2021)
~$\mathit{month}_\text{End}$:(02) ~$\mathit{day}_\text{End}$:(28)\\

Input (2) ~$\mathit{year}_\text{Start}$:(2020)
~$\mathit{month}_\text{Start}$:(02) ~$\mathit{day}_\text{Start}$:(28)

~$\mathit{year}_\text{End}$:(2019)
~$\mathit{month}_\text{End}$:(02) ~$\mathit{day}_\text{End}$:(28)\\

Input (3) ~$\mathit{year}_\text{Start}$:(2020)
~$\mathit{month}_\text{Start}$:(02) ~$\mathit{day}_\text{Start}$:(28)

~$\mathit{year}_\text{End}$:(2020)
~$\mathit{month}_\text{End}$:(02) ~$\mathit{day}_\text{End}$:(28)\\

Output: The expected result wil for the given inputs is:\\ 
Output (1) Valid\\ 
Output (2) Invalid\\
Output (3) Valid\\ 

\wss{The expected resultfor the given inputs}

Test Case Derivation: When the ending date is less than starting date, the system
activates the error handler. \\ 

\wss{Justify the expected value given in the Output field} 

How test will be performed: 

\begin{itemize} 
\item Input the value from the keyboard following the instruction of the
software interface. 
\item Verified the output showing on the screen by the test case derivation
instruction.
\end{itemize} 
\end{enumerate}

\subsubsection{Output Calculation}

This test covers the requirements, R3 to R6, in section \ref{Objectives}.
This test relates to the previous test input reading testing. After the 
system reads the inputs from the software interface, the system starts
calculating the outputs.


\wss{It would be nice to have a blurb here to explain why the subsections below
cover the requirements. References to the SRS would be good. If a section
coverstests for input constraints, you should reference the data constraints
table in the SRS.} 

\paragraph{The Robustness of the Calculation}

\begin{enumerate}
 

\item{CalculateOutput-id5\\}
Control: Automatic. Input the random cases that satisfied the values' property.

Initial State: Based on the assumption in SRS\cite{YS2019}, $I_{S}$ = 1.35

Input: Input the value of requirements of \progname that drive the calculation
of the Solar intensity. To calculate the solar intensity, it needs
~$\theta_{S_{day}}$

The given inputs are:\\ 
Input (1..50)~$\theta_{S_{day}}$: A random picked rational number\\

Output: The expected result wil for the given inputs is:\\ 
Output (1..50) : Show\verb|``OK, passed 49 tests ''| on the screen\\

\wss{The expected result for the given inputs}

Test Case Derivation: Based on the description in QuickCheck\cite{QuickCheck},
the output is justified if the output shows \verb|``OK, passed input-numbers of tests ''|

 \wss{Justify the expected value given in the Output field}

How test will be performed:
 \begin{itemize} 
\item Implement the test cases with QuickCheck\cite{QuickCheck}, describes in 
section \ref{Planfordv}.
\item Verified the output showing on the screen by the test case derivation
instruction.
\end{itemize} 
\end{enumerate}

\subsubsection{Output Verification}\label{STD_VO} 

This test covers the requirements, R2, R7 and R8, in section \ref{Objectives}. This
test
uses external data\ref{RD} to verify the output. Based on the SRS 
document\cite{YS2019}, the output data constraints can be found in the
table, Output Variables, under the section, Properties of a Correct Solution. 

\paragraph{The Correctness of the Calculation}
\begin{enumerate}




\item{VerifyOutput-id6\\} 

This test case used the external data from [Jacobsonand
Jadhav\cite{JacobsonandJadhav}] as the expected output and the expected
input latitude.\\ 

Control: Automatic. The test cases contain cases that $\Phi_P
= \text{expected input latitude}$ and $\Phi_P \ne \text{expected input
latitude}$. 

Initial State: Based on the assumption in SRS\cite{YS2019}, $I_{S}$:1.35, and
based on the assumption in [Jacobson and Jadhav\cite{JacobsonandJadhav}],
$\mathit{year}_\text{Start}$: 2018
~$\mathit{month}_\text{Start}$: 01 
~$\mathit{day}_\text{Start}$: 01;
~$\mathit{year}_\text{End}$: 2018 
~$\mathit{month}_\text{End}$: 12
~$\mathit{day}_\text{End}$: 31\\ 

Input: Input the value of requirements of
\progname that driven the calculation of the optimal tilt angle. To calculate
the solar intensity, it needs ~$\theta_{S_{day}}$, which is diven by the input
latitude. 

The given inputs are:\\
 Input (1)$\Phi_P$: 64.13 \\ 
Input(2)$\Phi_P$:63.13\\

Output: The expected result wil for the given inputs is the optimal tilt
angle:\\ Output (1) : 43\\ Output (2) : 43\\

\wss{The expected result for the given inputs}

Test Case Derivation: Based on the equation described in SRS\cite{YS2019}, we
get the \textbf{actual result}. Then we calculate the relative error using the
data in the
 [JacobsonandJadhav\cite{JacobsonandJadhav}] as our \textbf{expected result}. 
Therefore, $\text{relative error} \approx 0$ where $\text{relative error} =
 | 1 - \frac{\text{actual result}}{ \text{expected result}} |$ 

\wss{Justify the expected value given in the Output field}

How test will be performed: 
\begin{itemize}
\item Build a linear graph using the expected input latitude as the x-axis 
and expected output as the y-axis. 
\item Input the input values from a file, VerifyOutputId3.txt.
\item Calculate the \textbf{actual result} by the equation descibes in in
SRS\cite{YS2019}
\item Place the point($P_\text{actual input}$)(x-axis: input latitude, y-axis:
actual result) in the linear graph
\item Find the point($P_\text{upper bound}$),the lowest upper bound of
$P_\text{actual input}$ and point($P_\text{lower bound}$), the greatest upper
bound of $P_\text{actual input}$
\item Calculate the area between $P_\text{upper bound}$ and $P_\text{actual
input}$; and $P_\text{lower bound}$ and $P_\text{actual input}$ using the
equation,\\
$\mathit{Area}$ = $\frac{|(x_{\text{input latitude}}-x_{\text{expected
latitude}})| \times |(y_{\text{actual result}}-y_{\text{expected result}})|}{2}$
\item If $\mathit{Area}_{\text{actual input - \textbf{upper bound}}} <
\mathit{Area}_{\text{actual input - \textbf{lower bound}}}$,
then \textbf{expected result} = the y-axis of $P_\text{upper bound}$, otherwise
\textbf{expected result} = the y-axis of $P_\text{lower bound}$
\item Verified the output by the test case derivation instruction. 
\item If all the relative error of the test cases is approximately
0, then the test success, otherwise the test fails. 
\end{itemize}

\item{VerifyOutput-id7\\} 

This test case used the external data from [Landau\cite{Charles2001}]
as the expected output, expected input latitude\\

Control: Automatic. The test cases contain cases that $\Phi_P = \text{expected
input latitude}$, $\Phi_P \ne \text{expected input latitude}$. 

Initial State:
Basedon the assumption in SRS\cite{YS2019}, $I_{S}$: 1.35, and based on the
assumption in [Landau\cite{Charles2001}], the days duration of the winter in
northern hemisphere is from\\
~ $\mathit{year}_\text{Start}$: 2018
~$\mathit{month}_\text{Start}$: 10 
~$\mathit{day}_\text{Start}$: 05 \\
 to
~$\mathit{year}_\text{End}$: 2019 
~$\mathit{month}_\text{End}$: 03
~$\mathit{day}_\text{End}$: 05\\

 Input: Input the value of requirements of
\progname that driven the calculation of the Solar intensity. To calculate the
solar intensity, it needs ~$\theta_{S_{day}}$, which is diven by the input
latitude. 

The given inputs are:\\ 
Input (1)$\Phi_P$: 30 \\ 
Input (2)$\Phi_P$: 31\\

Output: The expected result wil for the given inputs is average the solar
intensity during winter:\\ 
Output (1) : 5.6\\ 
Output (2) : 5.6\\

\wss{The expected result for the given inputs}

Test Case Derivation: Based on the equation described in SRS\cite{YS2019}, we
get the expected result. Therefore, $\text{relative error} \approx 0$ where
$\text{relative error} = | 1 - \frac{\text{actual output}}{ \text{expected
output}} |$

\wss{Justify the expected value given in the Output field}

How test will be performed: 

\begin{itemize} 
\item Input the input values from a file, VerifyOutputId4.txt. 
\item Calculate the daily solar intensity during winter 
\item Calculate the average solar intensity during winter, using the equation,\\the average solar intensity = $\frac{\text{the sum of the daily solar
intensity}}{\text{days duration of winter}}$
\item Output the average solar intensity as the actual output
\item Verified the output by the test case derivation instruction. 
\item If all the relative error of the test cases is approximately 0, then the
test success, otherwise the test fails.
\end{itemize}
\end{enumerate}

\subsection{Tests for Nonfunctional Requirements}

\wss{The nonfunctional requirements for accuracy will likely just reference the
appropriate functional tests from above. The test cases should mention
reportingthe relative error for these tests.} \wss{Tests related to usability
could
include conducting a usability test and survey.} \subsubsection{Correctness}
\paragraph{The correctness of the System}

\begin{enumerate} 

\item{correctness-id1\\} 

Type: Dynamic analysis
The outputs of the system test under section \ref{STD_VO}. To get a
reliable output, \progname is expected to pass all the test cases under the
section \ref{STD_VO}.

How test will be performed: Active the test cases under section
\ref{STD_VO}, when the code is modified.

\item{correctness-id2\\}

Type: Dynamic analysis - Code Coverage

Use the code coverage tool, Haskell Program Coverage(HPC), to test the code
coverage. The description of HPC can be found in section \ref{Planfordv}.

Input/Condition: The main code of \progname

Output: The percentage of the coverage 

How test will be performed:

\begin{itemize} 
\item Use the compiler, ghc-6.8.1 or later version, to active Haskell Program
Coverage.
\item Enable hpc with the command line, -fhpc.
\item The test case success, if the output gets the 100% of the coverage,
otherwise
the test case fails. 
\end{itemize}
\end{enumerate} 


\subsubsection{Portability} 
\paragraph{The portability of the system}
\begin{enumerate}

\item{portability-id3\\}

Type: Manual
					
Initial State: -
					
Input/Condition: Implement \progname in diverse environments.
					
Output/Result: If the \progname works functionally.
					
How test will be performed:
 \begin{itemize}
\item Implement \progname ~on the
virtual machines with the system enviroment of Windows 10, MacOs. 
\item Run every function of \progname. 
\item If function works, then success, otherwise fail. 
\end{itemize} 
\end{enumerate}


\newpage
\subsection{Traceability Between Test Cases and Requirements}


\begin{table}[h!]
\centering
\begin{tabular}{|c|c|c|c|c|c|c|c|c|}
\hline        
	& R1& R2 & R3 &R4 & R5 &R6  &R7 &R8 \\
\hline
id1        & X &    &     &    &    &    &   &     \\ \hline
id2        & X &    &     &    &    &    &   &     \\ \hline
id3        & X &    &     &    &    &    &   &     \\ \hline
id4        & X &    &     &    &    &    &   &     \\ \hline
id5        &    &    & X  &    &    &    &   &     \\  \hline
id6      &    & X &     & X &    & X & X & X   \\ \hline
id 7      &    &    &     & X &    &    & X & X  \\ \hline


\hline
\end{tabular}
\caption{Traceability Between Functional Requirements Test Cases and
Requirements}
\label{Table:trace}
\end{table}

\begin{table}[h!]
\centering
\begin{tabular}{|c|c|c|c|c|c|c|}
\hline        
	& NFR1& NFR2 & NFR3 &NFR4 & NFR5 &NFR6 \\
\hline
id1        & X &    &     &    &    &     \\ \hline
id2        & X &    &     &    &    &     \\ \hline
id3        &    &    &     &    &    & X  \\ \hline


\hline
\end{tabular}
\caption{Traceability Between NonFunctional Requirements Test Cases and
Requirements}
\label{Table:trace}
\end{table}


\wss{Provide a table that shows which test cases are supporting which
requirements.}


\newpage

\bibliographystyle{unsrtnat}
\bibliography{../../../refs/References} 
%\bibliography{../../docs/SRS}


\newpage

\section{Appendix}


\subsection{Symbolic Parameters}

  \noindent \begin{tabular}{l l l} 
    \toprule		
    \textbf{Symbol} & \textbf{Description} & \textbf{Value}\\
    \midrule 
    $I_{S}$ & Solar insensity &1.35  \\
    \bottomrule
  \end{tabular}
  %	\caption{Provide a caption}
%\end{table}


 \subsection{Usability Survey Questions?}
\wss{This is a section that would be appropriate for some projects.}


\begin{table}[h!]
  \noindent \begin{tabular}{l l l} 
    \toprule		
    \textbf{Symbol} & \textbf{Answer} \\
    \midrule 
    Do you think this software helps? & Yes/ No  \\
    Do you think this software is easy to use? & Yes/ No  \\
    Do you think this software is easy to use for elders? & Yes/ No  \\
    Do you think this software is easy to use for children(under 12)? & Yes/ No  \\
    Do you think this software works flawlessly? & Yes/ No  \\
    How many time you open this software in a week? & \_\_\_\_\_\_\_\_\_times  \\
	 How many stars  would you like to give to this software?(1 - 10 starts) & \_\_\_\_\_\_\_\_\_starts  \\
    Do you like to recommend  this software to others? & Yes/ No  \\
    \bottomrule
  \end{tabular}
  \caption{Survey for home users}
\end{table}

\begin{table}[h!]
  \noindent \begin{tabular}{l l l} 
    \toprule		
    \textbf{Symbol} & \textbf{Answer} \\
    \midrule 
    The steps take to get the optimum tilt angle & \_\_\_\_\_\_\_\_\_steps  \\
    How long it need to get the optimum tilt angle & \_\_\_\_\_\_\_\_\_times  \\
    The steps take to get the expected solar energy gaining & \_\_\_\_\_\_\_\_\_steps  \\
    How long it need to get the expected solar energy gaining & \_\_\_\_\_\_\_\_\_times  \\
    The power consume after using the software for an hour & \_\_\_\_\_\_\_\_\_\%  \\
    The memory consume after install the software  & \_\_\_\_\_\_\_\_\_\%  \\
    How many error alerts you take at the fist time using the software? & \_\_\_\_\_\_\_\_\_times  \\
    How many error alerts you take at the second time using the software? & \_\_\_\_\_\_\_\_\_times  \\
    \bottomrule
  \end{tabular}
  \caption{Survey for researching  group}
\end{table}


\end{document}