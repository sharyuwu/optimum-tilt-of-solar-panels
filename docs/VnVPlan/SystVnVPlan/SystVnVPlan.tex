\documentclass[12pt, titlepage]{article}

\usepackage{amsmath, mathtools}
\usepackage{booktabs}
\usepackage{tabularx} 
\usepackage{hyperref} 
\hypersetup{ colorlinks, citecolor=black, filecolor=black, linkcolor=red,
urlcolor=blue }

\usepackage{xr}
\externaldocument{../../docs/SRS/SRS}


% For easy change of table widths
\newcommand{\colZwidth}{1.0\textwidth}
\newcommand{\colAwidth}{0.13\textwidth}
\newcommand{\colBwidth}{0.82\textwidth}
\newcommand{\colCwidth}{0.1\textwidth}
\newcommand{\colDwidth}{0.05\textwidth}
\newcommand{\colEwidth}{0.8\textwidth}
\newcommand{\colFwidth}{0.17\textwidth}
\newcommand{\colGwidth}{0.5\textwidth}
\newcommand{\colHwidth}{0.28\textwidth}

%\usepackage[round]{natbib}
\usepackage[square,sort,comma,numbers]{natbib} 
%% Comments

\usepackage{color}

\newif\ifcomments\commentstrue

\ifcomments
\newcommand{\authornote}[3]{\textcolor{#1}{[#3 ---#2]}}
\newcommand{\todo}[1]{\textcolor{red}{[TODO: #1]}}
\else
\newcommand{\authornote}[3]{}
\newcommand{\todo}[1]{}
\fi

\newcommand{\wss}[1]{\authornote{blue}{SS}{#1}} 
\newcommand{\plt}[1]{\authornote{magenta}{TPLT}{#1}} %For explanation of the template
\newcommand{\an}[1]{\authornote{cyan}{Author}{#1}}

%% Common Parts

\newcommand{\progname}{ProgName} % PUT YOUR PROGRAM NAME HERE %Every program
                                % should have a name


\begin{document}
\title{Project Title: System Verification and Validation Plan for Sun Catcher}

\author{Sharon (Yu-Shiuan) Wu} \date{\today} \maketitle
\pagenumbering{roman}

\section{Revision History}

\begin{tabularx}{\textwidth}{p{3cm}p{2cm}X}
\toprule {\bf Date} & {\bf Version} & {\bf Notes}\\ 
\midrule 
2019/11/04 1 & 1.0 & First version of System VnV\\ 
Date 2 & 1.1 & -\\ 
\bottomrule 
\end{tabularx} \newpage
\tableofcontents

\listoftables

\listoffigures

\newpage

\section{Symbols, Abbreviations and Acronyms}

\renewcommand{\arraystretch}{1.2}
\begin{tabular}{l l} \toprule \textbf{symbol} & \textbf{description}\\
 \midrule T & Test\\ 
SC & Sun Catcher\\ 
$\Phi_P$ & the latitude of the solar panel \\ 
$I_{S}$ &  the intensity of the sun measured by the satellites \\ 
$P_{A_{h}}$ & the height of the solar panel \\ 
$P_{A_{w}}$ & the width of the solar panel \\
$\theta_{S_{\text{date}}}$ & zenith angle of
sun in the date of a sequence of day
  \\ 
$\mathit{year}_\text{Start}$ & the year of the calcuation's starting date\\ 
$\mathit{month}_\text{Start}$ & the month of the calcuation's starting date\\ 
$\mathit{day}_\text{Start}$ & the day of the calcuation's starting date\\ 
$\mathit{year}_\text{End}$ & the year of the calcuation's ending date\\ 
$\mathit{month}_\text{End}$ & the month of the calcuation's ending date\\ 
$\mathit{day}_\text{End}$ & the day of the calcuation's ending date\\ 
\bottomrule
\end{tabular}\\ 

\wss{symbols, abbreviations or acronyms -- you can simply
  reference the SRS tables, if appropriate}

\wss{It would have been better to not
include these comments, since it will become confusing as I add new comments.
Using the diff feature on GitHub will help you to look at only the new comments.}

\wss{For the table above, I suggest removing the word ``the'' from the start of
  most of the descriptions.  You should also consistently capitalize the first
  word of each description.}

\newpage

\pagenumbering{arabic}

This document provides an outline of the system verification and validation for
the \progname. The general introduction section provides readers a summary of
the functions in \progname{} and the related documents as the resources for
testing. The plan section provides readers the plan for verifying and
validating SC's Software Requirements Specification (SRS) and introduces the
method, tools, and external data to implement the testing. The system test
description
provides the readers with the test cases related to functional and nonfunctional
requirements in \progname. The test builds for uncovering the errors and
boosting the confidence of the software while ensuring an acceptable
performance is provided. 

\wss{provide an introductory blurb and roadmap of the Verification and
Validation plan}


\section{General Information}

This section descrips the purpose of this document. The section Summary describes the general purpose of the software: \progname. The detail of the goal of \progname can be found inside the SRS document. The section Objectives descrides the object of  System VnV plan. It reflects to the requirements of \progname. The detail of the requirements can be found inside the SRS document. 

\wss{There should usually be text between two section headings.  In this case a
  ``roadmap'' of this section would be helpful.}
\an{Get it.}
\subsection{Summary}

The subsections below are the test cases of \progname. \progname ~is the
software that calculates the optimum tilt angle of the days duration that
decided by users. Then it takes the calculated angle to estimate the optimal
solar energy output.

\wss{Say what software is being tested. Give its name and a brief overview of
its general functions.}

\subsection{Objectives}\label{Objectives}

The goal of the test is to build confidence in the software's correctness and
strengthen the robustness of the software by testing its software requirements. The functional and nonfunctional requirements of \progname, and its related equations and its values' constraint are found in the SRS \cite{YS2019}. The Verification and Validation Plan follows the requirements described in the SRS \cite{YS2019}.

The object of the Verification and Validation Plan is:
\noindent \begin{itemize}

\item[ ] To build the robustness of the system input.
\item[ ] To build the confidence of the correctness of the system output.

\noindent \end{itemize}

\wss{You don't really want to copy and paste the SRS requirements.  This will cause maintenance problems.  Referencing the SRS here should be enough.  You
  can also use cross-references between documents to reference specific parts of
  your SRS.}\an{Deleted}


\wss{State what is intended to be accomplished. The objective will be around
the qualities that are most important for your project. You might have something
like: ``build confidence in the software correctness,'' ``demonstrate adequate
usability.'' etc. You won't list all of the qualities, just those that are most
important.}

\subsection{Relevant Documentation}\label{RD}

The document that related to the test objectives.

\begin{itemize} 
\item Software Requirements Specification for \progname [Wu~\cite{YS2019}] 
 \end{itemize} 
\wss{Reference relevant documentation. This will definitely include your SRS}

 \section{Plan}	
\subsection{Verification and Validation Team} The test team includes the
following members:\\ 
Main reviewer: Sharon (Yu-Shiuan) Wu\\ 
Secondary reviewer: Deema Alomair, Bo Cao, Sasha Soraine, Zhi Zhang, 
and Doctor Smith\\
\wss{Probably just you. :-)}
\subsection{SRS Verification Plan}

\begin{itemize}

\item Get feedback from the reviewers: Sasha Soraine, Zhi Zhang, and Doctor
Smith, after SRS is completed and put to the GitHub.
\item Check the document by using SRS-Checklist and Writing-Checklist before
publishing to GitHub.

\end{itemize}

\wss{List any approaches you intend to use for SRS verification. This may just
be ad hoc feedback from reviewers, like your classmates, or you may have
something more rigorous/systematic in mind..}

\subsection{Design Verification Plan}
\begin{itemize}

\item The design should be verified by complete and success the test cases in
the system VnV plan under section \ref{STD}.
\item The design should be verified by complete and success the test cases in the UnitVnVPlan. 
\item The design should satisfy all the functional and nonfunctional requirements stated in the SRS document \cite{YS2019}.

\end{itemize}

\wss{How are you going to verify the design?}

\wss{Plans for design verification}\label{Planfordv}


\subsection{Implementation Verification Plan} 
The following tools will be used to facilitate testing: 

\begin{itemize}
\item Mapping data with the external's data: Using the data provides in the external documents [Landau ] \cite{Charles2001} and [MarkandVijaysinh] \cite{JacobsonandJadhav} to verify the result values. It is used in the test section \ref{tP_CC}

\item Rubber Duck Debugging: Performed by author, Sharon (Yu-Shiuan) Wu. The 
author should verbally explain the code line by line.
\item Haskell Program Coverage: Dynamic Testing Tool, a tool-kit to record 
and display the code coverage of a Haskell Program. It aims to reinforce the 
correctness of the software and to eliminate the infeasibility problems.[Gill
and Runciman] \cite{GillandRunciman}
\item QuickCheck: Automatic testing tool for Haskell programs, a library for
random testing of program properties. It aims to boost the robustness of the 
software.[Claessen] \cite{QuickCheck}
\item Tasty: a testing framework for Haskell, using it to build automatic test cases for \ref{tP_IR} to \ref{tP_IB}
\end{itemize} 

\wss{You should at least point to the tests listed in this document and the unit testing plan.} \wss{In this section you would also give any details of any
plans for static verification of the implementation. Potential techniques
including code walkthroughs, code inspection, static analyzers, etc.}



\subsection{Software Validation Plan}

\progname{} should be valid by satisfied all the functional requirement in SRS 
plan.

Based on the physical concept of \progname, the author, Yu-Shiauan Wu, should
record the actual solar energy by using the output from \progname. Then verify
whether the calculated tilt angle can increase the energy gaining.

\wss{Validation means a comparison to actual experiments.  I don't think this is what
  you intend to do.}

\wss{If there is any external data that can be used for validation, you should
point to it here. If there are no plans for validation, you should state that
here.}

\section{System Test Description}\label{STD}	
\subsection{Tests for Functional Requirements}

The subsection below is designed to cover the functional requirements of
\progname, which also describes in section \ref{Objectives}.\\
The test is divided into four subsections, which are input reading, input
bounds,
output calculation, and output verification. Input reading testing is designed
for
testing the ability to receive information from the software interface. Input
bounds
testing and output calculation testing are designed for testing the robustness
of the
software. Output verification testing is designed for the correctness of the 
implemented equation.

\wss{Subsets of the tests may be in related, so this section is divided into
different areas. If there are no identifiable subsets for the tests, this level
of document structure can be removed.} \wss{Include a blurb here to explain why
the subsections below cover the requirements. References to the SRS would be
good.}

\subsubsection{Input Reading} \label{tP_IR}

This test covers the requirements of reading the inputs. \progname{}has to identity users' inputs and then assign the values to designated equations or modules.

\wss{Space are frequently missing between words.  Please check for this during
  your editing step.}

\wss{It would be nice to have a blurb here to explain why the subsections below
cover the requirements. References to the SRS would be good. If a section
covers tests for input constraints, you should reference the data constraints
table in the SRS.}

\paragraph{Identity users' input}
\begin{enumerate}

\item{InputReading-id1\\}

Control: Automatic
					
Initial State: No input value
					
Input: ($P_{A_{h}}$, $P_{A_{w}}$)


Output: The expected result will for the given inputs is based on the input value.
The output should display the input value on the screen.


\noindent \begin{tabular}{l l l l} 
    \toprule		
    \textbf{id} & \textbf{Input} & \textbf{Output}\\ 
	\midrule
   id1.1 &  (1455, 665)  & (1455, 665)\\
   id1.2 &  (1455.54, 665.13) & (1455.54, 665.13)\\
    \bottomrule
  \end{tabular}
  %	\caption{Provide a caption}
%\end{table}

\wss{The expected result for the given inputs}

Test Case Derivation: The test case is testing the ability of capture the input value. Then output the designed variables for  $P_{A_{h}}$ and $P_{A_{w}}$. Calculate an \textbf{absolute error}, such that $|$ expected output - actaul output $|$.
\wss{Justify the expected value given in the Output field} 

How the test will be performed:

\begin{itemize} 
\item Input the designed test case input values by using testing framework, Tasty. 
\item Display the output of the assigned variables on the screen.
\item  Display the absolute error on the screen.
\item  If all the absolute error = 0 then the test case pass. Otherwise the test case fail.
\end{itemize}

\wss{This test case should be automated, not manual.  You do not need to enter
  the values from the keyboard.  The values can be entered into code and the
  appropriate function(s) called.  This test is really a test that your getters
  and setters are working.} \an{Yes, plan to use testing framework  Tasty to implement automatic testing.}
  
\item{InputReading-id2\\} 

Control: Automatic. 

Initial State: No input value\\
Input: $\Phi_P$ \wss{I've fixed some spelling mistakes and spacing mistakes.  Please make an effort for more attention to detail.} \an{Yes, I will do my proof read.} \\



Output: The expected result will for the given inputs is based on the input value.
The output should display the input value on the screen.


\noindent \begin{tabular}{l l l l} 
    \toprule		
    \textbf{id} & \textbf{Input} & \textbf{Output}\\ 
	\midrule
   id2.1 &  90  & 90\\
   id2.2 & -90  & -90\\
   id2.3 & 3.2  & 3.2\\
   id2.4 & -3.2  & -3.2\\
   id2.5 & 0  & 0\\
    \bottomrule
  \end{tabular}
  %	\caption{Provide a caption}
%\end{table}

\wss{The expected result for the given inputs}
  
The test case is testing the ability of capture the input value. Then output the designed variables for  $\Phi_P$. Calculate an \textbf{absolute error}, such that $|$ expected output - actaul output $|$.

\wss{The test case for
  invalid input should be a separate test case.  In the invalid input case an
  exception should be raised.  You can write a unit test that verifies that the
  correct exception is raised, when it is expected to be raised.} \an{OK, move the input boundary to unit testing.}

\wss{Justify the expected value given in the Output field} 

How the test will be performed:

\begin{itemize} 
\item Input the designed test case input values by using testing framework, Tasty. 
\item Display the output of the assigned variables on the screen.
\item  Display the absolute error on the screen.
\item  If all the absolute error = 0 then the test case pass. Otherwise the test case fail.
\end{itemize}

\item{InputReading-id3\\}
Control: Automatic.\\
 
Initial State: No any given value.\\
Input: ($\mathit{year}_\text{Start}$, $\mathit{month}_\text{Start}$, $\mathit{day}_\text{Start}$)  ($\mathit{year}_\text{End}$, $\mathit{month}_\text{End}$, $\mathit{day}_\text{End}$)\\ 


Output: The expected result will for the given inputs is based on the input value.
The output should display the input value on the screen.

\noindent \begin{tabular}{l l l l} 
    \toprule		
    \textbf{id} & \textbf{Input} & \textbf{Output}\\ 
	\midrule
   id3.1 & (2020, 02, 28) - (2021, 02, 28) & (2020, 02, 28)  (2021, 02, 28)\\
   id3.2 & (2020, 02, 28) - (2020, 02, 28)  & (2020, 02, 28)  (2020, 02, 28)\\
    \bottomrule
  \end{tabular}
  %	\caption{Provide a caption}
%\end{table}

\wss{The expected result for the given inputs}

Test Case Derivation: The test case is testing the ability of capture the input value. Then output the designed variables for  ($\mathit{year}_\text{Start}$, $\mathit{month}_\text{Start}$, $\mathit{day}_\text{Start}$) and ($\mathit{year}_\text{End}$, $\mathit{month}_\text{End}$, $\mathit{day}_\text{End}$). Calculate an \textbf{absolute error}, such that $|$ expected output - actaul output $|$.

\wss{Justify the expected value given in the Output field} 

How the test will be performed:

\begin{itemize} 
\item Input the designed test case input values by using testing framework, Tasty. 
\item Display the output of the assigned variables on the screen.
\item  Display the absolute error on the screen.
\item  If all the absolute error = 0 then the test case pass. Otherwise the test case fail.
\end{itemize}

\end{enumerate}

\subsubsection{Output Calculation}\label{tP_OC}


This test covers the requirements, R3 to R5, in section \ref{Objectives}.
This test relates to the previous test input reading testing. After the 
system reads the inputs from the software interface, the system starts
calculating the outputs.


\wss{It would be nice to have a blurb here to explain why the subsections below
cover the requirements. References to the SRS would be good. If a section
covers tests for input constraints, you should reference the data constraints
table in the SRS.} 


\subsubsection{Output Verification}\label{tP_VO} 

This test covers the requirements of the verification of the outputs. This test
uses external data\ref{RD} to verify the output.

\paragraph{The Correctness of the Calculation}\label{tP_CC}
\begin{enumerate}


\item{VerifyOutput-id4\\} 

This test case used the external data from [Jacobsonand
Jadhav] \cite{JacobsonandJadhav} as the expected output and the expected
input latitude.\\ 

Control: Automatic. The test cases contain cases that $\Phi_P
= \text{expected input latitude}$ and $\Phi_P \ne \text{expected input
latitude}$. 

Initial State: Based on the assumption in SRS\cite{YS2019}, $I_{S}$:1.35, and
based on the assumption in [Jacobson and Jadhav] \cite{JacobsonandJadhav},
$\mathit{year}_\text{Start}$: 2018
~$\mathit{month}_\text{Start}$: 01 
~$\mathit{day}_\text{Start}$: 01;
~$\mathit{year}_\text{End}$: 2018 
~$\mathit{month}_\text{End}$: 12
~$\mathit{day}_\text{End}$: 31\\ 

Input: $\Phi_P$\\

Output: The expected result will for the given inputs is the optimal tiltangle


\noindent \begin{tabular}{l l l l} 
    \toprule		
    \textbf{id} & \textbf{Input} & \textbf{Output}\\ 
	\midrule
   id4.1 &  64.13 & 43\\
   id4.2 &  63.13 & 43\\
    \bottomrule
  \end{tabular}
  %	\caption{Provide a caption}
%\end{table}

\wss{The expected result for the given inputs}

Test Case Derivation: Based on the equation described in SRS\cite{YS2019}, we
get the \textbf{actual result}. Then we calculate the relative error using the
data in the
 [JacobsonandJadhav]\cite{JacobsonandJadhav} as our \textbf{expected result}. 
Calculate a $\textbf{relative error}$ such that $| 1 - \frac{\text{actual result}}{ \text{expected result}} |$ 

\wss{Justify the expected value given in the Output field}

How the test will be performed: 
\begin{itemize}
\item Build a linear graph using the expected input latitude as the x-axis 
and expected output as the y-axis. 
\item Input the input values from a file, verifyOutputId6.txt.
\item Calculate the \textbf{actual result} by the equation descibes in in
SRS\cite{YS2019}
\item Place the point($P_\text{actual input}$)(x-axis: input latitude, y-axis:
actual result) in the linear graph
\item Find the point($P_\text{upper bound}$),the lowest upper bound of
$P_\text{actual input}$ and point($P_\text{lower bound}$), the greatest upper
bound of $P_\text{actual input}$
\item Calculate the area between $P_\text{upper bound}$ and $P_\text{actual
input}$; and $P_\text{lower bound}$ and $P_\text{actual input}$ using the
equation,\\
$\mathit{Area}$ = $\frac{|(x_{\text{input latitude}}-x_{\text{expected
latitude}})| \times |(y_{\text{actual result}}-y_{\text{expected result}})|}{2}$
\item If $\mathit{Area}_{\text{actual input - \textbf{upper bound}}} <
\mathit{Area}_{\text{actual input - \textbf{lower bound}}}$,
then \textbf{expected result} = the y-axis of $P_\text{upper bound}$, otherwise
 \textbf{expected result} = the y-axis of $P_\text{lower bound}$
\item Verified the output by the test case derivation instruction. 
\item Display the relative error in the file, resultVerifyOutputId6.txt
\item If all the \textbf{relative error} $\leq$ errorTolerance, then the test success, otherwise the test fails. 
\end{itemize}

\item{VerifyOutput-id5\\} 


This test case used the external data from [Landau]\cite{Charles2001}  \wss{This
  is redundant to manually do the citation and to use BibTeX.  Just use BibTeX.
  If you want to use the author year style, use the Natbib option for BibTeX.} as the expected output, expected input latitude\\

Control: Automatic. The test cases contain cases that $\Phi_P = \text{expected
input latitude}$, $\Phi_P \ne \text{expected input latitude}$. 

Initial State:
Basedon the assumption in SRS\cite{YS2019}, $I_{S}$: 1.35, and based on the
assumption in [Landau]\cite{Charles2001}, the days duration of the winter in
northern hemisphere is from\\
~ $\mathit{year}_\text{Start}$: 2018
~$\mathit{month}_\text{Start}$: 10 
~$\mathit{day}_\text{Start}$: 05 \\
 to
~$\mathit{year}_\text{End}$: 2019 
~$\mathit{month}_\text{End}$: 03
~$\mathit{day}_\text{End}$: 05\\

Input: $\Phi_P$\\


Output: The expected result will for the given inputs is average the solar
intensity during winter.\\ 

\noindent \begin{tabular}{l l l l} 
    \toprule		
    \textbf{id} & \textbf{Input} & \textbf{Output}\\ 
	\midrule
   id5.1 & 30 & 5.6\\
   id5.2 & 31 & 5.6\\
    \bottomrule
  \end{tabular}
  %	\caption{Provide a caption}
%\end{table}

\wss{Use table to summarize your test cases.}\an{Yes}

\wss{The expected result for the given inputs}

Test Case Derivation: Based on the equation described in SRS\cite{YS2019}, we get the expected result. Calculate a $\textbf{relative error}$ such that $| 1 - \frac{\text{actual output}}{ \text{expected
output}} |$ \wss{Your error will not be 0.  It will be a small number.  Your
test case should require reporting this number.} \an{Yes, I was trying to express the relative error will be close to 0 by $\approx 0$. }

\wss{Justify the expected value given in the Output field}

How the test will be performed: 

\begin{itemize} 
\item Input the input values from a file, verifyOutputId7.txt. 
\item Calculate the daily solar intensity during winter 
\item Calculate the average solar intensity during winter, using the equation,\\the average solar intensity = $\frac{\text{the sum of the daily solar
intensity}}{\text{days duration of winter}}$
\item Display the relative error in the file, resultVerifyOutputId7.txt. 
\item Output the average solar intensity as the actual output
\item Verified the output by the test case derivation instruction. 
\item If all the \textbf{relative error}  $\leq$ errorTolarance, then the
test success, otherwise the test fails.
\end{itemize}
\end{enumerate}

\subsection{Tests for Nonfunctional Requirements}

\wss{The nonfunctional requirements for accuracy will likely just reference the
appropriate functional tests from above. The test cases should mention
reporting the relative error for these tests.} \wss{Tests related to usability
could
include conducting a usability test and survey.}

\subsubsection{Reliability}
\paragraph{The reliability of the System}

\begin{enumerate} 
\item{reliability-id1\\} 
Type: Manual

Input/Condition: Software requirements

Output: Test report.

How the test will be performed: Follow the instruction of the software requirements. Provide the test report.\\
The test case for the \progname requirements is from section \ref{tP_IR} to section \ref{tP_CC}

\item{reliability-id2\\} 

Input/Condition: Software requirements changed.

Output: The corresponding trace table \ref{Table:Rtrace}.

How the test will be performed: If the software requirements change, then the trace table has to modify according to the changing. The trace table has to match the relationship between software requirements and the test cases.

\end{enumerate} 

\subsubsection{Correctness}
\paragraph{The correctness of the System}

\begin{enumerate} 


\item{correctness-id3\\}

Type: Dynamic analysis - Code Coverage \wss{good idea to report this number.}
\an{Yes, it does. My output will be the percentage of the coverage.}

Use the code coverage tool, Haskell Program Coverage(HPC), to test the code
coverage. The description of HPC can be found in section \ref{Planfordv}.

Input/Condition: The main code of \progname

Output: The number of how percentage that the software has been coverage by the implementation.

How the test will be performed:

\begin{itemize} 
\item Use the compiler, ghc-6.8.1 or later version, to active Haskell Program
Coverage.
\item Enable hpc with the command line, -fhpc.
\item Implement the code of the \progname
\item Display the coverage in the file "correctnessid1.txt".
\item The test case success, if the output gets the 100\% of the coverage,
otherwise
the test case fails. 
\end{itemize}
\end{enumerate} 


\subsubsection{Portability} 
\paragraph{The portability of the system}
\begin{enumerate}

\item{portability-id4\\}

Type: Manual
					
Initial State: -
					
Input/Condition: Implement \progname in diverse environments.
					
Output/Result: The result from all the test case.
					

How the test will be performed:
\begin{itemize}
\item Implement \progname on the
virtual machines with Windows operating system like Windows 10, Windows 8, Windows 7.
\item Implement \progname on the
virtual machines with the Mac operating system like macOS Catalina , macOS Mojave

\item Implement \progname on the
virtual machines with the Linux operating system
\wss{What is
  the OS and what is the virtual OS?  Why not Linux too?  Virtual box is easy to
install and run.}\an{Yes, I have make the statement more explicite}
\item Run the test cases in system VnV plan and unit VnV plan in all of the identified environment
.\wss{What do you mean to run every
    function?  Why not simply say that the test suite will be run in all of the
    identified environments?}\an{Yes, that is what I mean}
\item If function works, then success, otherwise fail.
\end{itemize}

\end{enumerate}



\subsubsection{Usability} 
\paragraph{The usability of the system}
\begin{enumerate}

\item{usability-id5\\}

Input/Condition: The output result of the \progname

Output/Result: The energy absorption of the shows in the result should be increased by the time of adjustments the solar panel.

How the test will be performed:\\
 \progname should provide the optimum angle of adjust one time and adjust multiple times during the inputs' period.

\begin{itemize} 
\item Output the result to the file "FinalResult.txt".
\item Observe the result, if the energy absorption increases by the time of adjustments, then the test case success. Otherwise the test case fail.
\end{itemize} 

\item{usability-id6\\}

Input/Condition: Usability Survey Questions, describing in the Table \ref{survey}

Output/Result: \\
$\#$ Based on the number of inputs.
The steps take to get the optimum tilt angle should be $\leq$ 4 steps

The steps take to get the expected solar energy gaining $\leq$ 4 steps

How many error alerts you take at the fist time using the software? $\leq$ 2 alerts

How many error alerts you take at the second time using the software? $\leq$ 2 alerts\\

$\#$ Based on the scale of the software.
How long it need to get the optimum tilt angle $\leq$ 1 mins

How long it need to get the expected solar energy gaining $\leq$ 1 mins

The power consume after using the software for an hour $\leq$ 1 % of the bettery

The memory consume after install the software $\leq$ 1 % of the memory

How the test will be performed:
The survey should be filled by the people with or without technology background. \\
The result of the survey are expected as above.

\item{usability-id7\\}

Input/Condition: Usability Survey Questions, describing in the Table \ref{homeSurvey}

Output/Result: \\
The Yes/No questions in the survey should be all filled with Yes.

How many time you open this software in a week? $\geq$ 0\\

10 $\leq$ How many stars  would you like to give to this software?(1 - 10 starts) 10 $\geq$ answer$\geq$ 7

How the test will be performed:
The survey should be filled by the people who have been using the software for a period of time. \\
The result of the survey are expected as above.

\end{enumerate}



\wss{Your project is fairly straightforward Sharon.  If you want to aim for a
  maximum mark in this course, you'll need to do more than just implement your
  project.  One place where you could add some extra challenge to the project is
  to consider assessing usability.  I'm open to discussion if there is something
  else you would like to add instead.  I see that you have a usability survey in
  the appendix, but you do not reference it from the main document.}

\newpage
\subsection{Traceability Between Test Cases and Requirements}


\begin{table}[h!]
\centering
\begin{tabular}{|c|c|c|c|c|c|c|c|c|c|}
\hline        
	& R1& R2 & R3 &R4 & R5 &R6  &R7 &R8 &R9 \\
\hline
id1        & X &    &     &    &    &    &   &    &    \\ \hline
id2        &    & X &     &    &    &    &   &    &     \\ \hline
id3        &     & X &     &    &    &    &   &    &    \\ \hline
id4        &     & X &     &    &    &    &   &    &      \\ \hline
id5        &    &    &     & X &    &    &   &    &     \\  \hline


\hline
\end{tabular}
\caption{Traceability Between Functional Requirements Test Cases and
Requirements}
\label{Table:Rtrace}
\end{table}

\begin{table}[h!]
\centering
\begin{tabular}{|c|c|c|c|c|c|c|}
\hline        
	& NFR1& NFR2 & NFR3 &NFR4 & NFR5 &NFR6 \\
\hline
id1        & X &    &     &    &    &     \\ \hline
id2        & X &    &     &    &    &     \\ \hline
id3        &    &    &     &    &    & X  \\ \hline
id4        &    &    &     &    &    & X  \\ \hline
id5        &    &    &     &    &    & X  \\ \hline
id6        &    &    &     &    &    & X  \\ \hline


\hline
\end{tabular}
\caption{Traceability Between NonFunctional Requirements Test Cases and
Requirements}
\label{Table:NFRtrace}
\end{table}


\wss{Provide a table that shows which test cases are supporting which
requirements.}


\newpage

\bibliographystyle{plain}
\bibliography{../../../refs/References} 
%\bibliography{../../docs/SRS}


\newpage

\section{Appendix}


\subsection{Symbolic Parameters}

  \noindent \begin{tabular}{l l l} 
    \toprule		
    \textbf{Symbol} & \textbf{Description} & \textbf{Value}\\
    \midrule 
    $I_{S}$ & Solar insensity &1.35  \\
errorTolerance & Error Tolerance & 1\\
    \bottomrule
  \end{tabular}
  %	\caption{Provide a caption}
%\end{table}


\subsection{Usability Survey Questions}
\wss{This is a section that would be appropriate for some projects.}


\begin{table}[h!]
  \noindent \begin{tabular}{l l l} 
    \toprule		
    \textbf{Symbol} & \textbf{Answer} \\
    \midrule 
    Do you think this software helps? & Yes/ No  \\
    Do you think this software is easy to use? & Yes/ No  \\
    Do you think this software is easy to use for elders? & Yes/ No  \\
    Do you think this software is easy to use for children(under 12)? & Yes/ No  \\
    Do you think this software works flawlessly? & Yes/ No  \\
    How many time you open this software in a week? & \_\_\_\_\_\_\_\_\_times  \\
	 How many stars  would you like to give to this software?(1 - 10 starts) & \_\_\_\_\_\_\_\_\_starts  \\
    Do you like to recommend  this software to others? & Yes/ No  \\
    \bottomrule
  \end{tabular}
  \caption{Survey for home users}
   \label{homeSurvey}
\end{table}

\begin{table}[h!]
  \noindent \begin{tabular}{l l l} 
    \toprule		
    \textbf{Symbol} & \textbf{Answer} \\
    \midrule 
    The steps take to get the optimum tilt angle & \_\_\_\_\_\_\_\_\_steps  \\
    How long it need to get the optimum tilt angle & \_\_\_\_\_\_\_\_\_times  \\
    The steps take to get the expected solar energy gaining & \_\_\_\_\_\_\_\_\_steps  \\
    How long it need to get the expected solar energy gaining & \_\_\_\_\_\_\_\_\_times  \\
    The power consume after using the software for an hour & \_\_\_\_\_\_\_\_\_\%  \\
    The memory consume after install the software  & \_\_\_\_\_\_\_\_\_\%  \\
    How many error alerts you take at the fist time using the software? & \_\_\_\_\_\_\_\_\_times  \\
    How many error alerts you take at the second time using the software? & \_\_\_\_\_\_\_\_\_times  \\
    \bottomrule
  \end{tabular}
  \caption{Survey for researching  group}
   \label{survey}
\end{table}


\end{document}