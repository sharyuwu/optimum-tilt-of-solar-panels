\documentclass[12pt, titlepage]{article}

\usepackage{booktabs}
\usepackage{tabularx}
\usepackage{hyperref}
\hypersetup{
    colorlinks,
    citecolor=black,
    filecolor=black,
    linkcolor=red,
    urlcolor=blue
}
%\usepackage[round]{natbib}
\usepackage[square,sort,comma,numbers]{natbib} 
\input{../../Comments}
%% Common Parts

\newcommand{\progname}{Sun Catcher} % PUT YOUR PROGRAM NAME HERE %Every program
                                % should have a name


\begin{document}

\title{Project Title: Unit Verification and Validation Plan for \progname{}} 
\author{Sharon (Yu-Shiuan) Wu}
\date{\today}
	
\maketitle

\pagenumbering{roman}

\section{Revision History}

\begin{tabularx}{\textwidth}{p{3cm}p{2cm}X}
\toprule {\bf Date} & {\bf Version} & {\bf Notes}\\
\midrule
2019/12/18 & 1.0 & First Version\\
Date 2 & 1.1 & Notes\\
\bottomrule
\end{tabularx}

~\newpage

\tableofcontents

\listoftables

\wss{Do not include if not relevant}

\listoffigures

\wss{Do not include if not relevant}

\newpage

\section{Symbols, Abbreviations and Acronyms}

\renewcommand{\arraystretch}{1.2}
\begin{tabular}{l l} 
  \toprule		
  \textbf{symbol} & \textbf{description}\\
  \midrule 
  T & Test\\
  \bottomrule
\end{tabular}\\

\wss{symbols, abbreviations or acronyms -- you can reference the SRS, MG or MIS
  tables if needed}

\newpage

\pagenumbering{arabic}

This document ... \wss{provide an introductory blurb and roadmap of the
  unit V\&V plan}

\section{General Information}

\subsection{Purpose}

The purpose of this document is to verify the software, \progname{}. The test cases in this document follows the module defined in Designed documentation, MG and MIS. This document is not an repitation of the ducument, SystemVnVPlan.
\wss{Identify software that is being unit tested (verified).}

\subsection{Scope}

In this document, the module, Day ADT Module, will not be tested. Day ADT Module is developed by using external framework ``time: A time library" which is mentain by external developer, Ashley Yakeley. The framework's  document can be found in \cite{time}. Although Day ADT Module is an extension of the external library, it use the functions from the sources directly without overwriting the original sources. The module,  Table-layout Module, has a low priority for verification. The purpose of this module is 
to write the output value to the designed file. Because this module is not effect to the output value, this module is considered having lower priority than others.

\wss{What modules are outside of the scope.  If there are modules that are
  developed by someone else, then you would say here if you aren't planning on
  verifying them.  There may also be modules that are part of your software, but
  have a lower priority for verification than others.  If this is the case,
  explain your rationale for the ranking of module importance.}

\section{Plan}
	
\subsection{Verification and Validation Team}
Main reviewer: Sharon (Yu-Shiuan) Wu\\ 
Secondary reviewer: Doctor Smith\\
\wss{Probably just you.  :-)}

\subsection{Automated Testing and Verification Tools}

\begin{itemize}

\item Haskell Program Coverage: Dynamic Testing Tool, a tool-kit to record 
and display the code coverage of a Haskell Program. It aims to reinforce the 
correctness of the software and to eliminate the infeasibility problems.[Gill
and Runciman] \cite{GillandRunciman}
\item Tasty: a testing framework for Haskell.

\end{itemize} 

\wss{What tools are you using for automated testing.  Likely a unit testing
  framework and maybe a profiling tool, like ValGrind.  Other possible tools
  include a static analyzer, make, continuous integration tools, test coverage
  tools, etc.  Explain your plans for summarizing code coverage metrics.}

\subsection{Non-Testing Based Verification}

\begin{itemize}
\item code walkthrough: Code walkthrough will be hold between Sharon (Yu-Shiuan) Wu and Doctor. Kal. The code notes with clear comment. The code should be writen by using external framework ``HaTeX". The code should not be repeated and redundant.
\item hlint: a framework for giving suggestions on the source code. The code should be not has any warning from hlint.
\end{itemize} 
\wss{List any approaches like code inspection, code walkthrough, symbolic
  execution etc.  Enter not applicable if you do not plan on any non-testing
  based verification.}

\section{Unit Test Description}

The test cases follow the module descibed in designed document, MIS. MIS can be found in the link \url{https://github.com/sharyuwu/optimum-tilt-of-solar-panels/blob/master/docs/VnVPlan/UnitVnVPlan/UnitVnVPlan.pdf}.\\

Input verification Module has the highest priority for verification. The purpose of this module is to test whether or not the input value is valid. Because the software will fail if implement an invalid input, this module has the highest priority. \\

Calculation Modul has the second priority for verification. The purpose of this module is to do necessary calculate for the output value for the requirements of the software. The accuracy of the output if this module will affect the final output value.\\

Control Modul has a lower priority for verification than others. The purpose of this module is to connect and associate other modules. Because this module is not effect on the value of the output, this module has a lower priority.

\wss{Reference your MIS and explain your overall philosophy for test case
  selection.}

\subsection{Tests for Functional Requirements}

\wss{Most of the verification will be through automated unit testing.  If
  appropriate specific modules can be verified by a non-testing based
  technique.  That can also be documented in this section.}

\subsubsection{Module 1}

\wss{Include a blurb here to explain why the subsections below cover the module.
  References to the MIS would be good.  You will want tests from a black box
  perspective and from a white box perspective.  Explain to the reader how the
  tests were selected.}

\begin{enumerate}

\item{test-id1\\}

Type: \wss{Functional, Dynamic, Manual, Automatic, Static etc. Most will
  be automatic}
					
Initial State: 
					
Input: 
					
Output: \wss{The expected result for the given inputs}

Test Case Derivation: \wss{Justify the expected value given in the Output field}

How test will be performed: 
					
\item{test-id2\\}

Type: \wss{Functional, Dynamic, Manual, Automatic, Static etc. Most will
  be automatic}
					
Initial State: 
					
Input: 
					
Output: \wss{The expected result for the given inputs}

Test Case Derivation: \wss{Justify the expected value given in the Output field}

How test will be performed: 

\item{...\\}
    
\end{enumerate}

\subsubsection{Module 2}

...

\subsection{Tests for Nonfunctional Requirements}

\wss{If there is a module that needs to be independently assessed for
  performance, those test cases can go here.  In some projects, planning for
  nonfunctional tests of units will not be that relevant.}

\wss{These tests may involve collecting performance data from previously
  mentioned functional tests.}

\subsubsection{Module ?}
		
\begin{enumerate}

\item{test-id1\\}

Type: \wss{Functional, Dynamic, Manual, Automatic, Static etc. Most will
  be automatic}
					
Initial State: 
					
Input/Condition: 
					
Output/Result: 
					
How test will be performed: 
					
\item{test-id2\\}

Type: Functional, Dynamic, Manual, Static etc.
					
Initial State: 
					
Input: 
					
Output: 
					
How test will be performed: 

\end{enumerate}

\subsubsection{Module ?}

...

\subsection{Traceability Between Test Cases and Modules}

\wss{Provide evidence that all of the modules have been considered.}

\bibliographystyle{plain}
\bibliography{../../../refs/References} 
%\bibliography{SRS}

\newpage

\section{Appendix}

\wss{This is where you can place additional information, as appropriate}

\subsection{Symbolic Parameters}

\wss{The definition of the test cases may call for SYMBOLIC\_CONSTANTS.
Their values are defined in this section for easy maintenance.}

\end{document}