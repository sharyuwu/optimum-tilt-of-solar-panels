\documentclass[12pt, titlepage]{article}

\usepackage{amsmath, mathtools}
\usepackage{booktabs}
\usepackage{tabularx}
\usepackage{hyperref}
\hypersetup{
    colorlinks,
    citecolor=black,
    filecolor=black,
    linkcolor=red,
    urlcolor=blue
}
%\usepackage[round]{natbib}
\usepackage[square,sort,comma,numbers]{natbib} 
%% Comments

\usepackage{color}

\newif\ifcomments\commentstrue

\ifcomments
\newcommand{\authornote}[3]{\textcolor{#1}{[#3 ---#2]}}
\newcommand{\todo}[1]{\textcolor{red}{[TODO: #1]}}
\else
\newcommand{\authornote}[3]{}
\newcommand{\todo}[1]{}
\fi

\newcommand{\wss}[1]{\authornote{blue}{SS}{#1}} 
\newcommand{\plt}[1]{\authornote{magenta}{TPLT}{#1}} %For explanation of the template
\newcommand{\an}[1]{\authornote{cyan}{Author}{#1}}

%% Common Parts

\newcommand{\progname}{ProgName} % PUT YOUR PROGRAM NAME HERE %Every program
                                % should have a name


\usepackage{xr}
\externaldocument{../../docs/SRS/SRS}

\begin{document}

\title{Project Title: Unit Verification and Validation Plan for \progname{}} 
\author{Sharon (Yu-Shiuan) Wu}
\date{\today}
	
\maketitle

\pagenumbering{roman}

\section{Revision History}

\begin{tabularx}{\textwidth}{p{3cm}p{2cm}X}
\toprule {\bf Date} & {\bf Version} & {\bf Notes}\\
\midrule
2019/12/18 & 1.0 & First Version\\
Date 2 & 1.1 & Notes\\
\bottomrule
\end{tabularx}

~\newpage

\tableofcontents

\listoftables

\wss{Do not include if not relevant}

\listoffigures

\wss{Do not include if not relevant}

\newpage

\section{Symbols, Abbreviations and Acronyms}

\renewcommand{\arraystretch}{1.2}
\begin{tabular}{l l} 
  \toprule		
  \textbf{symbol} & \textbf{description}\\
  \midrule 
  T & Test\\
  \bottomrule
\end{tabular}\\

\wss{symbols, abbreviations or acronyms -- you can reference the SRS, MG or MIS
  tables if needed}

\newpage

\pagenumbering{arabic}

This document ... \wss{provide an introductory blurb and roadmap of the
  unit V\&V plan}

\section{General Information}

\subsection{Purpose}

The purpose of this document is to verify the software, \progname{}. The test cases in this document follows the module defined in Designed documentation, MG and MIS. This document is not an repitation of the ducument, SystemVnVPlan.
\wss{Identify software that is being unit tested (verified).}

\subsection{Scope}

In this document, the module, Day ADT Module, will not be tested. Day ADT Module is developed by using external framework ``time: A time library" which is mentain by external developer, Ashley Yakeley. The framework's  document can be found in \cite{time}. Although Day ADT Module is an extension of the external library, it use the functions from the sources directly without overwriting the original sources. The module,  Table-layout Module, has a low priority for verification. The purpose of this module is 
to write the output value to the designed file. Because this module is not effect to the output value, this module is considered having lower priority than others.

\wss{What modules are outside of the scope.  If there are modules that are
  developed by someone else, then you would say here if you aren't planning on
  verifying them.  There may also be modules that are part of your software, but
  have a lower priority for verification than others.  If this is the case,
  explain your rationale for the ranking of module importance.}

\section{Plan}
	
\subsection{Verification and Validation Team}
Main reviewer: Sharon (Yu-Shiuan) Wu\\ 
Secondary reviewer: Doctor Smith\\
\wss{Probably just you.  :-)}

\subsection{Automated Testing and Verification Tools}

\begin{itemize}

\item Haskell Program Coverage: Dynamic Testing Tool, a tool-kit to record 
and display the code coverage of a Haskell Program. It aims to reinforce the 
correctness of the software and to eliminate the infeasibility problems.[Gill
and Runciman] \cite{GillandRunciman}
\item Tasty: a testing framework for Haskell.

\end{itemize} 

\wss{What tools are you using for automated testing.  Likely a unit testing
  framework and maybe a profiling tool, like ValGrind.  Other possible tools
  include a static analyzer, make, continuous integration tools, test coverage
  tools, etc.  Explain your plans for summarizing code coverage metrics.}

\subsection{Non-Testing Based Verification}

\begin{itemize}
\item code walkthrough: Code walkthrough will be hold between Sharon (Yu-Shiuan) Wu and Doctor. Kal. The code notes with clear comment. The code should be writen by using external framework ``HaTeX". The code should not be repeated and redundant.
\item hlint: a framework for giving suggestions on the source code. The code should be not has any warning from hlint.
\end{itemize} 
\wss{List any approaches like code inspection, code walkthrough, symbolic
  execution etc.  Enter not applicable if you do not plan on any non-testing
  based verification.}

\section{Unit Test Description}

The test cases follow the module descibed in designed document, MIS. MIS can be found in the link \url{https://github.com/sharyuwu/optimum-tilt-of-solar-panels/blob/master/docs/VnVPlan/UnitVnVPlan/UnitVnVPlan.pdf}.\\

Input verification Module has the highest priority for verification. The purpose of this module is to test whether or not the input value is valid. Because the software will fail if implement an invalid input, this module has the highest priority. \\

Calculation Modul has the second priority for verification. The purpose of this module is to do necessary calculate for the output value for the requirements of the software. The accuracy of the output if this module will affect the final output value.\\

Control Modul has a lower priority for verification than others. The purpose of this module is to connect and associate other modules. Because this module is not effect on the value of the output, this module has a lower priority.

\wss{Reference your MIS and explain your overall philosophy for test case
  selection.}

\subsection{Tests for Functional Requirements}

\wss{Most of the verification will be through automated unit testing.  If
  appropriate specific modules can be verified by a non-testing based
  technique.  That can also be documented in this section.}


\subsection{Modules}
\subsubsection{Input Verification Module} 
\wss{Include a blurb here to explain why the subsections below cover the module.
  References to the MIS would be good.  You will want tests from a black box
  perspective and from a white box perspective.  Explain to the reader how the
  tests were selected.}

\begin{enumerate}
\item{InputBounds-id1\\} 
This test case tests the capability of capture the invalid input value of $\Phi_P$.

Control: Automatic. 

Initial State: No input value\\
Input: $\Phi_P$ 


Output: The expected result will for the given inputs is based on the input value.
If $\Phi_P$ is between the value of [-90..90], output ``valid". Otherwise ouput ``invalid".


\noindent \begin{tabular}{l l l l} 
    \toprule		
    \textbf{id} & \textbf{Input} & \textbf{Output}\\ 
	\midrule
   id1.1 &  90  & valid\\
   id1.2 & -90  & valid\\
   id1.3 &  89.9  & valid\\
   id1.4 & -89.9  & valid\\
   id1.5 & 0  & valid\\
   id1.6 & 91  & invalid\\
   id1.7 & -91  & invalid\\
   id1.8 & 90.1  & invalid\\
   id1.9 & -90.1  & invalid\\
    \bottomrule
  \end{tabular}
  %	\caption{Provide a caption}
%\end{table}



Test Case Derivation: The test case is testing the ability of capture the input value, $\Phi_P$. The constraint of $\Phi_P$  is driven by the geographic coordinate that the range from North pole to South pole is from 90 to -90. This constraint can be found in SRS \cite{YS2019}.


How the test will be performed:

\begin{itemize} 
\item Input the designed test case input values by using testing framework, Tasty. 
\item Write the output in the file, resultInputBounds-id1.txt.
\item Verified the output showing in the file by match the result with the expected output above.
\item If all the result match with the expected output, then the test case success.
\item Otherwise the test case fails.
\end{itemize}

\item{InputBounds-id2\\}

This test case test the capability of capture the invalid value of dayT.

Control: Automatic. 
  
Initial State: No any given value.\\

Input: Based on calendar, the given inputs is: 
($\mathit{year}_\text{Start}$, $\mathit{month}_\text{Start}$, $\mathit{day}_\text{Start}$)

Output: The expected result will for the given inputs is showing if the input date exist in the standard calendar.\\
The existence of the day is based on the fact, if the day exist in the standard calendar.

\noindent \begin{tabular}{l l l l} 
    \toprule		
    \textbf{id} & \textbf{Input} & \textbf{Output}\\ 
	\midrule
   id2.1 & (0, 0, 0) & not exist\\
   id2.2 & (-1, -1, -1)  & not exist\\
   id2.3 & (2020, -1, 29)  & not exist\\
   id2.4 & (2020, 02, -1)  & not exist\\
   id2.5 & (-1, 02, 29)  & not exist\\
   id2.6 & (2020, 13, -1)  & not exist\\
   id2.7 & (2020, 02, 29)  & not exist\\
   id2.8 & (2020, 02, 28)  & exist\\
    \bottomrule
  \end{tabular}
  %	\caption{Provide a caption}
%\end{table}
 
\wss{The expected result for the given inputs}

Test Case Derivation: The expected result for the given input is based on the standard calendar. If the given input is exist in the calendar, display ``exist", otherwise display ``not exist". \\

 \wss{Justify the expected value given in the Output field} 

How the test will be performed:
\begin{itemize} 
\item Input the designed test case input values by using testing framework, Tasty. 
\item Write the output in the file, resultInputBounds-id2.txt.
\item Verified the output showing in the file by match the result with the expected output above.
\item If all the result match with the expected output, then the test case success.
\item Otherwise the test case fails.
\end{itemize}

\item{InputBounds-id3\\}

This test case test the capability of capture the invalid input days, which means the end day is smaller than start day.

Control: Automatic.
 
Initial State: No any given value.\\
Input: ($\mathit{year}_\text{Start}$, $\mathit{month}_\text{Start}$, $\mathit{day}_\text{Start}$) - ($\mathit{year}_\text{End}$, $\mathit{month}_\text{End}$, $\mathit{day}_\text{End}$)\\ 


Output: The expected result will for the given inputs is based on the standard calendar.
If the end day $<$ start day then output ``invalid", otherwise output ``valid".

\noindent \begin{tabular}{l l l l} 
    \toprule		
    \textbf{id} & \textbf{Input} & \textbf{Output}\\ 
	\midrule
   id3.1 & (2020, 02, 28) - (2021, 02, 28) & valid\\
   id3.2 & (2020, 02, 28) - (2019, 02, 28)  & invalid\\
   id3.3 & (2020, 02, 28) - (2020, 01, 28)  & invalid\\
   id3.4 & (2020, 02, 28) - (2020, 02, 27)  & invalid\\
   id3.5 & (2020, 02, 28) - (2020, 02, 28)  & valid\\
    \bottomrule
  \end{tabular}
  %	\caption{Provide a caption}


Test Case Derivation: This test is driven by the standard calendar system.\\ By the fact days can be sorted, if the end day is the day after the start day in the  calendar, output ``valid". Otherwise output ``invalid".


How the test will be performed: 

\begin{itemize} 
\item Input the designed test case input values by using testing framework, Tasty. 
\item Write the output in the file, resultInputBounds-id3.txt.
\item Verified the output showing in the file by match the result with the expected output above.
\item If all the result match with the expected output, then the test case success.
\item Otherwise the test case fails.
\end{itemize}

\end{enumerate}


\subsubsection{Calculation Module} 

\begin{enumerate}
\item{calculation-id4\\} 
This test case tests the capability of calculate the zenith angle.

Control: Automatic. 

Initial State: Giving a fixed sun declination angle. $\delta_{date}$ =  20\\
Input: $\Phi_P$ 


Output: The expected result will for the given inputs is based on the input value.
The output is calculated by the method descibed in MIS. 

\noindent \begin{tabular}{l l l l} 
    \toprule		
    \textbf{id} & \textbf{Input} & \textbf{Output}\\ 
	\midrule
   id4.1 &  43.250943  & 23.250943\\
   id4.2 & -43.250943  & -23.250943\\
   id4.3 & -89.250943  & -69.250943\\
   id4.4 &  89.250943  & 69.250943\\
    \bottomrule
  \end{tabular}
  %	\caption{Provide a caption}
%\end{table}

Test Case Derivation: The output of this test case should follow the constraint of the latitude. The values should between [-90..90]. The equation of calculating the output is:\\
($\delta_{date}$ * $\Phi_P$  $<$ 0 $\Rightarrow$ $\delta_{date}$ + $\Phi_P$ $|$ True $\Rightarrow$ $\delta_{date}$ - $\Phi_P$). Calculate an \textbf{absolute error}, such that $|$ expected output - actaul output $|$.


How the test will be performed:

\begin{itemize} 
\item Input the designed test case input values by using testing framework, Tasty. 
\item Verified the output showing in the file by calculating the \textbf{absolute error}.
\item Write the absolute error in the file, resultCalculation-id4.txt.
\item If all the absolute error is equal to 0, then the test case success.
\item Otherwise the test case fails.
\end{itemize}

\item{calculation-id5\\} 
This test case tests the capability of calculate the sun Intensity.

Control: Automatic. 

Initial State: Giving a fixed solar intensity $I_{S}$  =  1.35 \\
Input: $\theta_{S_{\text{date}}}$ 


Output: The expected result will for the given inputs is based on the input value.
The output is calculated by the method descibed in MIS. 

\noindent \begin{tabular}{l l l l} 
    \toprule		
    \textbf{id} & \textbf{Input} & \textbf{Output}\\ 
	\midrule
   id5.1 &  23.250943  & 1.0883942\\
   id5.2 & -23.250943  & 1.0883942\\
   id5.3 & -69.250943  & 2.8226608\\
   id5.4 &  69.250943  & 2.8226608\\
    \bottomrule
  \end{tabular}
  %	\caption{Provide a caption}
%\end{table}

Test Case Derivation: The equation of calculating the output is:\\
$I_{S} \cdot (\frac{1.00}{\text{energy}})^{sec(\theta_{S_{\text{date}}})} $. Calculate an $\textbf{relative error}$ such that $| 1 - \frac{\text{actual result}}{ \text{expected result}} |$.


How the test will be performed:

\begin{itemize} 
\item Input the designed test case input values by using testing framework, Tasty. 
\item Verified the output showing in the file by calculating the \textbf{relative error}.
\item Write the relative error in the file, resultCalculation-id5.txt.
\item If all the relative error $<$ error Tolerance, then the test case success.
\item Otherwise the test case fails.
\end{itemize}

\subsubsection{Control Module} 

This test case test the coverage of the Control Module.

\item{control-id6\\}

Type: Dynamic analysis - Code Coverage 


Input/Condition: Call the main function in code of Control Module

Output: The number of how percentage that the software has been coverage by the implementation.

Test Case Derivation: Use the code coverage tool, Haskell Program Coverage(HPC), to test the code coverage. The Control Module is controlling the process of geting the final output. Therefore, by testing the coverage of it, the output number can represent the coverage of the entire program.

How the test will be performed:

\begin{itemize} 
\item Use the compiler, ghc-6.8.1 or later version, to active Haskell Program
Coverage.
\item Enable hpc with the command line, -fhpc.
\item Implement the code of the \progname
\item Display the output
\item Write the result in the file, resultControl-id6.txt.
\item The test case success, if the output gets the 100\% of the coverage,
\item Otherwise the test case fails. 
\end{itemize}

\end{enumerate} 




\subsection{Tests for Nonfunctional Requirements}

This software do not have any nonfunctional requirements for unit testing.

\wss{If there is a module that needs to be independently assessed for
  performance, those test cases can go here.  In some projects, planning for
  nonfunctional tests of units will not be that relevant.}

\wss{These tests may involve collecting performance data from previously
  mentioned functional tests.}




\subsection{Traceability Between Test Cases and Modules}


\begin{table}[h!]
\centering
\begin{tabular}{|c|c|c|c|c|c|c|c|c|c|}
\hline        
	& R1& R2 & R3 &R4 & R5 &R6  &R7 &R8 &R9 \\
\hline
id1     &   & X &     &    &    &    &   &    &     
\\ \hline
id2     &   & X &     &    &    &    &   &    &     \\ 
\hline
id3     &   & X &     &    &    &    &   &    &   
\\ \hline
id4     &   &    &     & X &    &    &   &    &      \\ \hline
id5     &   &    &     & X & X & X & X &    &     \\  \hline
id6     &   &     & X & X &    &    &   & X & X  \\ 
\hline
\end{tabular}
\caption{Traceability Between Functional Requirements Test Cases and
Requirements}
\label{Table:Rtrace}
\end{table}
\wss{Provide evidence that all of the modules have been considered.}

\bibliographystyle{plain}
\bibliography{../../../refs/References} 
%\bibliography{SRS}

\newpage

\section{Appendix}

\wss{This is where you can place additional information, as appropriate}

\subsection{Symbolic Parameters}
\wss{The definition of the test cases may call for SYMBOLIC\_CONSTANTS.
Their values are defined in this section for easy maintenance.}

  \noindent \begin{tabular}{l l l} 
    \toprule		
    \textbf{Symbol} & \textbf{Description} & \textbf{Value}\\
    \midrule 
    $I_{S}$ & Solar insensity &1.35  \\
error Tolerance & Error Tolerance & 1\\
    \bottomrule
  \end{tabular}
  %	\caption{Provide a caption}
%\end{table}




\end{document}