\documentclass[12pt, titlepage]{article}

\usepackage{amsmath, mathtools}

\usepackage[round]{natbib}
\usepackage{amsfonts}
\usepackage{amssymb}
\usepackage{graphicx}
\usepackage{colortbl}
\usepackage{xr}
\usepackage{hyperref}
\usepackage{longtable}
\usepackage{xfrac}
\usepackage{tabularx}
\usepackage{float}
\usepackage{siunitx}
\usepackage{booktabs}
\usepackage{multirow}
\usepackage[section]{placeins}
\usepackage{caption}
\usepackage{fullpage}

\hypersetup{
bookmarks=true,     % show bookmarks bar?
colorlinks=true,       % false: boxed links; true: colored links
linkcolor=red, % color of internal links (change box color with linkbordercolor)
citecolor=blue,      % color of links to bibliography
filecolor=magenta,  % color of file links
urlcolor=cyan          % color of external links
}

\usepackage{array}

\externaldocument{../../SRS/SRS}

%% Comments

\usepackage{color}

\newif\ifcomments\commentstrue

\ifcomments
\newcommand{\authornote}[3]{\textcolor{#1}{[#3 ---#2]}}
\newcommand{\todo}[1]{\textcolor{red}{[TODO: #1]}}
\else
\newcommand{\authornote}[3]{}
\newcommand{\todo}[1]{}
\fi

\newcommand{\wss}[1]{\authornote{blue}{SS}{#1}} 
\newcommand{\plt}[1]{\authornote{magenta}{TPLT}{#1}} %For explanation of the template
\newcommand{\an}[1]{\authornote{cyan}{Author}{#1}}


\newcommand{\progname}{Sun Catcher}

\begin{document}

\title{Module Interface Specification for \progname}

\author{Sharon (Yu-Shiuan) Wu}

\date{\today}

\maketitle

\pagenumbering{roman}

\section{Revision History}

\begin{tabularx}{\textwidth}{p{3cm}p{2cm}X}
\toprule {\bf Date} & {\bf Version} & {\bf Notes}\\
\midrule
2019/11/25 & 1.0 & First Version\\
Date 2 & 1.1 & Notes\\
\bottomrule
\end{tabularx}

~\newpage

\section{Symbols, Abbreviations and Acronyms}

See SRS Documentation at \url{https://github.com/sharyuwu/optimum-tilt-of-solar-panels/blob/master/docs/SRS/SRS.pdf}

\wss{Also add any additional symbols, abbreviations or acronyms}

\newpage

\tableofcontents

\newpage

\pagenumbering{arabic}

\section{Introduction}

The following document details the Module Interface Specifications for \progname
\wss{Fill in your project name and description}

Complementary documents include the System Requirement Specifications
and Module Guide.  The full documentation and implementation can be
found at \url{https://github.com/sharyuwu/optimum-tilt-of-solar-panels}.  \wss{provide the url for your repo}

\section{Notation}

\wss{You should describe your notation.  You can use what is below as
  a starting point.}

The structure of the MIS for modules comes from \citet{HoffmanAndStrooper1995},
with the addition that template modules have been adapted from
\cite{GhezziEtAl2003}.  The mathematical notation comes from Chapter 3 of
\citet{HoffmanAndStrooper1995}.  For instance, the symbol := is used for a
multiple assignment statement and conditional rules follow the form $(c_1
\Rightarrow r_1 | c_2 \Rightarrow r_2 | ... | c_n \Rightarrow r_n )$.

The following table summarizes the primitive data types used by \progname. 

\begin{center}
\renewcommand{\arraystretch}{1.2}
\noindent 
\begin{tabular}{l l p{7.5cm}} 
\toprule 
\textbf{Data Type} & \textbf{Notation} & \textbf{Description}\\ 
\midrule
character & char & a single symbol or digit\\
integer & $\mathbb{Z}$ & a number without a fractional component in  (-$\infty$,
$\infty$) \\
natural number & $\mathbb{N}$ & a number without a fractional component in [1,
$\infty$) \\
real & $\mathbb{R}$ & any number in  (-$\infty$, $\infty$)\\
bool &$\mathbb{B}$ & a statement of True or False\\
sequence & [ ] & a sequence of the same type\\


\bottomrule
\end{tabular} 
\end{center}

\noindent
The specification of \progname \ uses some derived data types: sequences,
strings, and
tuples. Sequences are lists filled with elements of the same data type. Strings
are sequences of characters. Tuples contain a list of values, potentially of
different types. In addition, \progname \ uses functions, which
are defined by the data types of their inputs and outputs. Local functions are
described by giving their type signature followed by their specification.



\section{Module Decomposition}

The following table is taken directly from the Module Guide document for this
project.

\begin{table}[h!]
\centering
\begin{tabular}{p{0.3\textwidth} p{0.6\textwidth}}
\toprule
\textbf{Level 1} & \textbf{Level 2}\\
\midrule

{Hardware-Hiding Modules} & ~ \\
\midrule

\multirow{7}{0.3\textwidth}{Behaviour-Hiding
 Module}& Control Module\\
& Input/Read Parameters Module\\
& Input Verification Module\\
& Solar Energy Absorption Module\\
& Optimum Tilt Angle Module\\
& Calculation Module\\
\midrule

\multirow{3}{0.3\textwidth}{Software Decision Module} 
& Day ADT Module\\
& Table-layout Module\\
& Sun Catcher Type 
Module\\
\bottomrule

\end{tabular}
\caption{Module Hierarchy}
\label{TblMH}
\end{table}

\newpage
~\newpage


\section{MIS of Control Module} \label{ModuleC} 

\subsection{Module}
Control

\subsection{Uses}
SunCatTy\\
DayT\\
InputPara (Section~\ref{ModuleIP})\\ \wss{Please leave space between the name and the
  reference.  You should also identify what type of reference it is, like you
  would for a Figure or a Table.  That is, you should say (Section~\ref{ModuleIP}).}\\
InputVer (Section ~\ref{ModuleIV})\\
OutputPara (Section ~\ref{ModuleOP})\\
TiltAng (Section ~\ref{ModuleTA})\\
Energy (Section ~\ref{ModuleE})\\


\subsection{Syntax}

\subsubsection{Exported Constants}
filename = "analemma.txt"

\subsubsection{Exported Access Programs}

\begin{center}
\begin{tabular}{p{2cm} p{4cm} p{4cm} p{2cm}}
\hline
\textbf{Name} & \textbf{In} & \textbf{Out} & \textbf{Exceptions} \\
\hline
input  & - & - & - \\
getOneAng \wss{main?}\an{changed name} & - & - & - \\
getTwoAng & - & - & - \\
output & - & - & - \\


\hline
\end{tabular}
\end{center}


\subsection{Semantics}

\subsubsection{State Variables}
\wss{You use this type, but I don't think you import (use) the module
  that defines it} \an{Change DayT as Day ADT} \an{State Variable dayS and dayE only exit in Input Parameters Module}\\

\wss{Why are these two state variables here?  I thought you were going to use
  the Input Parameters module?}
\wss{Now that I think about it, given that you are planning on implementing with
  Haskell, you might want to avoid having modules with state information.} \an{Yes, Agree. But I will still keep some of my state variable inside the document for the purpose the easy understanding. Then implement in a slightly different way.}

\subsubsection{Environment Variables}


\subsubsection{Assumptions}


\subsubsection{ Access Routine Semantics}

\noindent  input ( ):

\begin{itemize}
\item transition: Implement the InputPara and  the environment variables for Output by following steps.

Get ($\Phi_P$: DegreeT, $P_{A_{\text{h}}}$: real, $P_{A_{\text{w}}}$: real, $\mathit{year}_\text{Start}$: natural number, $\mathit{month}_\text{Start}$: natural number, $\mathit{day}_\text{Start}$: natural number, $\mathit{year}_\text{End}$: natural number, $\mathit{month}_\text{End}$: natural number, $\mathit{day}_\text{End}$: natural number) from users' input.\\


load\_anale\_data (filename)\\
InputPara.init ( )\\

$\#$ Verify the input values using Input Verfication Module\\
verifiedLat ( InputPara.getla ( ) )\\
verifiedP ( InputPara.getph ( ), InputPara.getpw ( ) )\\
verifiedD (InputPara.getdayS, InputPara.getdayE)\\


\end{itemize}

\noindent  getOneAng ( ):

\begin{itemize}
\item transition:Implied the one optimum cut by following steps


$\#$ Get the zenith angle between dayS and
dayE\\
$\theta_{S_{\text{date}}}$ = Calculation.getzenList ( InputPara.getdec,   localDaySandPer ( ),  localDaySandDayE ( ) , InputPara.getla)\\

$\#$ Get the optimum tilt angle using $\theta_{S_{\text{date}}}$\\
$\theta_{T}$ = getilt ( $\theta_{S_{\text{date}}}$, localsunIn($\theta_{S_{\text{date}}}$) )\\


$\#$ Get the estimated solar absorption using Optimum Tilt Angle Module\\

$P_{E}$ = getenergy (InputPara.getpw, InputPara.getph, localtiltSunIn ( localsunIn ($\theta_{S_{\text{date}}}$, $\theta_{T}$), $\theta_{S_{\text{date}}}$ )\\

$\#$ Output $P_{E}$ and $\theta_{T}$ to the Table-layout Module\\
addresult ($\theta_{T}$, $P_{E}$, \{InputPara.getdayS\} )\\



\end{itemize}

\noindent  getTwoAng ( ):

\begin{itemize}
\item transition:Implied the two optimum cut by following steps\\

$\theta_{S_{\text{date}}}$ = Calculation.getzenList ( InputPara.getdec,   localDaySandPer ( ),  localDaySandDayE ( ) , InputPara.getlala)\\

$\#$ Get the zenith angle between dayS and
$\frac{dayE + dayS}{2}$ and $\frac{dayE + dayS}{2}$ to dayE\\
zenithSet =  \{$\theta_{S_{\text{date}}}[1.. | \frac{\theta_{S_{\text{date}}}}{2}|]$\} $\cup$ \{$\theta_{S_{\text{date}}}[| \frac{\theta_{S_{\text{date}}}}{2}|.. | \theta_{S_{\text{date}}}|]$\}\\
 
$\#$ Get the optimum tilt angle using $\theta_{S_{\text{date}}}$\\
$\theta_{T}$ = $\theta_{T}$ $\cup$ ( zenList : DegreeT[ ] $|$ zenList $\in$ zenithSet $\bullet$ getilt ( zenList, localsunIn(zenList) ) )\\

$\#$ Get the estimated solar absorption using Optimum Tilt Angle Module\\

$P_{E}$ = $P_{E}$ $\cup$  ( zenList : DegreeT[ ] $|$ zenList $\in$ zenithSet $\bullet$ getenergy (InputPara.getpw, InputPara.getph, localtiltSunIn ( localsunIn (zenList, $\theta_{T}$), zenList )))\\

$\#$ Output $P_{E}$ and $\theta_{T}$ to the Table-layout Module\\
addresult ($\theta_{T}$, $P_{E}$, \{InputPara.getdayS, InputPara.getdayS.addDay($| \frac{\theta_{S_{\text{date}}}}{2}|]$) \})\\

\end{itemize}

\noindent  output ( ):
\begin{itemize}
\item output: out :=\\
$\#$ Write the output values in the file using the Table-layout Module\\
display ( )
\end{itemize}
\subsubsection{Local Functions}

$\#$ The days detween start day and perhelion\\
localDaySandPer: Integer\\
localDaySandPer = InputPara.getdayS. perhelion\\

$\#$ The days detween start day and end day\\
localDaySandDayE: Integer\\
localDaySandDayE =InputPara.getdayE.countDiff (InputPara.getdayS)\\

$\#$ Calculate the total sun intensity for the base case\\
localsunIn: a sequence of DegreeT $\rightarrow$ real\\
localsunIn (zen[ ]) = Calculation.sumSunIn( 1.35, zen)

$\#$ Get the optimum sun intensity\\
localtiltSunIn: DegreeT $\times$ real $\rightarrow$ real\\
localtiltSunIn (degree, intensity ) = TiltAngle.getiltInten (intensity, degree )



\wss{It is nice to have a newpage between modules.}\an{OK}

\newpage

\section{Sun Catcher Type Module} \label{ModuleSCTM} 

\subsection{Module}
SunCatTy

\subsection{Uses}

N/A\\

\subsection{Exported Types}

DegreeT = $\mathbb{R}$\\

\subsection{Syntax}

\subsubsection{Exported Constants}
N/A\\

\subsubsection{Exported Access Programs}
N/A\\

\subsection{Semantics}
N/A\\
\subsubsection{State Variables}
N/A\\

\subsubsection{Environment Variables}

N/A\\

\subsubsection{Assumptions}
N/A\\

\newpage


\section{Day ADT Module} \label{ModuleADTD} 

\subsection{Template Module}
DayT

\subsection{Uses}

SunCatTy\ref{ModuleSCTM} \\

\subsection{Exported Types}

DayT = ? \wss{It seems confusing to me that you have a DayT, that is not an
  ADT, and a DayDurT that is an ADT.  Couldn't you just have an ADT for DayT?}\\

\subsection{Syntax}

\subsubsection{Exported Constants}


\subsubsection{Exported Access Programs}

\begin{center}
\begin{tabular}{p{3cm} p{5cm} p{4cm} p{2cm}}
\hline
\textbf{Name} & \textbf{In} & \textbf{Out} & \textbf{Exceptions} \\
\hline
new DayT& $\mathbb{N}$, $\mathbb{N}$, $\mathbb{N}$ & DayT & invalid\_argument \\
addDay &  $\mathbb{N}$  & DayT & - \\
perihelion &  -  & Integer & - \\
countDiff &  DayT  & Integer & - \\
$\leq$ &  DayT  & $\mathbb{B}$ & - \\
$>$ &  DayT  & $\mathbb{B}$ & - \\
\hline
\end{tabular}
\end{center}

\wss{The input to the DayDurT could just be DayT, DayT; you don't really need to
  make the input a tuple.} \an{Agree}

\subsection{Semantics}
DayT: natural number

\subsubsection{State Variables}

stateDay : DayT \wss{You have these same state variables in another module}\\

\subsubsection{Environment Variables}

N/A\\

\subsubsection{Assumptions}

\wss{It would be neat if you defined a comparison access program to your ADT so
  that you have greater than implemented.  As it is, saying greater than is ambiguous.}
\an{Change the expession of getdiff to countDiff. It uses compare function to check if day1 is greater than day2}

\subsubsection{ Access Routine Semantics}

\noindent  new DayT (y, m, d):
\begin{itemize}
\item transition:  stateDay = (m = 1 $\lor$ m = 2 $\Rightarrow$ MonthOneTwo (y, m, d) $|$ True $\Rightarrow$ MonthNotOneTwo (y, m, d)\\

\wss{You don't actually say otherwise using our notation.  You could just say
  True.  You also don't have a dummy variable.  I don't really know how to read
  this expression.}
\item output: out := self
\item exception: 
\end{itemize}

\noindent  addDay (times):
\begin{itemize}
\item transition: 
\item output: out := (times $\ne$ 0 $\Rightarrow$ localnextday (stateDay).addDay(times -1) $|$ True $\Rightarrow$ stateDay)

\item exception:
\end{itemize}

\noindent  perihelion ( ):
\begin{itemize}
\item transition: 
\item output: out := (stateDay.m = 12 $	\wedge$ stateDay.d $\geq	$ 21 $\Rightarrow$  countDiff (new DayT (stateDay.y, 12, 21)) $|$ True $\Rightarrow$ countDiff (new DayT ( stateDay.y - 1, 12, 21)))
\item exception:
\end{itemize}

\noindent  countDiff ( day ):
\begin{itemize}
\item transition: 
\item output: out := (day $\leq$ stateDay $\Rightarrow$ 1 + countDiff (localnextday (day)) $|$ day $>$  stateDay $\Rightarrow$ 0)
\item exception:
\end{itemize}


\noindent  $\leq$ ( day ):
\begin{itemize}
\item transition: 
\item output: out := (day is placed before than stateDay Gregorian Calender $\lor	$ inday and day is the same day $\Rightarrow$ True $|$ False)
\item exception:
\end{itemize}

\noindent $>$  ( day ):
\begin{itemize}
\item transition: 
\item output: out := (day is placed before than stateDay Gregorian Calender $\lor	$ day and stateDay is the same day $\Rightarrow$ True $|$ False)
\item exception:
\end{itemize}
\wss{You shouldn't have a transition and an output.  I don't actually see why
  you need the daduraL state variable.  I would think you could calculate the
  duration as needed and simply output it?}

\subsubsection{Local Functions}

calculateB: $\mathbb{N}$ $\rightarrow $ $\mathbb{N}$\\
calculateB (a) = 2 - a + (a / 4)\\

MonthOneTwo: $\mathbb{N} \times \mathbb{N} \times \mathbb{N} \rightarrow \mathbb{N}$\\
MonthOneorTwo (y, m, d) = 365.25 $\times$ (y - 1) + 30.6001 $\times$ (m + 13) + d - calculateB(y /100) +  1720995\\


MonthNotOneTwo: $\mathbb{N} \times \mathbb{N} \times \mathbb{N} \rightarrow \mathbb{N}$\\
MonthNotOneorTwo (y, m, d) = 365.25 $\times$ y + 30.6001 $\times$ (m + 1) + d - calculateB(y / 100) +  1720995)

localnextday : DayT $\rightarrow$ DayT
localnextday ( inputDay ) = The next day of the  inputDay according to the Gregorian Calender.

\newpage


\section{MIS of Input Parameters Module} \label{ModuleIP} 

\subsection{Module}
InputPara

\subsection{Uses}
HardH\\
Day ADT Module (Section \ref{ModuleADTD})

\subsection{Syntax}

\subsubsection{Exported Constants}


\subsubsection{Exported Access Programs}

\begin{center}
\begin{tabular}{p{4cm} p{2cm} p{4cm} p{2cm}}
\hline
\textbf{Name} & \textbf{In} & \textbf{Out} & \textbf{Exceptions} \\
\hline 
fromKeyBoard & - & - & Key\_error \\
loadAnaleFile & string & - & File\_error \\
getla & - & DegreeT & - \\
getph & - & real & - \\
getpw & - & real & - \\
getdayS & - & DayT & - \\
getdayE & - & DayT & - \\
getdec & - & a sequence of real & - \\


\hline
\end{tabular}
\end{center}


\subsection{Semantics}

\subsubsection{State Variables}
latitude: DegreeT\\
dayS: DayT \wss{You have these state variables already in other modules.}\an{Delete dayS and dayE in other module}\\
dayE: DayT\\
panH: real\\
panW: real\\
declination: a sequence [366] of real


\subsubsection{Environment Variables}
key: Input variables from keyboard.

\subsubsection{Assumptions}
When users chick the submit bottom from system interface, InputPara.init ( ) implement.

\subsubsection{ Access Routine Semantics}

\noindent  fromKeyBoard ( ):
\begin{itemize}
\item transition: Get the values from users' input.\\
latitude, panH, panW = HardH. $\Phi_P$, HardH. $P_{A_{\text{h}}}$, HardH. $P_{A_{\text{w}}}$\\

dayS := new DayT (HardH. $\mathit{year}_\text{Start}$, HardH. $\mathit{month}_\text{Start}$, HardH. $\mathit{day}_\text{Start}$) \\

dayE :=  new DayT ( HardH. $\mathit{year}_\text{End}$, HardH. $\mathit{month}_\text{End}$, HardH. $\mathit{day}_\text{End}$)\\
\item output:
\item exception: If the data type of captured values do not match with the parameters' data type $\Rightarrow$ Key\_error
\end{itemize}



\noindent  loadAnaleFile ( fileName ):
\begin{itemize}
\item transition: declination [0..366] := read data from the file analemma.txt.\\
The text file has the following format, where declination\_i denotes the angle of sun declination for day 1 to day 366.
All data values are separate into rows, where each row has a value. There is 366 rows in the file.
\begin{center}
declination\_0\\
declination\_1\\
declination\_2\\
.\\
.\\
.\\
declination\_366\\
\end{center}

\item output:
\item exception: If the file is not found $\Rightarrow$ File\_error
\end{itemize}

\noindent  getla ( ):
\begin{itemize}
\item transition: 
\item output: latitude
\item exception: 
\end{itemize}

\noindent  getph ( ):
\begin{itemize}
\item transition: 
\item output: panH
\item exception: 
\end{itemize}

\noindent  getpw ( ):
\begin{itemize}
\item transition: 
\item output: panW
\item exception: 
\end{itemize}

\noindent  getdayS ( ):
\begin{itemize}
\item transition: 
\item output: dayS
\item exception: 
\end{itemize}

\noindent  getdayE ( ):
\begin{itemize}
\item transition: 
\item output: dayE
\item exception: 
\end{itemize}

\noindent  getdec ( ):
\begin{itemize}
\item transition: 
\item output: declination
\item exception: 
\end{itemize}

\subsubsection{Local Functions}



\newpage



\section{MIS of Input Verfication Module} \label{ModuleIV} 

\subsection{Module}
InputVer

\subsection{Uses}
InputPara\ref{ModuleIP}

\subsection{Syntax}

\subsubsection{Exported Constants}


\subsubsection{Exported Access Programs}

\begin{center}
\begin{tabular}{p{2cm} p{5cm} p{2cm} p{5cm}}
\hline
\textbf{Name} & \textbf{In} & \textbf{Out} & \textbf{Exceptions} \\
\hline 
verifiedLat & DegreeT& $\mathbb{B}$ & - \\
verifiedP &  real, real & $\mathbb{B}$ & - \\
verifiedD & DayT, DayT, & $\mathbb{B}$ & - \\
\hline
\end{tabular}
\end{center}

\wss{Rather than simply rely on exceptions, you could have verify return a
  Boolean.}

\subsection{Semantics}

\subsubsection{State Variables}



\subsubsection{Environment Variables}
\# Display warning message on the screen\\
screen: Hardware.screen\\

\subsubsection{Assumptions}
The input values is input from InputPara


\subsubsection{ Access Routine Semantics}

\noindent  verifiedLat (latitude):
\begin{itemize}
\item transition: 
\item output: (latitude $>$ 90 $\lor$ latitude $<$ - 90 $\Rightarrow$ screen.display |`` The degree of the latitude can't be greater than 90 or smaller than -90. " $|$ True)
\item exception: exc := 
\end{itemize}

\noindent  verifiedP (ph, pw):
\begin{itemize}
\item transition: 
\item output: ( ph $<$ 0 $\lor$ pw $<$ 0 $\Rightarrow$ screen.display |`` The height and the weight of the panel can't be negative. " $|$ True)
\item exception: exc := 
\end{itemize}

\noindent  verifiedD (ds, de):
\begin{itemize}
\item transition: 
\item output: ( ds $<$ de $\Rightarrow$ screen.display |`` The end day can't be smaller than the start day. " $|$ True)
\item exception: exc := 
\end{itemize}

\subsubsection{Local Functions}



\newpage



\section{MIS of Table-layout Module} \label{ModuleT} 

\subsection{Module}
Table\\

\subsection{Uses}
OutputPara\ref{ModuleOP}\\
HardH\\


\subsection{Syntax}

\subsubsection{Exported Constants}

mainRow1:  ``Adjust"\\
mainRow2: ``Energy Absorption/day"\\
onceColumn: ``Optimum Angle" \\


\subsubsection{Exported Access Programs}

\begin{center}
\begin{tabular}{p{2cm} p{5cm} p{5cm} p{2cm}}
\hline
\textbf{Name} & \textbf{In} & \textbf{Out} & \textbf{Exceptions} \\
\hline 
addresult &  a set of DegreeT, a set of real, a set of DayT & - & - \\
display & - & - & - \\

\hline
\end{tabular}
\end{center}


\subsection{Semantics}

\subsubsection{State Variables}

mainTable: a set of \{ tuples of ( 
cutTime: natural number, energy: real) \}\\
tiltTable: a set of \{ tuples of (time: a set of DayT, angle: a set of DegreeT) \}\\\\


\subsubsection{Environment Variables}
screen: Write tables  (output variables) in the file. \wss{You should define an
  environment variable for the screen.}

\subsubsection{Assumptions}


\subsubsection{ Access Routine Semantics}

\noindent  addresult ( angle, energy, day ):
\begin{itemize}
\item transition: mainTable = mainTable $\cup$ \{ ( $|$angle$|$ ,localAverage (energy) )  \}\\
tiltTable = tiltTable $\cup$ \{ (day, angle) \}
\wss{I
    don't know how to read this.}
\an{Change the expression}
\item output:
\item exception: 
\end{itemize}


\noindent  display ( ):
\begin{itemize}
\item transition:
\item output: out := display a table that shows the result in the file.

The main table has the following format, where the first row following heading: mainRow1, denotes the time of adjust the solar panel; the second row following heading, mainRow2, denotes the result of estimating energy absorption per day\\

mainSet: a set of tuples $|$ mainSet $\in$ mainTable $\wedge$ main: ( 
cutTime: natural number, energy: real) $|$ main $\in$ mainSet\\


\begin{center}
\begin{tabular}{|p{4.5cm}|p{4cm}|p{1cm}|}

\hline 
Adjust & main.cutTime   \\
\hline 
Energy Absorption/day & main.energy    \\
\hline 
\end{tabular}
\end{center}

The angle table has following format,  where the  first column heading: onceColumn, denotes the angle for adjusting the solar panel; the row heading denotes the time for adjusting the angle.\\


tiltSet: a set of tuples $|$ tiltSet $\in$ tiltTable $\wedge$ tilt: (time: a set of DayT,  angle: a set of DegreeT)  $|$ tilt $\in$ tiltSet \ \\

\begin{center}
\begin{tabular}{|p{4.5cm}|p{4cm}|p{1cm}|}

\hline 
  & Optimum Angle   \\
\hline 
tilt.time & tilt.angle \\
\hline 
\end{tabular}
\end{center}


\item exception: 
\end{itemize}


\subsubsection{Local Functions}
localAverage : a set of real $\rightarrow$ real\\
localAverage (energy) =  + ( i: real $|$ i $\in$ energy $\bullet$  i) / $|$energy$|$\\



\newpage




\section{MIS of Optimum Tilt Angle Module} \label{ModuleTA} \wss{Use labels for
  cross-referencing}

\subsection{Module}
TitleAng
\wss{Short name for the module}

\subsection{Uses}
SunInten\ref{ModuleSI}\\
ZenithAng\ref{ModuleZA}\\

\subsection{Syntax}


\subsubsection{Exported Constants}

$I_{S}$ := 1.35\\


\subsubsection{Exported Access Programs}

\begin{center}
\begin{tabular}{p{2cm} p{4cm} p{4cm} p{2cm}}
\hline
\textbf{Name} & \textbf{In} & \textbf{Out} & \textbf{Exceptions} \\
\hline 

getilt &  DegreeT[ ], real  & DegreeT & - \\
getiltInten &  DegreeT, real  & real & - \\


\wss{accessProg} & - & - & - \\
\hline
\end{tabular}
\end{center}


\subsection{Semantics}

\subsubsection{State Variables}

tiltDegree: DegreeT\\

\wss{Do you really need state variables?  I think you could use an input-output
  module here.  The same comment applies elsewhere.}

\wss{Not all modules will have state variables.  State variables give the module  a memory.}

\subsubsection{Environment Variables}

N/A
\wss{This section is not necessary for all modules.  Its purpose is to capture
  when the module has external interaction with the environment, such as for a
  device driver, screen interface, keyboard, file, etc.}

\subsubsection{Assumptions}


\wss{Try to minimize assumptions and anticipate programmer errors via
exceptions, but for practical purposes assumptions are sometimes appropriate.}

\subsubsection{Access Routine Semantics}


\noindent \wss{accessProg} getilt (zenithL[ ], sunIn ):
\begin{itemize}
\item transition: 
\item output: out := tiltDegree, such that  \\
$\forall$(zen: DegreeT $|$ j $\in$ zenithL
  $\bullet$ tiltDegree = localMax (tiltDegree, zen, sunIn) \wss{This is a complicated
    expression.  I think you aren't quite following our notation.  Also, for a
    complicated expression like this, it helps to add local functions.}
\item exception: 
\end{itemize}

\noindent getiltInten (degree, sunIn):
\begin{itemize}
\item transition:
\item output: Calculation.sglSunIn (degree, sunIn )
\item exception: 
\end{itemize}


\wss{A module without environment variables or state variables is unlikely to
  have a state transition.  In this case a state transition can only occur if
  the module is changing the state of another module.}

\wss{Modules rarely have both a transition and an output.  In most cases you
  will have one or the other.}

\subsubsection{Local Functions}

$\#$ Get two tuples, then return a tuple.\\
localMax: DegreeT $\times$ DegreeT $\times$ real $\rightarrow$ DegreeT\\
localMax(origDegree, newDegree, sunIn) = ( getiltInten (origDegree, sunIn) $\geq$ getiltInten (newDegree, sunIn) $\Rightarrow$ origDegree $|$ True $\Rightarrow$ newDegree )\\


\wss{As appropriate} \wss{These functions are for the purpose of specification.
  They are not necessarily something that is going to be implemented
  explicitly.  Even if they are implemented, they are not exported; they only
  have local scope.}




\newpage



\section{MIS of Solar Energy Absorption Module} \label{ModuleE} 

\subsection{Module}
Energy

\subsection{Uses}
InputPara\ref{ModuleIP}\\
ZenithAng\ref{ModuleZA}\\ 
TiltAng\ref{ModuleTA}

\subsection{Syntax}

\subsubsection{Exported Constants}


\subsubsection{Exported Access Programs}

\begin{center}
\begin{tabular}{p{2cm} p{5cm} p{4cm} p{1cm}}
\hline
\textbf{Name} & \textbf{In} & \textbf{Out} & \textbf{Exceptions} \\
\hline 
getenergy & real, real, real, DegreeT [ ] & a set of real & - \\


\hline
\end{tabular}
\end{center}


\subsection{Semantics}

\subsubsection{State Variables}

energyL: a sequence of real\\

\subsubsection{Environment Variables}

N/A

\subsubsection{Assumptions}



\subsubsection{ Access Routine Semantics}

\noindent  getenergy ( pw, ph, maxinten, zen[ ] ):
\begin{itemize}
\item transition:
\item output:  localEnergy ( localSunIn ( zen, maxinten ),pw, ph)
\item exception: 
\end{itemize}


\subsubsection{Local Functions}
localSunIn: a sequence of DegreeT $\times$ real $\rightarrow$ a set of real\\
localSunIn (zen[ ], maxinten) = $\cup$ (i: DegreeT $|$ i $\in$ z $\bullet$ \{ SunInten.single(maxinten, i) \})\\

localEnergy: a set of real $\times$ real $\times$ real $\rightarrow$ a set of real\\
localEnergy (sunIn, pw, ph) =  $\cup$ ( i: integer $|$ i $\in$ sunIn $\bullet$ \{ pw $\times$ ph $\times$ 18.7 $\times$ 0.75 $\times$ i \})



\newpage


\section{MIS of Calculation} \label{ModuleC} 

\subsection{Module}
Calculation\\


\subsection{Uses}


\subsection{Syntax}

\subsubsection{Exported Constants}
$I_{S}$ := 1.35\\

\subsubsection{Exported Access Programs}

\begin{center}
\begin{tabular}{p{2cm} p{4cm} p{4cm} p{2cm}}
\hline
\textbf{Name} & \textbf{In} & \textbf{Out} & \textbf{Exceptions} \\
\hline
getzenList & DegreeT[ ], integer, integer & - & - \\
sumSunIn & DegreeT[ ], real & real & - \\
sglSunIn & DegreeT, real & real & - \\

\hline
\end{tabular}
\end{center}

\subsection{Semantics}

\subsubsection{State Variables}


\subsubsection{Environment Variables}


\subsubsection{Assumptions}


\subsubsection{Access Routine Semantics}

\noindent getzenList(zenList[ ], i , diff, latitude):\\
$\#$ Calculate every elemant in list, zenList[i.. i+diff]. Then output the list
\begin{itemize}
\item transition: 
\item output: $\|$ (zen: DegreeT $|$ zen $\in$ zenList[i.. i+diff] $\bullet$ localZenAngle (zen, latitude)
\item exception: 
\end{itemize}

\noindent sumSunIn (  zenList[ ], energy ):
\begin{itemize}
\item output: out := + (zen: DegreeT $|$ zen $\in$ zenList $\bullet$ 
\begin{center}\large
$I_{S} \cdot (\frac{1.00}{\text{energy}})^{sec(\text{zen})} $)
\end{center}
\item exception: exc := $|$zenList$|$ = 0 $\Rightarrow$ sequence\_empty
\end{itemize}

\noindent sglSunIn (zen, energy):
\begin{itemize}
\item output: out := 
\begin{center}\large
$I_{S} \cdot (\frac{1.00}{\text{energy}})^{sec(zen)} $
\end{center}
\item exception: 
\end{itemize}


\subsubsection{Local Functions}
localZenAngle: DegreeT $\times$ real $\rightarrow$ Degree\\
localZenAngle (zen, latitude) = (zen * latitude $<$ 0 $\Rightarrow$ zen + latitude $|$ True $\Rightarrow$ zen - latitude)

\newpage

%\bibliographystyle {plainnat}
%\bibliography {../../../refs/References}

\newpage

\wss{Your MIS seems more complicated than it has to be.  The purpose of all the
  state variables is not clear.  I suggest you review all of the modules to see
  if they are necessary and whether you really need the state variables.  You
  also probably do not need so many modules.  You could combine the modules that
  export calculations into a ``Calculation'' module.  Also, you might want to
  explain what is going on in words, in case your math doesn't say what you
  think it is saying.}

\section{Appendix} \label{Appendix}

\wss{Extra information if required}

\end{document}