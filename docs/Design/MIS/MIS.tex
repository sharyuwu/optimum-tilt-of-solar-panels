\documentclass[12pt, titlepage]{article}

\usepackage{amsmath, mathtools}

\usepackage[round]{natbib}
\usepackage{amsfonts}
\usepackage{amssymb}
\usepackage{graphicx}
\usepackage{colortbl}
\usepackage{xr}
\usepackage{hyperref}
\usepackage{longtable}
\usepackage{xfrac}
\usepackage{tabularx}
\usepackage{float}
\usepackage{siunitx}
\usepackage{booktabs}
\usepackage{multirow}
\usepackage[section]{placeins}
\usepackage{caption}
\usepackage{fullpage}

\hypersetup{
bookmarks=true,     % show bookmarks bar?
colorlinks=true,       % false: boxed links; true: colored links
linkcolor=red, % color of internal links (change box color with linkbordercolor)
citecolor=blue,      % color of links to bibliography
filecolor=magenta,  % color of file links
urlcolor=cyan          % color of external links
}

\usepackage{array}

\externaldocument{../../SRS/SRS}

\input{../../Comments}

\newcommand{\progname}{Sun Catcher}

\begin{document}

\title{Module Interface Specification for \progname}

\author{Author Name}

\date{\today}

\maketitle

\pagenumbering{roman}

\section{Revision History}

\begin{tabularx}{\textwidth}{p{3cm}p{2cm}X}
\toprule {\bf Date} & {\bf Version} & {\bf Notes}\\
\midrule
Date 1 & 1.0 & Notes\\
Date 2 & 1.1 & Notes\\
\bottomrule
\end{tabularx}

~\newpage

\section{Symbols, Abbreviations and Acronyms}

See SRS Documentation at \wss{give url}

\wss{Also add any additional symbols, abbreviations or acronyms}

\newpage

\tableofcontents

\newpage

\pagenumbering{arabic}

\section{Introduction}

The following document details the Module Interface Specifications for
\wss{Fill in your project name and description}

Complementary documents include the System Requirement Specifications
and Module Guide.  The full documentation and implementation can be
found at \url{...}.  \wss{provide the url for your repo}

\section{Notation}

\wss{You should describe your notation.  You can use what is below as
  a starting point.}

The structure of the MIS for modules comes from \citet{HoffmanAndStrooper1995},
with the addition that template modules have been adapted from
\cite{GhezziEtAl2003}.  The mathematical notation comes from Chapter 3 of
\citet{HoffmanAndStrooper1995}.  For instance, the symbol := is used for a
multiple assignment statement and conditional rules follow the form $(c_1
\Rightarrow r_1 | c_2 \Rightarrow r_2 | ... | c_n \Rightarrow r_n )$.

The following table summarizes the primitive data types used by \progname. 

\begin{center}
\renewcommand{\arraystretch}{1.2}
\noindent 
\begin{tabular}{l l p{7.5cm}} 
\toprule 
\textbf{Data Type} & \textbf{Notation} & \textbf{Description}\\ 
\midrule
character & char & a single symbol or digit\\
integer & $\mathbb{Z}$ & a number without a fractional component in (-$\infty$,
$\infty$) \\
natural number & $\mathbb{N}$ & a number without a fractional component in [1,
$\infty$) \\
real & $\mathbb{R}$ & any number in (-$\infty$, $\infty$)\\
degree & $\mathbb{R}$ & any number in (-$\infty$, $\infty$)\\

\bottomrule
\end{tabular} 
\end{center}

\noindent
The specification of \progname \ uses some derived data types: sequences,
strings, and
tuples. Sequences are lists filled with elements of the same data type. Strings
are sequences of characters. Tuples contain a list of values, potentially of
different types. In addition, \progname \ uses functions, which
are defined by the data types of their inputs and outputs. Local functions are
described by giving their type signature followed by their specification.



\section{Module Decomposition}

The following table is taken directly from the Module Guide document for this
project.

\begin{table}[h!]
\centering
\begin{tabular}{p{0.3\textwidth} p{0.6\textwidth}}
\toprule
\textbf{Level 1} & \textbf{Level 2}\\
\midrule

{Hardware-Hiding Module} & ~ \\
\midrule

\multirow{7}{0.3\textwidth}{Behaviour-Hiding
 Module}& Control Module\\
& Input Parameters Module\\
& Input Verify Module\\
& Output Parameters Module\\
& Solar Energy Absorption Module\\
& Optimum Tilt Angle Module\\
& Sun Intensity Equation Module\\
& Zenith Angle Equation Module\\
& Sun Declination Module\\
& Days Module\\
\midrule

\multirow{3}{0.3\textwidth}{Software Decision Module} 
& Count days Module\\
& Table-layout Module\\
& Sequence Data Structure Module\\
\bottomrule

\end{tabular}
\caption{Module Hierarchy}
\label{TblMH}
\end{table}

\newpage
~\newpage

\section{Day ADT Module} \label{ModuleADTD} 

\subsection{Template Module}
Day

\subsection{Uses}

N/A\\

\subsection{Exported Types}

DayT = ?\\

\subsection{Syntax}

\subsubsection{Exported Constants}


\subsubsection{Exported Access Programs}

\begin{center}
\begin{tabular}{p{2cm} p{4cm} p{4cm} p{2cm}}
\hline
\textbf{Name} & \textbf{In} & \textbf{Out} & \textbf{Exceptions} \\
\hline
initday&  $\mathbb{N}$,$\mathbb{N}$,$\mathbb{N}$  & DayT & - \\
day &  -  & natural number & - \\
month &  -  & natural number & - \\
year &  -  & natural number & - \\

\hline
\end{tabular}
\end{center}


\subsection{Semantics}

\subsubsection{State Variables}
d1 :real \\
m1 :real \\
y1 :real \\

\subsubsection{Environment Variables}

N/A\\

\subsubsection{Assumptions}

\noindent  initday(d,m,y):
\begin{itemize}
\item transition: d1, m1, y1 := d,m,y
\item output: out := self
\end{itemize}

\noindent day( ):
\begin{itemize}
\item transition:
\item output: out := d1
\end{itemize}

\noindent  month( ):
\begin{itemize}
\item transition:
\item output: out := m1
\item exception: 
\end{itemize}

\noindent  year( ):
\begin{itemize}
\item transition:
\item output: out := y1
\item exception: 
\end{itemize}


\section{Analemma ADT Module} \label{ModuleADTD} 

\subsection{Template Module}
Analemma

\subsection{Uses}

N/A\\

\subsection{Exported Types}

AnalemmaT = ?\\

\subsection{Syntax}

\subsubsection{Exported Constants}


\subsubsection{Exported Access Programs}

\begin{center}
\begin{tabular}{p{2cm} p{4cm} p{4cm} p{2cm}}
\hline
\textbf{Name} & \textbf{In} & \textbf{Out} & \textbf{Exceptions} \\
\hline
initanale&  real,real,real  & AnalemmaT & - \\
x &  -  & real & - \\
y &  -  & real & - \\
z &  -  & real & - \\

\hline
\end{tabular}
\end{center}


\subsection{Semantics}

\subsubsection{State Variables}
x1 :real \\
y1 :real \\
z1 :real \\

\subsubsection{Environment Variables}

N/A\\

\subsubsection{Assumptions}

\noindent  initanale(x,y,z):
\begin{itemize}
\item transition: x1, y1, z1 := x,y,z
\item output: out := self
\end{itemize}

\noindent x( ):
\begin{itemize}
\item transition:
\item output: out := x1
\end{itemize}

\noindent  y( ):
\begin{itemize}
\item transition:
\item output: out := y1
\item exception: 
\end{itemize}

\noindent  z( ):
\begin{itemize}
\item transition:
\item output: out := z1
\item exception: 
\end{itemize}



\section{MIS of Optimum Tilt Angle Module} \label{ModuleOT} \wss{Use labels for
  cross-referencing}

\subsection{Module}
tile angle
\wss{Short name for the module}

\subsection{Uses}
Sun Intensity,zenith angle

\subsection{Syntax}


\subsubsection{Exported Constants}

$I_{S}$ := 1.35\\
$I_{S_{\text{total}}}$: real\\
$I_{S_{\text{total}}}$:= intenSum($I_{S}$)\\
$\theta_{S_{\text{date}}}$: a sequence of degree\\
$\theta_{S_{\text{date}}}$:= zenith( )\\

\subsubsection{Exported Access Programs}

\begin{center}
\begin{tabular}{p{2cm} p{4cm} p{4cm} p{2cm}}
\hline
\textbf{Name} & \textbf{In} & \textbf{Out} & \textbf{Exceptions} \\
\hline 
angle &  -  & degree & - \\



\wss{accessProg} & - & - & - \\
\hline
\end{tabular}
\end{center}


\subsection{Semantics}

\subsubsection{State Variables}


max.inten: real\\
max.inten :=  intenSingle($I_{S_{\text{total}}}$, $\theta_{S_{\text{date}}}$[0]),\\
max.deg : degree\\
max.deg = $\theta_{S_{\text{date}}}$[0]\\

\wss{Not all modules will have state variables.  State variables give the module  a memory.}

\subsubsection{Environment Variables}

N/A
\wss{This section is not necessary for all modules.  Its purpose is to capture
  when the module has external interaction with the environment, such as for a
  device driver, screen interface, keyboard, file, etc.}

\subsubsection{Assumptions}


\wss{Try to minimize assumptions and anticipate programmer errors via
exceptions, but for practical purposes assumptions are sometimes appropriate.}
\subsubsection{Access Routine Semantics}

\noindent \wss{accessProg} angle( ):
\begin{itemize}
\item transition:\\
($\forall$z: degree $|$ z $\in$ $\theta_{S_{\text{date}}}$ $	\bullet$ max.inten = ifMax (max.inten, intenSingle($I_{S_{\text{total}}}$,z))$\Rightarrow$ nothing change $|$  otherwise $\Rightarrow$ max.inten =  intenSingle($I_{S_{\text{total}}}$,z),
max.deg = z)

\item output: max.deg \\


\item exception: \wss{if appropriate} 
\end{itemize}

\wss{A module without environment variables or state variables is unlikely to
  have a state transition.  In this case a state transition can only occur if
  the module is changing the state of another module.}

\wss{Modules rarely have both a transition and an output.  In most cases you
  will have one or the other.}

\subsubsection{Local Functions}

ifMax: real $\times$ real $\rightarrow$ real\\
ifMax(x, y) = (x $\geq$ y $\Rightarrow$ x $|$ y $>$ x $\Rightarrow$ y )

\wss{As appropriate} \wss{These functions are for the purpose of specification.
  They are not necessarily something that is going to be implemented
  explicitly.  Even if they are implemented, they are not exported; they only
  have local scope.}



\section{MIS of Solar Energy Absorption Module} \label{ModuleSEA} 

\subsection{Module}
Energy Absorption

\subsection{Uses}
input parameter, solar intensity, tilt angle,zenith angle\\

\subsection{Syntax}

\subsubsection{Exported Constants}

$I_{S}$ := 1.35\\
$I_{S_{\text{total}}}$: real\\
$I_{S_{\text{total}}}$:= intenSum($I_{S}$)\\
$I_{S_{\text{max}}}$ : real\\
$I_{S_{\text{max}}}$
 := intenSingle($I_{S}$, angle( ))\\
$\theta_{S_{\text{date}}}$: a sequence of degree\\
$\theta_{S_{\text{date}}}$ := zenith( )\\
\subsubsection{Exported Access Programs}

\begin{center}
\begin{tabular}{p{2cm} p{4cm} p{4cm} p{2cm}}
\hline
\textbf{Name} & \textbf{In} & \textbf{Out} & \textbf{Exceptions} \\
\hline 
energy & - & real[ ] & - \\

\hline
\end{tabular}
\end{center}


\subsection{Semantics}

\subsubsection{State Variables}
$I_{S_{\text{daily}}}$: real[ ]

\subsubsection{Environment Variables}

N/A

\subsubsection{Assumptions}


\noindent  energy( ):
\begin{itemize}
\item transition: $I_{S_{\text{daily}}}$ := ($\forall$ x : degree $|$ x $\in$ $\theta_{S_{\text{date}}}$ $\bullet$ intenSingle($I_{S_{\text{max}}}$, x))

\item output: out := ( $\forall$ x: real $|$ x $\in$ $I_{S_{\text{daily}}}$ $\bullet$ $P_{A_{\text{w}}}$ $\times$ $P_{A_{\text{h}}}$ $\times$ 18.7 $\times$ 0.75 $\times$ x)
 
\item exception: 
\end{itemize}


\subsubsection{Local Functions}


\section{MIS of Sun Intensity Equation Module} \label{ModuleSI} 
\wss{Use labels for cross-referencing}

\wss{You can reference SRS labels, such as R\ref{R_Inputs}.}

\wss{It is also possible to use \LaTeX for hypperlinks to external documents.}

\subsection{Module}
Sun Intensity

\subsection{Uses}
zenith angle

\subsection{Syntax}

\subsubsection{Exported Constants}
$I_{S}$ := 1.35\\
$\theta_{S_{\text{date}}}$: a squence of degree\\
$\theta_{S_{\text{date}}}$ := zenith( )

\subsubsection{Exported Access Programs}

\begin{center}
\begin{tabular}{p{2cm} p{4cm} p{4cm} p{2cm}}
\hline
\textbf{Name} & \textbf{In} & \textbf{Out} & \textbf{Exceptions} \\
\hline 
intenSum & real & real & - \\
intenSingle & real, degree & real & - \\

\hline
\end{tabular}
\end{center}


\subsection{Semantics}

\subsubsection{State Variables}


\subsubsection{Environment Variables}

N/A

\subsubsection{Assumptions}


\noindent intenSum( i ):
\begin{itemize}
\item output: out := +($\forall$ d: degree $|$ d $\in$ $\theta_{S_{\text{date}}}$ $\bullet$ $I_{S} \cdot \frac{1.00}{i}^{sec(d)} $)
\item exception: 
\end{itemize}

\noindent intenSingle(i, d):
\begin{itemize}
\item output: out := $1.35 \cdot \frac{1.00}{i}^
{sec(d)} $
\item exception: 
\end{itemize}

\subsubsection{Local Functions}


\section{MIS of Zenith Angle Equation Module} \label{ModuleZA} 

\subsection{Module}
Zenith angle


\subsection{Uses}
Date Duration Module\ref{ModuleDD}


\subsection{Syntax}

\subsubsection{Exported Constants}
$\Phi_{P}$: degree\\
$\delta_{\text{date}}$: a squence of degree\\
$\delta_{\text{date}}$:= declination( ) 

\subsubsection{Exported Access Programs}

\begin{center}
\begin{tabular}{p{2cm} p{4cm} p{4cm} p{2cm}}
\hline
\textbf{Name} & \textbf{In} & \textbf{Out} & \textbf{Exceptions} \\
\hline 
zenith &  -  & degree[ ] & - \\

\hline
\end{tabular}
\end{center}


\subsection{Semantics}

\subsubsection{State Variables}

zenithS: a squence of degree


\subsubsection{Environment Variables}

N/A

\subsubsection{Assumptions}




\noindent \wss{accessProg}zenith( ):
\begin{itemize}
\item transition: zenithS := ($\forall$ d: degree $|$ d $\in$ $\delta_{\text{date}}$ $\bullet$ $\Phi_{P}$ $\times$ d $\geq$ 0 $\Rightarrow$ $\Phi_{P}$ - d $|$ otherwise $\Rightarrow$ $\Phi_{P}$ + d)

\item output: out := self
\item exception: 
\end{itemize}



\subsubsection{Local Functions}


\newpage

\section{MIS of Sun Declination Module} \label{ModuleSD} 

\subsection{Module}
Sun Declination


\subsection{Uses}
date duration


\subsection{Syntax}

\subsubsection{Exported Constants}
date: a squence of integer\\
date := dateduration( )

\subsubsection{Exported Access Programs}

\begin{center}
\begin{tabular}{p{2cm} p{4cm} p{4cm} p{2cm}}
\hline
\textbf{Name} & \textbf{In} & \textbf{Out} & \textbf{Exceptions} \\
\hline 
intidec & - & AnalemmaT[ ] & -\\
declination & - & degree[ ] & - \\

\hline
\end{tabular}
\end{center}


\subsection{Semantics}

\subsubsection{State Variables}

declinationS: a squence of degree\\
initdec: a squence of AnalemmaT

\subsubsection{Environment Variables}

file: Input a file that contains a sequence of (x: real, y: real, z: real) of 366 days



\subsubsection{Assumptions}



\noindent  intidec( ):
\begin{itemize}
\item transition: initdec := ($\forall$ d: integer,$\forall$ i: integer $|$ d $\in$ date, i $\in$  [0...] $\bullet$ dec.z[i] := file[d].$z$, dec.x[i] := file[d].$x$,dec.y[i] := file[d].$y$)

\item output: self
\item exception: 
\end{itemize}

\noindent  declination( ):
\begin{itemize}
\item transition: declinationS := ($\forall$: i:DayT $|$ i $\in$ intidec( ) $\bullet$ $\arcsin \frac{i.z}{ \sqrt{i.x^2 + i.y^2 + i.z^2}}$)

\item output: self
\item exception: 
\end{itemize}
 

\subsubsection{Local Functions}

\section{MIS of Days Module} \label{ModuleDD} 
\subsection{Module}
Days

\subsection{Uses}
Input parameter

\subsection{Syntax}

\subsubsection{Exported Constants}

end: DayT\\
end.d = $\mathit{day}_\text{End}$\\
end.m = $\mathit{month}_\text{End}$\\
end.y = $\mathit{year}_\text{End}$\\

\subsubsection{Exported Access Programs}

\begin{center}
\begin{tabular}{p{2cm} p{4cm} p{4cm} p{2cm}}
\hline
\textbf{Name} & \textbf{In} & \textbf{Out} & \textbf{Exceptions} \\
\hline 
perihelion & DayT & DayT & - \\
dateduration & - & integer[ ] & - \\


\hline
\end{tabular}
\end{center}
 (day.m = 12 $	\wedge$ day.d $\geq	$ 21 $\Rightarrow$ fixday.y = day.y $|$ otherwise $\Rightarrow$ fixday.y := day.y - 1)

\subsection{Semantics}

\subsubsection{State Variables}
days : a sequence of integer

fixday : DayT

start: DayT\\
start.d := $\mathit{day}_\text{Start}$\\
start.m := $\mathit{month}_\text{Start}$\\
start.y := $\mathit{year}_\text{Start}$\\

\subsubsection{Environment Variables}

None

\subsubsection{Assumptions}

\noindent perihelion(day):
\begin{itemize}
\item transition: fixday :=  (day.m = 12 $	\wedge$ day.d $\geq	$ 21 $\Rightarrow$ fixday.d = 21, fixday.m = 12, fixday.y = day.y $|$ otherwise $\Rightarrow$ fixday.d = 21, fixday.m = 12, fixday.y := day.y - 1)
\item output: out := fixday
\item exception: 
\end{itemize}

\noindent dateduration(day):
\begin{itemize}
\item transition: days := ( negetiveD(start, end) = 1 $\Rightarrow$  countdays(perihelion(start), start)), start := adday(start)
\item output: days
\item exception: 
\end{itemize}


\subsubsection{Local Functions}
negetiveD: DayT $\times$ DayT $\rightarrow $ integer\\
negetiveD(day1, day2) := (countdays(day1, day2) $\geq$ 0 $\Rightarrow$ 1 $|$ otherwise $\Rightarrow$ -1)

\section{MIS of Count Days Module} \label{ModuleDD} 
\subsection{Module}
Count days


\subsection{Uses}
N/A

\subsection{Syntax}

\subsubsection{Exported Constants}


\subsubsection{Exported Access Programs}

\begin{center}
\begin{tabular}{p{2cm} p{4cm} p{4cm} p{2cm}}
\hline
\textbf{Name} & \textbf{In} & \textbf{Out} & \textbf{Exceptions} \\
\hline 
countdays & DayT, DayT & integer & - \\
addays & DayT & DayT & - \\

\hline
\end{tabular}
\end{center}

\subsection{Semantics}

\subsubsection{State Variables}



\subsubsection{Environment Variables}

None

\subsubsection{Assumptions}



\noindent countdays(day1, day2):
\begin{itemize}
\item transition:
\item output: out := Count the days from the start  date, day, to the end date, day, but not including the end date.
\item exception: 
\end{itemize}

\noindent adday(day):
\begin{itemize}
\item transition: 
\item output: out := the next day of the date, day.
\item exception: 
\end{itemize}


\subsubsection{Local Functions}



\newpage

\bibliographystyle {plainnat}
\bibliography {../../../refs/References}

\newpage

\section{Appendix} \label{Appendix}

\wss{Extra information if required}

\end{document}