\documentclass[12pt, titlepage]{article}

\usepackage{fullpage}
\usepackage[round]{natbib}
\usepackage{multirow}
\usepackage{booktabs}
\usepackage{tabularx}
\usepackage{graphicx}
\usepackage{float}
\usepackage{xr}
\usepackage{hyperref}
\externaldocument{../../SRS/SRS}

\hypersetup{
    colorlinks,
    citecolor=black,
    filecolor=black,
    linkcolor=red,
    urlcolor=blue
}

\input{../../Comments}
%% Common Parts

\newcommand{\progname}{Sun Catcher} % PUT YOUR PROGRAM NAME HERE %Every program
                                % should have a name


\newcounter{acnum}
\newcommand{\actheacnum}{AC\theacnum}
\newcommand{\acref}[1]{AC\ref{#1}}

\newcounter{ucnum}
\newcommand{\uctheucnum}{UC\theucnum}
\newcommand{\uref}[1]{UC\ref{#1}}

\newcounter{mnum}
\newcommand{\mthemnum}{M\themnum}
\newcommand{\mref}[1]{M\ref{#1}}

\begin{document}

\title{Module Guide for \progname{}} 
\author{Sharon (Yu-Shiuan) Wu}
\date{\today}

\maketitle

\pagenumbering{roman}

\section{Revision History}

\begin{tabularx}{\textwidth}{p{3cm}p{2cm}X}
\toprule {\bf Date} & {\bf Version} & {\bf Notes}\\
\midrule
2019/11/25 & 1.0 & First version update to Git\\
Date 2 & 1.1 & Notes\\
\bottomrule
\end{tabularx}

\newpage

\section{Reference Material}

This section records information for easy reference.

\subsection{Abbreviations and Acronyms}

\renewcommand{\arraystretch}{1.2}
\begin{tabular}{l l} 
  \toprule		
  \textbf{symbol} & \textbf{description}\\
  \midrule 
  AC & Anticipated Change\\
  DAG & Directed Acyclic Graph \\
  M & Module \\
  MG & Module Guide \\
  OS & Operating System \\
  R & Requirement\\
  SC & Scientific Computing \\
  SRS & Software Requirements Specification\\
  \progname & Explanation of program name\\
  UC & Unlikely Change \\
  \wss{etc.} & \wss{...}\\
  \bottomrule
\end{tabular}\\

\newpage

\tableofcontents

\listoftables

\listoffigures

\newpage

\pagenumbering{arabic}

\section{Introduction}

Decomposing a system into modules is a commonly accepted approach to developing
software.  A module is a work assignment for a programmer or programming
team~\citep{ParnasEtAl1984}.  We advocate a decomposition
based on the principle of information hiding~\citep{Parnas1972a}.  This
principle supports design for change, because the ``secrets'' that each module
hides represent likely future changes.  Design for change is valuable in SC,
where modifications are frequent, especially during initial development as the
solution space is explored.  

Our design follows the rules layed out by \citet{ParnasEtAl1984}, as follows:
\begin{itemize}
\item System details that are likely to change independently should be the
  secrets of separate modules.
\item Each data structure is implemented in only one module.
\item Any other program that requires information stored in a module's data
structures must obtain it by calling access programs belonging to that
module.\end{itemize}

After completing the first stage of the design, the Software Requirements
Specification (SRS), the Module Guide (MG) is developed~\citep{ParnasEtAl1984}.
The MG
specifies the modular structure of the system and is intended to allow both
designers and maintainers to easily identify the parts of the software.  The
potential readers of this document are as follows:

\begin{itemize}
\item New project members: This document can be a guide for a new project
member to easily understand the overall structure and quickly find the
  relevant modules they are searching for.
\item Maintainers: The hierarchical structure of the module guide improves the
maintainers' understanding when they need to make changes to the system. It is
important for a maintainer to update the relevant sections of the document
  after changes have been made.
\item Designers: Once the module guide has been written, it can be used to
  check for consistency, feasibility and flexibility. Designers can verify the
  system in various ways, such as consistency among modules, feasibility of the
  decomposition, and flexibility of the design.
\end{itemize}

The rest of the document is organized as follows. Section
\ref{SecChange} lists the anticipated and unlikely changes of the software
requirements. Section \ref{SecMH} summarizes the module decomposition that
was constructed according to the likely changes. Section \ref{SecConnection}
specifies the connections between the software requirements and the
modules. Section \ref{SecMD} gives a detailed description of the
modules. Section \ref{SecTM} includes two traceability matrices. One checks
the completeness of the design against the requirements provided in the SRS.
The other shows the relation between anticipated changes and the modules.
Section\ref{SecUse} describes the use relation between modules.

\section{Anticipated and Unlikely Changes} \label{SecChange}

This section lists possible changes to the system. According to the likeliness
of the change, the possible changes are classified into two
categories. Anticipated changes are listed in Section \ref{SecAchange}, and
unlikely changes are listed in Section \ref{SecUchange}.

\subsection{Anticipated Changes} \label{SecAchange}

Anticipated changes are the source of the information that is to be hidden
inside the modules. Ideally, changing one of the anticipated changes will only
require changing the one module that hides the associated decision. The
approach adopted here is called design for
change.

\begin{description}
\item[\refstepcounter{acnum} \actheacnum \label{acH}:] The specific
  hardware on which the software is running.
\item[\refstepcounter{acnum} \actheacnum \label{acIP}:] The initial of the
  input parameter. \wss{The initial what of the input parameter?}
\item[\refstepcounter{acnum} \actheacnum \label{acIC}:] The constraint of the input data.
\item[\refstepcounter{acnum} \actheacnum \label{acO}:] The initial of the
  output parameter. \wss{The initial?}
\item[\refstepcounter{acnum} \actheacnum \label{acR}:] The initial of the
  parameter of the read file.
\item[\refstepcounter{acnum} \actheacnum \label{acSEA}:] The algorithm of the
 solar energy absorption.
\item[\refstepcounter{acnum} \actheacnum \label{acOTP}:] The algorithm of the 
optimum tilt angle.
\item[\refstepcounter{acnum} \actheacnum \label{acSI}:] The equation of the sun
intensity.
\item[\refstepcounter{acnum} \actheacnum \label{acZA}:] The algorithm of the
 zenith angle.
\item[\refstepcounter{acnum} \actheacnum \label{acDD}:] The structure of the 
day duration ADT.
\item[\refstepcounter{acnum} \actheacnum \label{acT}:] The structure of 
the table-layout.
\item[\refstepcounter{acnum} \actheacnum \label{acSCT}:]  The structure of 
sun catcher type.
\end{description}

\subsection{Unlikely Changes} \label{SecUchange}

The module design should be as general as possible. However, a general system
is more complex. Sometimes this complexity is not necessary. Fixing some design
decisions at the system architecture stage can simplify the software design. If
these decisions should later need to be changed, then many parts of the design
will potentially need to be modified. Hence, it is not intended that these
decisions will be changed.

\begin{description}
\item[\refstepcounter{ucnum} \uctheucnum \label{ucIO}:] Input/Output devices
  (Input: File and/or Keyboard, Output: File, Memory, and/or Screen).
\item[\refstepcounter{ucnum} \uctheucnum \label{ucIO}:] The goal of \progname
is always predict the optimum title angle and the solar energy absorption and show the difference of the solar energy absorptions.
\end{description}

\section{Module Hierarchy} \label{SecMH}

This section provides an overview of the module design. Modules are summarized
in a hierarchy decomposed by secrets in Table \ref{TblMH}. The modules listed
below, which are leaves in the hierarchy tree, are the modules that will
actually be implemented.

\begin{description}
\item [\refstepcounter{mnum} \mthemnum \label{mHH}:] Hardware Hiding Modules
\item [\refstepcounter{mnum} \mthemnum \label{mIC}:] Control Module
\item [\refstepcounter{mnum} \mthemnum \label{mIP}:] Input Parameters Module
\item [\refstepcounter{mnum} \mthemnum \label{mIV}:] Input Verification Module
\item [\refstepcounter{mnum} \mthemnum \label{mOP}:] Output Parameters Module
Module\item [\refstepcounter{mnum} \mthemnum \label{mR}:] Read File Module
\item [\refstepcounter{mnum} \mthemnum \label{mSE}:] Solar Energy Absorption Module
\item [\refstepcounter{mnum} \mthemnum \label{mOTA}:] Optimum Tilt Angle Module
\item [\refstepcounter{mnum} \mthemnum \label{mSIE}:] Sun Intensity Equation
Module
\item [\refstepcounter{mnum} \mthemnum \label{mZAE}:] Zenith Angle Equation
Module
\item [\refstepcounter{mnum} \mthemnum \label{mDD}:] Day Duration ADT Module
\item [\refstepcounter{mnum} \mthemnum \label{mT}:] Table-layout Module
\item [\refstepcounter{mnum} \mthemnum \label{mSCT}:] Sun Catcher Type Module
\end{description}


\begin{table}[h!]
\centering
\begin{tabular}{p{0.3\textwidth} p{0.6\textwidth}}
\toprule
\textbf{Level 1} & \textbf{Level 2}\\
\midrule

{Hardware-Hiding Modules} & ~ \\
\midrule

\multirow{7}{0.3\textwidth}{Behaviour-Hiding
 Module}& Control Module\\
& Input Parameters Module\\
& Input Verification Module\\
& Output Parameters Module\\
& Read File Module\\
& Solar Energy Absorption Module\\
& Optimum Tilt Angle Module\\
& Sun Intensity Equation Module\\
& Zenith Angle Equation Module\\
\midrule

\multirow{3}{0.3\textwidth}{Software Decision Module} 
& Day Duration ADT Module\\
& Table-layout Module\\
& Sun Catcher Type 
Module\\
\bottomrule

\end{tabular}
\caption{Module Hierarchy}
\label{TblMH}
\end{table}




\section{Connection Between Requirements and Design} \label{SecConnection}

The design of the system is intended to satisfy the requirements developed in
the SRS. In this stage, the system is decomposed into modules. The connection
between requirements and modules is listed in Table \ref{TblRT}.

\wss{The intention of this section is to document decisions that are made
  ``between'' the requirements and the design.  To satisfy some requirements,
  design decisions need to be made.  Rather than make these decisions implicit,
  they are explicitly recorded here.  For instance, if a program has security
  requirements, a specific design decision may be made to satisfy those
requirements with a password. In scientific examples, the choice of
algorithm could potentially go here, if that is a decision that is exposed by
the interface.}

\section{Module Decomposition} \label{SecMD}

Modules are decomposed according to the principle of ``information hiding''
proposed by \citet{ParnasEtAl1984}. The \emph{Secrets} field in a module
decomposition is a brief statement of the design decision hidden by the
module. The \emph{Services} field specifies \emph{what} the module will do
without documenting \emph{how} to do it. For each module, a suggestion for the
implementing software is given under the \emph{Implemented By} title. If the
entry is \emph{OS}, this means that the module is provided by the operating
system or by standard programming language libraries. \emph{\progname{}} means
the
module will be implemented by the \progname{} software.

Only the leaf modules in the hierarchy have to be implemented. If a dash
(\emph{--}) is shown, this means that the module is not a leaf and will not
have to be implemented.

\subsection{Hardware Hiding Modules (\mref{mHH})}

\begin{description}
\item[Secrets:]The data structure and algorithm used to implement the virtual
  hardware.
\item[Services:]Serves as a virtual hardware used by the rest of the
  system. This module provides the interface between the hardware and the
  software. So, the system can use it to display outputs or to accept inputs.
\item[Implemented By:] OS
\end{description}

\subsection{Behaviour-Hiding Module}

\begin{description}
\item[Secrets:]The contents of the required behaviours.
\item[Services:]Includes programs that provide externally visible behaviour of
  the system as specified in the software requirements specification (SRS)
  documents. This module serves as a communication layer between the
  hardware-hiding module and the software decision module. The programs in this
  module will need to change if there are changes in the SRS.
\item[Implemented By:] --
\end{description}

\subsubsection{Control Module (\mref{mC}) \wss{Nothing is currently labelled as
  mC, so this reference does not work.}}

\begin{description}
\item[Secrets:] The structure that modules connect to each other.
\item[Services:] Implement the designed modules.
\item[Implemented By:] \progname
\end{description}

\subsubsection{Input Parameters Module (\mref{mIP})}

\begin{description}
\item[Secrets:] The paramenters \wss{spell check} of inputs.
\item[Services:] Assigns the input values to input paramenters \wss{spell check}.

\item[Implemented By:] \progname
\end{description}


\subsubsection{Input Verification Module (\mref{mIV})}

\begin{description}
\item[Secrets:] The constraint of the input values.
\item[Services:] Verfies \wss{spell check!} the input values
\item[Implemented By:] \progname
\end{description}

\subsubsection{Output Parameters Module (\mref{mOP})}

\begin{description}
\item[Secrets:] The parameters of outputs.
\item[Services:] Assigns the calculated results to output parameters.
\item[Implemented By:] \progname
\end{description}

\subsubsection{Read File Module (\mref{mR})}

\begin{description}
\item[Secrets:] The parameters for reading the file.
\item[Services:] Assgins \wss{spell check} the input file to the parameter.
\item[Implemented By:] \progname
\end{description}

\subsubsection{Solar Energy Absorption Module (\mref{mSEA})}

\begin{description}
\item[Secrets:]The algorithm of calculating solar energy absorption.
\item[Services:] Calculates solar energy absorption.

\item[Implemented By:] \progname
\end{description}


\subsubsection{Optimum Tilt Angle Module (\mref{mOTA})}

\begin{description}
\item[Secrets:]The algorithm of calculating optimum tilt angle and optimum sun intensity associated with the optimum tilt angle.
\item[Services:] Calculates the optimum tilt angle and optimum sun intensity.
\item[Implemented By:] \progname
\end{description}

\subsubsection{Sun Intensity Equation Module (\mref{mSIE})}

\begin{description}
\item[Secrets:]The equation of calculating the sun intensity.
\item[Services:] Calculates the sun intensity.
\item[Implemented By:] \progname
\end{description}

\subsubsection{Zenith Angle Equation Module (\mref{mZAE})}

\begin{description}
\item[Secrets:]The algorithm of calculating the zenith angle.
\item[Services:] Calculates the zenith angle.
\item[Implemented By:] \progname

\end{description}


\subsection{Software Decision Module}

\begin{description}
\item[Secrets:] The design decision based on mathematical theorems, physical
  facts, or programming considerations. The secrets of this module are
  \emph{not} described in the SRS.
\item[Services:] Includes data structure and algorithms used in the system that
  does not provide direct interaction with the user. 
  % Changes in these modules are more likely to be motivated by a desire to
  % improve performance than by externally imposed changes.
\item[Implemented By:] --
\end{description}


\subsubsection{Day Duration ADT Module (\mref{mDD})}

\begin{description}
\item[Secrets:] The structure of defining day duration's type and the algorithm of calculating the day duration.
\item[Services:] Creates variables with day duration's type and calculates the day duration.
\item[Implemented By:] Data.Time
\end{description}

\subsubsection{Table-layout Module (\mref{mP})}

\begin{description}
\item[Secrets:] The structue \wss{spell check} of the table layout.
\item[Services:] Creats \wss{spell check - there are many spelling mistakes in
    this document} a table.
\item[Implemented By:] Text.Layout.Table
\end{description}
\subsubsection{Etc.}

\subsubsection{Sun Catcher Type Module (\mref{mP})}

\begin{description}
\item[Secrets:] The structue of defining the sun catcher's type.
\item[Services:] Creats variables with sun catcher's type.
\item[Implemented By:] \progname
\end{description}
\subsubsection{Etc.}


\section{Traceability Matrix} \label{SecTM}

This section shows two traceability matrices: between the modules and the
requirements and between the modules and the anticipated changes.

% the table should use mref, the requirements should be named, use something
% like fref
\begin{table}[H]
\centering
\begin{tabular}{p{0.2\textwidth} p{0.6\textwidth}}
\toprule
\textbf{Req.} & \textbf{Modules}\\
\midrule
R1 & \mref{mHH}, \mref{mIP}, \mref{mR}, \mref{mIC}\\
R2 & \mref{mIV}, \mref{mIC}\\
R3 & \mref{mOP}, \mref{mT}, \mref{mIC}\\
R4 & \mref{mIC}\\
R5 & \mref{mSIE}, \mref{mZAE}, \mref{mIC}\\
R6 & \mref{mSE}, \mref{mSIE}, \mref{mZAE}, \mref{mIC}\\
R7 &\mref{mOP}, \mref{mIC}\\
R8 & \mref{mT}, \mref{mIC}\\
R9 & \mref{mIC}\\
\bottomrule
\end{tabular}
\caption{Trace Between Requirements and Modules}
\label{TblRT}
\end{table}

\begin{table}[H]
\centering
\begin{tabular}{p{0.2\textwidth} p{0.6\textwidth}}
\toprule
\textbf{AC} & \textbf{Modules}\\
\midrule
\acref{acH} & \mref{mHH}\\
\acref{acIP} & \mref{mIP}\\
\acref{acIC} & \mref{mIV}\\
\acref{acO} & \mref{mOP}\\
\acref{acR} & \mref{mR}\\
\acref{acSEA} & \mref{mSE}\\
\acref{acOTP} & \mref{mOTA}\\
\acref{acSI} & \mref{mSIE}\\
\acref{acZA} & \mref{mZAE}\\
\acref{acDD} & \mref{mDD}\\
\acref{acT} & \mref{mT}\\
\acref{acSCT} & \mref{mSCT}\\
\bottomrule
\end{tabular}
\caption{Trace Between Anticipated Changes and Modules}
\label{TblACT}
\end{table}

\section{Use Hierarchy Between Modules} \label{SecUse}

In this section, the uses hierarchy between modules is
provided. \citet{Parnas1978} said of two programs A and B that A {\em uses} B
if correct execution of B may be necessary for A to complete the task described
in
its specification. That is, A {\em uses} B if there exist situations in which
the correct functioning of A depends upon the availability of a correct
implementation of B.  Figure \ref{FigUH} illustrates the use relation between
the modules. It can be seen that the graph is a directed acyclic graph
(DAG). Each level of the hierarchy offers a testable and usable subset of the
system, and modules in the higher level of the hierarchy are essentially
simpler because they use modules from the lower levels.

\begin{figure}[H]
	\center
  \includegraphics[scale=0.5]{Fig_Hierarchy.jpg}
 \caption{\label{FigUH} Use hierarchy among modules}
\end{figure}

\wss{I suggest combining Read File with Input Parameters.  Also, as it is, the
  Read File does not use Input Parameters, so there is not way for the Read File
  module to change the state of the inputs.  Zenith Angle shouldn't really need
  to use Read File.  If some of the Input Parameters are dates, then why doesn't
  the Input Parameter module use the Day Duration ADT?  I also wonder if the Sun
  Intensity and Zenith Angle modules should be combined?}

%\section*{References}

\bibliographystyle {plainnat}
\bibliography{../../../refs/References}

\wss{Your references are not working.}

\end{document}